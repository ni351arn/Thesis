\begin{table}[H]
\caption{Übersicht zum Muster Automatisierte Release-Absicherung}
\label{tab:automatisierteReleaseAbsicherung}
\footnotesize
\renewcommand{\arraystretch}{1.2}
\setlength{\tabcolsep}{7pt}

\begin{tabularx}{\linewidth}{p{3.5cm} X}
\toprule
\textbf{Automatisierte Release-Absicherung} & \textbf{Beschreibung} \\
\midrule

Konkreter Plattform-Task &
Automatisiertes Rollback bei Canary-Deployments (schrittweise Ausrollung) durch Überwachung von Abweichungen in Latenz und Fehlerraten (Release-Gating).
\\

Empfohlene KI-Methode &
Unsupervised Learning (Autoencoder) zur Anomalieerkennung oder Reinforcement Learning (z.B.\ DQN) zur Optimierung des Rollback-Zeitpunkts \cite{tamanampudiAIEnhancedContinuousIntegration}.
\\

KPI/Mehrwert &
MTTR (Mean Time To Recovery) bis zu -40\%; Systemverfügbarkeit bis zu +18\%; Vorfallerkennung um bis zu 35\% beschleunigt \cite{kankanalaAIMLDevOps2024,jossonpaulkalapparambathAdvancingDistributedSystems2025, tamanampudiAIEnhancedContinuousIntegration}.
\\

Limitationen &
Risiko von False Positives (unnötige Rollbacks); zusätzlicher Validierungs- und Governance-Aufwand bei RL-Entscheidungen (z.B.\ Nachvollziehbarkeit/Erklärbarkeit).
\\

Einsatzvoraussetzungen &
Etablierte Baseline für Normalverhalten in Service-Mesh-gestützten Telemetriedaten (Traffic-Steuerung \& Observability; Latenz, Fehlerraten, SLOs) sowie automatisierte Rollback-Mechanismen und Echtzeit-Streaming der Deployment-Metriken.
\\

\bottomrule
\end{tabularx}
\end{table}
