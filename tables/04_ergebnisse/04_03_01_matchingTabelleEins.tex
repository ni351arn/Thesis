\begin{table}[!htbp]
\caption{Übersicht zum Muster Intelligente Betriebsautomatisierung und Orchestrierung}
\label{tab:betriebOrchestrierung}
\footnotesize
\renewcommand{\arraystretch}{1.2}
\setlength{\tabcolsep}{7pt}

\begin{tabularx}{\linewidth}{p{3.5cm} X}
\toprule
\textbf{Intelligente Betriebsautomatisierung und Orchestrierung} & \textbf{Beschreibung} \\
\midrule

Konkrete Plattform-\newline aufgabe &
\gls{ki}-gestützte Bewertung von Betriebszuständen im Clusterbetrieb, inklusive Ursachenhinweisen und empfohlenen Maßnahmen.
\\

Empfohlene \gls{ki}-Methode &
Sprachmodelle zur Analyse von Betriebszuständen, \gls{dl} zur Vorhersage von Lastverläufen sowie \gls{rl} zur Optimierung operativer Maßnahmen \cite{zaaloukCLOUDNATIVEARTIFICIAL,enemosahEnhancingDevOpsEfficiency2025,jossonpaulkalapparambathAdvancingDistributedSystems2025}.
\\

KPI/Mehrwert &
52 \% weniger manuelle Betriebseingriffe, 40 \% schnellere Fehlerbehebung (MTTR) sowie ca. 30 \% Kostenreduktion \cite{tamminediAutomatingKubernetesOperations2024,jossonpaulkalapparambathAdvancingDistributedSystems2025}.
\\

Limitationen &
Hohe Modellkomplexität, eingeschränkte Nachvollziehbarkeit sowie hoher Rechenaufwand für Training und Betrieb.
\\

Einsatzvoraussetzungen &
Durchgängige Beobachtbarkeits-Pipeline (Protokolle und Metriken), hochwertige historische Betriebsdaten sowie automatisierbarer Zugriff auf die Kubernetes-API \cite{zaaloukCLOUDNATIVEARTIFICIAL,reddygopireddyIntegratingAIDevOps2022}.
\\

Konkrete Tools &
K8sGPT, KEDA, Dynatrace, Sentry \cite{zaaloukCLOUDNATIVEARTIFICIAL,supritpattanayakIntegratingAIDevOps2024}.
\\

\bottomrule
\end{tabularx}
\end{table}
