\begin{table}[H]
\caption{Übersicht zum Muster Proaktives Ressourcen-Management}
\label{tab:proaktivesRessourcenmanagement}
\footnotesize
\renewcommand{\arraystretch}{1.2}
\setlength{\tabcolsep}{7pt}

\begin{tabularx}{\linewidth}{p{3.5cm} X}
\toprule
\textbf{Proaktives Ressourcen-Management} & \textbf{Beschreibung} \\
\midrule

Konkreter Plattform-Task &
Automatisierte Anpassung von Instanztypen (EC2/RDS) oder Kubernetes-Pods (HPA) basierend auf der Vorhersage von CPU-/RAM-Spitzen.
\\

Empfohlene KI-Methode &
Deep Learning (z..B. LSTM) zur Zeitreihenprognose oder Reinforcement Learning zur dynamischen Lastverteilung in Echtzeit.
\\

KPI/Mehrwert &
Ressourcenauslastung bis zu +25 \% (RL vs.\ Heuristik) \cite{poudelAIDrivenIntelligentAutoScaling2025a},
Kosteneffizienz durch dynamische Allokation mit bis zu 55 \% geringeren Cloud-Kosten \cite{karthikputhrayaRoleCloudNativeArchitectures2025}.
\\

Limitationen &
Hoher Rechenaufwand für das Modelltraining (z..B. LSTMs), Cold-Start-Latenzen in Serverless-Umgebungen, benötigt ca.\ 2 Monate historische Daten \cite{amteCloudNativeAIChallenges2025,poudelAIDrivenIntelligentAutoScaling2025a}.
\\

Einsatzvoraussetzungen &
Mind.\ 2 Monate historische Zeitreihen-Telemetrie (feingranular, z..B. 1-15 Minuten) aus CloudWatch/Prometheus sowie automatisierter API-Schreibzugriff für Instanz- bzw.\ Skalierungsanpassungen \cite{poudelAIDrivenIntelligentAutoScaling2025a}.
\\

Konkrete Tools &
AWS CloudWatch, Boto3 SDK, Kubernetes HPA, KEDA (Event-Driven Autoscaling), Knative, Megalix \cite{bajwaCLOUDNATIVEARCHITECTURESLARGESCALE2025, zaaloukCLOUDNATIVEARTIFICIAL, poudelAIDrivenIntelligentAutoScaling2025a, supritpattanayakIntegratingAIDevOps2024}.
\\

\bottomrule
\end{tabularx}
\end{table}
