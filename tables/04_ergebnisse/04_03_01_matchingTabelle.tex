% (Im Dokument an der Stelle, wo die Tabelle hin soll)

\clearpage % wichtig: neue Seite starten

\begin{landscape}
\begin{table}[p]
\begin{flushleft} % startet links (nicht zentriert)

\caption{Übersicht der KI-Anwendungsfelder: Tasks, Daten, Methoden und Ziele}
\label{tab:anwendungsfelder-5x5}

\footnotesize
\renewcommand{\arraystretch}{1.15}
\setlength{\tabcolsep}{4pt}

\begin{tabularx}{\linewidth}{p{3.6cm} Y Y Y Y}
\toprule
\textbf{Anwendungsfelder} &
\textbf{Konkrete Plattform-Task} &
\textbf{Genutzte Daten} &
\textbf{KI-Methode} &
\textbf{Ziel} \\
\midrule

Ressourcenoptimierung
& \emph{Vorausschauendes Auto-Scaling von Cloud-Ressourcen (z. B. AWS EC2, RDS, oder Kubernetes-Cluster), Dynamische Anpassung der VM-Konfigurationen und Right-sizing.")}
& \emph{"Datenquelle: Historische Leistungsdaten (CPU-Auslastung, Netzwerk-Traffic, Disk I/O, Speicherkapazität). , Nutzung: Gesammelt über AWS CloudWatch oder Prometheus/Grafana (für Kubernetes). , Die Daten werden zur Vorhersage zukünftiger Ressourcenbedarfstrends und Workload-Muster genutzt."}
& \emph{Prädiktive Analytik (Zeitreihenanalyse). Spezifische Algorithmen: Long Short-Term Memory (LSTM) neuronale Netze. 
RL zur dynamischen Anpassung der Ressourcenzuteilung.}
& \emph{"Reduzierung der Betriebskosten. 
Steigerung der Ressourceneffizienz und -auslastung (z. B. 25\% Verbesserung der Ressourcenauslastung). 
Wartung der optimalen Anwendungs-Performance."} \\
\addlinespace[2pt]

Zuverlässigkeit / CI/CD Automation
& \emph{Prädiktive Fehlererkennung in Build- und Deployment-Pipelines. 
Automatisierte Rollbacks bei erkannten Anomalien. 
Adaptive Deployment-Strategien (z. B. dynamische Anpassung von Blue/Green- oder Canary-Rollouts). }
& \emph{Datenquelle: Historische Build-Logs (mit Zeitstempeln, Fehlermeldungen, Warnungen). 
Test-Ergebnisse (Pass/Fail-Raten). 
System-Performance-Metriken (Latenz, Fehlerraten, Ressourcennutzung während des Deployments). 
Nutzung: Analyse von Mustern, die auf bevorstehende Fehler hindeuten. Kontinuierliches Monitoring zur Auslösung von automatisierten Korrekturmaßnahmen.}
& \emph{SL für Fehler-Vorhersage(Random Forests, Entscheidungsbäume, Neuronale Netze). 
UL für Anomalieerkennung ( Autoencoder, Isolation Forests, K-Means Clustering).
RL für optimierte Rollbacks und Testpriorisierung. (Deep Q-Networks (DQN), PPO, Q-Learning)"}
& \emph{Reduzierung der Deployment-Zeit (z. B. 40\% Reduzierung der Testzeit). Reduzierung der Fehlerraten (Error Rate Reduction, z. B. 15\%). Minimierung der Downtime (z. B. 30\% Reduzierung durch proaktive Rollbacks). Verbesserung der Software-Zuverlässigkeit.} \\
\addlinespace[2pt]

Sicherheit / Bedrohungserkennung
& \emph{(eintragen: z.B. Detection, Alert Triage, Policy Violations, Secret/Leak Scans)}
& \emph{(eintragen: z.B. Security-Logs, Netzwerk-Flows, Audit-Logs, IAM-Events)}
& \emph{(eintragen: z.B. Anomalieerkennung, Clustering, Klassifikation)}
& \emph{(eintragen: z.B. Angriffe erkennen, False Positives senken, Compliance unterstützen)} \\
\addlinespace[2pt]

Betrieb / Orchestrierung
& \emph{(eintragen: z.B. Incident Response, Self-Healing, Runbook Automation, Triage)}
& \emph{(eintragen: z.B. Alerts, Tickets, SLO/SLI, Logs/Traces, Topologie-Daten)}
& \emph{(eintragen: z.B. Root-Cause-Ansätze, Empfehlungssysteme, Agenten/Workflow-Automation)}
& \emph{(eintragen: z.B. Stabilität erhöhen, Operator entlasten, schnellere Entstörung)} \\
\bottomrule
\end{tabularx}

\end{flushleft}
\end{table}
\end{landscape}

\clearpage % wichtig: danach wieder zurück ins Hochformat
