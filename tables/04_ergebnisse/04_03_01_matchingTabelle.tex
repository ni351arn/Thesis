\begin{table}[htbp]
\caption{Predictive Auto-Scaling \& Resource Optimization}
\label{tab:predictiveAutoScaling}
\footnotesize
\renewcommand{\arraystretch}{1.2}
\setlength{\tabcolsep}{6pt}

\begin{tabularx}{\linewidth}{p{4.2cm} Y Y}
\toprule
\textbf{Kategorie} & \textbf{Details} & \textbf{Beispielpaper} \\
\midrule

Plattform-Aufgabe &
\textit{Vorausschauende Skalierung von Instanzen (EC2, RDS) und Pod-Counts basierend auf Lastprognosen} &
\textit{\cite{poudelAIDrivenIntelligentAutoScaling2025a}, \cite{kankanalaAIMLDevOps2024}} \\

Datenquellen &
\textit{Metriken: CPU-Auslastung, Memory, Disk I/O, Netzwerk-Traffic (CloudWatch/Prometheus)} &
\textit{\cite{poudelAIDrivenIntelligentAutoScaling2025a}, \cite{kankanalaAIMLDevOps2024}} \\

Datenverwendung &
\textit{Analyse historischer Zeitreihen zur Identifikation von saisonalen Mustern und Lastspitzen} &
\textit{\cite{poudelAIDrivenIntelligentAutoScaling2025a}, \cite{kankanalaAIMLDevOps2024}} \\

KI-Methoden &
\textit{LSTM (Long Short-Term Memory) für Zeitreihen; Reinforcement Learning (Q-Learning, PPO) zur Optimierung der Ressourcenallokation} &
\textit{\cite{poudelAIDrivenIntelligentAutoScaling2025a}, \cite{jossonpaulkalapparambathAdvancingDistributedSystems2025}} \\

Konkrete Tools &
\textit{AWS Boto3 SDK, KEDA (Kubernetes Event-driven Autoscaling), TensorFlow/PyTorch für Modelltraining} &
\textit{\cite{poudelAIDrivenIntelligentAutoScaling2025a}, \cite{kankanalaAIMLDevOps2024}, \cite{zaaloukCLOUDNATIVEARTIFICIAL}} \\

Vorteile (Ziele) &
\textit{Reduzierung der Cloud-Kosten um bis zu 25-50\%; Vermeidung von Performance-Engpässen durch Skalierung vor dem Erreichen von Schwellenwerten} &
\textit{\cite{poudelAIDrivenIntelligentAutoScaling2025a}, \cite{karthikputhrayaRoleCloudNativeArchitectures2025}} \\

\bottomrule
\end{tabularx}
\end{table}
