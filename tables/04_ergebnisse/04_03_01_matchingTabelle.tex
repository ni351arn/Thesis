% --- Tabelle 1: Prädiktives Auto-Scaling & Ressourcenoptimierung ---
\begin{table}[H]
\caption{Prädiktives Auto-Scaling \& Ressourcenoptimierung}
\label{tab:predictiveAutoScaling}
\footnotesize
\renewcommand{\arraystretch}{1.2}
\setlength{\tabcolsep}{6pt}

\begin{tabularx}{\linewidth}{p{4.2cm} Y Y}
\toprule
\textbf{Kategorie} & \textbf{Beschreibung} & \textbf{Beispielquellen} \\
\midrule

Plattformaufgabe &
Vorausschauende Skalierung von Instanzen (EC2, RDS) und Pod-Anzahlen basierend auf Lastprognosen. &
\cite{poudelAIDrivenIntelligentAutoScaling2025a}, \cite{kankanalaAIMLDevOps2024} \\

Datenquellen &
Metriken: CPU-Auslastung, Arbeitsspeicher, Disk-I/O, Netzwerkverkehr (CloudWatch/Prometheus). &
\cite{poudelAIDrivenIntelligentAutoScaling2025a}, \cite{kankanalaAIMLDevOps2024} \\

Datennutzung &
Analyse historischer Zeitreihen zur Identifikation saisonaler Muster und Lastspitzen. &
\cite{poudelAIDrivenIntelligentAutoScaling2025a}, \cite{kankanalaAIMLDevOps2024} \\

KI-/ML-Methoden &
LSTM (Long Short-Term Memory) für Zeitreihen; Reinforcement Learning (Q-Learning, PPO) zur Optimierung der Ressourcenzuteilung. &
\cite{poudelAIDrivenIntelligentAutoScaling2025a}, \cite{jossonpaulkalapparambathAdvancingDistributedSystems2025} \\

Werkzeuge (Tools) &
AWS Boto3 SDK, KEDA (Kubernetes Event-driven Autoscaling), TensorFlow/PyTorch für das Modelltraining. &
\cite{poudelAIDrivenIntelligentAutoScaling2025a}, \cite{kankanalaAIMLDevOps2024}, \cite{zaaloukCLOUDNATIVEARTIFICIAL} \\

Ziele/Nutzen &
Reduzierung der Cloud-Kosten um bis zu 25--50\%; Vermeidung von Performance-Engpässen durch Skalierung vor dem Erreichen von Schwellenwerten. &
\cite{poudelAIDrivenIntelligentAutoScaling2025a}, \cite{karthikputhrayaRoleCloudNativeArchitectures2025} \\

\bottomrule
\end{tabularx}
\end{table}
