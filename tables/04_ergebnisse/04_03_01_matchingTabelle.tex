\begin{landscape}
\begin{table}[p]
  \centering
  \caption{Matching-Übersicht}
  \label{tab:matching-usecases}

  \footnotesize
  \renewcommand{\arraystretch}{1.15}
  \setlength{\tabcolsep}{3pt}
  \setlength{\emergencystretch}{2em} % reduziert Overfull/Abschneiden bei langen Wörtern

  % Spaltentypen: feste Breite + X-Spalten mit "RaggedRight" (Umbruch + Silbentrennung)
  \newcolumntype{P}[1]{>{\RaggedRight\arraybackslash}p{#1}}
  \newcolumntype{Y}{>{\RaggedRight\arraybackslash}X}

  \begin{tabularx}{\linewidth}{@{}P{0.20\linewidth}YYYYP{0.12\linewidth}@{}}
    \toprule
    \textbf{Architekturpattern} &
    \textbf{Kernkomponenten} &
    \textbf{Abhängigkeiten} &
    \textbf{Stärken} &
    \textbf{Limitationen} &
    \textbf{Beispielpaper} \\
    \midrule


    \makecell[l]{Erkennung von \\Abweichungen} &
    Vorhersage-/Prognosemodell (Last, Ausfälle, Verhalten)	Prädiktive Analytik, ML-Modelle (Klassifikation, Regression), Historische Daten (Build-Logs, Code-Änderungen). &
    Qualität/Verfügbarkeit der Trainingsdaten, Geeignete Algorithmusauswahl, Cloud-Native-Architekturen. &
    Reduziert Build-Fehler/verbessert Pipeline-Effizienz, Intelligente Ressourcenverteilung/Lastoptimierung, Vorhersage von Ausfällen/Deployment-Problemen, Dynamische Anpassung an Traffic-Spitzen. &
    Umgang mit unstrukturierten Daten ist komplex, Rechenintensiv (bei großen Datensätzen), Risiko des Overfittings (bei Entscheidungsbäumen). &
    P1, P2, P3, P11, P13, P14, P15, P16, P17, P20, P21 \\
    \midrule

    \makecell[l]{Vorhersage-/\\Prognosemodell} &
    Entscheidungs- und Optimierungsschleife (dynamische Steuerung)	Reinforcement Learning (RL) Algorithmen, Dynamische Ressourcenzuteilung, Autoscaling-Konfigurationen, Feedback-Schleifen.	&
    Genaue Performance-Metriken (Latenz), Definierte Belohnungs-/Strafmechanismen (für RL), Integration in die CI/CD-Pipeline.	&
    Dynamisches Skalieren/Optimieren der Ressourcennutzung, Minimierung der Betriebskosten, Optimale Verwaltung dynamischer Workloads, Adaptives Deployment. &
    Komplex in der Implementierung (RL), Erfordert klar definierte Trainingsmechanismen, Hohe Rechenleistung.	&
    P2, P3, P11, P12, P13, P14, P15, P16, P17, P19, P20, P21 \\
    \midrule

    \makecell[l]{Entscheidungs- \\und Optimierungsschleife} &
    Betriebs-/Integrationspattern für KI, Workflow-Orchestrierung (Kubeflow, Airflow, MLFlow), CI/CD-Komponente, Monitoring (Prometheus, Grafana), Feedback-Schleifen (für Retraining). &
    Cloud-Native-Prinzipien (Containerisierung, Microservices), DevOps-Automatisierung, Team-Kollaboration (MLOps-Rollen).	&
    Erhöhte Reproduzierbarkeit/Verfolgbarkeit (Data, Code, Model), Kontinuierliches Training/Monitoring, Überwachung der Modell-Performance (Model Drift/Verschlechterung), Steigerung der Zuverlässigkeit/Effizienz von ML-Workflows.	&
    Komplexität der Architektur/Implementierung, Ressourcenintensiv, Benötigt spezialisiertes Fachwissen.	&
    P2, P6, P7, P11, P14, P15, P16, P18, P19, P20, P21 \\
    \midrule

    \makecell[l]{Resilienz, Self-Healing\\und Predictive Maintenance} &
    Governance-/Compliance-orientierter KI-Betrieb	Datenschutz und Sicherheit, Eigentum und Herkunft, Security-Tools (z.B. Snyk, Azure Sentinel), Verschlüsselungsmechanismen, Identitätsmanagement. &
    Einhaltung von Vorschriften, Transparenz und Erklärbarkeit (XAI), Kontinuierliche Sicherheitsüberwachung.	&
    Schutz sensibler Daten, Einhaltung gesetzlicher Anforderungen/Compliance, Vertrauensbildung/Zuverlässigkeit, Reduzierung des Risikos.	&
    Erhöhte Komplexität der Workflows, Interoperabilitätsprobleme (in Multi-Cloud), Schwierigkeiten bei der Erklärbarkeit (XAI) fortgeschrittener Modelle, Hohe Kosten für Compliance. &
    P3, P4, P9, P11, P12, P14, P15, P16, P17, P18, P19, P20, P21\\
    \midrule
    \bottomrule
  \end{tabularx}
\end{table}
\end{landscape}