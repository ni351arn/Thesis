\begin{table}[!htbp]
\caption{Übersicht zum Muster DevSecOps (Security Ops)}
\label{tab:devsecops}
\footnotesize
\renewcommand{\arraystretch}{1.2}
\setlength{\tabcolsep}{7pt}

\begin{tabularx}{\linewidth}{p{3.5cm} X}
\toprule
\textbf{DevSecOps \newline (Security Ops)} & \textbf{Beschreibung} \\
\midrule

Konkrete Plattform-\newline aufgabe &
Priorisierung von Sicherheits-Scans sowie Echtzeit-Erkennung von Angriffen (z. B. DDoS, Malware) anhand von Netzwerk- und Audit-Telemetrie.
\\

Empfohlene \gls{ki}-Methode &
\gls{dl} (z. B. CNNs auf flow-basierten Merkmalsrepr\"asentationen) oder Gradient Boosting (z. B. XGBoost) f\"ur unausgewogene Sicherheitsdaten \cite{uddohAIBasedThreatDetection2021}.
\\

KPI/Mehrwert &
Erkennungsgenauigkeit bis ca.\ 95 \%; Reduktion von Fehlalarmen um ca.\ 20 \%; hohe Detektionsraten auch f\"ur neuartige Angriffsmuster (studienabhängig) \cite{uddohAIBasedThreatDetection2021}.
\\

Limitationen &
Hoher Rechenbedarf (insbesondere bei \gls{dl}-Inferenz); Datenschutz- und Compliance-Risiken (DSGVO) bei der Analyse paketbasierter Protokolle und potenziell personenbezogener Daten.
\\

Einsatzvoraussetzungen &
Echtzeit-Erfassung von Netzwerkverkehr (z. B. Flow-Daten) und Protokolldaten. Externe Bedrohungsinformationen zur Kontextualisierung und Priorisierung \cite{uddohAIBasedThreatDetection2021,kathiresanCybersecurityRiskModeling2025}.
\\

Konkrete Tools &
IBM Watson, Microsoft Sentinel, Amazon GuardDuty, Trivy, Grype, Snyk, Falco, Aqua Security, Kyverno, Clair \cite{zaaloukCLOUDNATIVEARTIFICIAL,kankanalaAIMLDevOps2024,sikhaCloudNativeApplicationDevelopment2023}
\\

\bottomrule
\end{tabularx}
\end{table}
