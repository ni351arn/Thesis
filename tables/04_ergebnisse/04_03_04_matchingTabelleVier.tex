\begin{table}[H]
\caption{Übersicht zum Muster DevSecOps (Security Ops)}
\label{tab:devsecops}
\footnotesize
\renewcommand{\arraystretch}{1.2}
\setlength{\tabcolsep}{7pt}

\begin{tabularx}{\linewidth}{p{3.5cm} X}
\toprule
\textbf{DevSecOps (Security Ops)} & \textbf{Beschreibung} \\
\midrule

Konkreter Plattform-Task &
Priorisierung von Sicherheits-Scans sowie Echtzeit-Erkennung von Angriffen (z.B.\ DDoS, Malware) anhand von Netzwerk- und Audit-Telemetrie.
\\

Empfohlene KI-Methode &
Deep Learning (z.B.\ CNNs auf flow-basierten Merkmalsrepr\"asentationen) oder Gradient Boosting (z.B.\ XGBoost) f\"ur unausgewogene Sicherheitsdaten \cite{uddohAIBasedThreatDetection2021}.
\\

KPI/Mehrwert &
Erkennungsgenauigkeit bis ca.\ 95 \%; Reduktion von Fehlalarmen um ca.\ 20 \%; hohe Detektionsraten auch f\"ur neuartige Angriffsmuster (studienabhängig) \cite{uddohAIBasedThreatDetection2021}.
\\

Limitationen &
Hoher Rechenbedarf (insbesondere bei Deep-Learning-Inferenz); Datenschutz- und Compliance-Risiken (DSGVO) bei der Analyse paketbasierter Protokolle und potenziell personenbezogener Daten.
\\

Einsatzvoraussetzungen &
Echtzeit-Erfassung von Netzwerk-Flows (z.B.\ NetFlow/sFlow) und Logdaten (z.B.\ Cloud-Audit-Logs) sowie Integration von Threat-Intelligence-Informationen (z.B.\ MITRE ATT\&CK) zur Kontextualisierung und Priorisierung.
\\

Konkrete Tools &
IBM Watson, Microsoft Sentinel, Amazon GuardDuty, Trivy, Grype, Snyk, Falco, Aqua Security, Kyverno, Clair \cite{kankanalaAIMLDevOps2024, zaaloukCLOUDNATIVEARTIFICIAL, sikhaCloudNativeApplicationDevelopment2023}
\\

\bottomrule
\end{tabularx}
\end{table}
