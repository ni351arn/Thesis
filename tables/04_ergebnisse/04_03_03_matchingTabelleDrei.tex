\begin{table}[H]
\caption{Übersicht zum Muster Intelligente Build-Fehlerdiagnose}
\label{tab:intelligenteBuildFehlerdiagnose}
\footnotesize
\renewcommand{\arraystretch}{1.2}
\setlength{\tabcolsep}{7pt}

\begin{tabularx}{\linewidth}{p{3.5cm} X}
\toprule
\textbf{Intelligente Build-Fehlerdiagnose} & \textbf{Beschreibung} \\
\midrule

Konkreter Plattform-Task &
Früherkennung von Build-Fehlschlägen durch Analyse von CI-Logs (z.B.\ Jenkins/Git) sowie automatisierte Clusterung/Filterung von Log-Signaturen zur Ursachenanalyse.
\\

Empfohlene KI-Methode &
Supervised Learning (z.B.\ Random Forest, XGBoost) zur Klassifikation von Fehlerrisiken und Fehlertypen auf Basis historischer Build-Metadaten und Log-Features \cite{kankanalaAIMLDevOps2024,tamanampudiAIEnhancedContinuousIntegration}.
\\

KPI/Mehrwert &
Deployment-Zeit bis zu -30\%; Vorhersagegenauigkeit ca.\ 87\%; Reduzierung der Fehlersuchzeit bis zu 40\% \cite{kankanalaAIMLDevOps2024,tamanampudiAIEnhancedContinuousIntegration}.
\\

Limitationen &
Class Imbalance (wenige Fehlersamples); Konzeptdrift bei Änderungen an Build-Pipelines/Dependencies erfordert kontinuierliches Monitoring und Retraining.
\\

Einsatzvoraussetzungen &
Zentralisierte und konsistent strukturierte Build- und Test-Logs sowie eine ausreichend große Historie gelabelter Daten (Erfolg vs.\ Fehlertyp) inklusive stabiler Referenz auf Commit, Pipeline-Stage und Artefaktversion.
\\

Konkrete Tools &
K8sGPT, Jenkins X (mit AI-Plugins), CircleCI Insights, DeepCode, SonarQube, GitLab CI, GitHub Copilot \cite{kankanalaAIMLDevOps2024, enemosahEnhancingDevOpsEfficiency2025, zaaloukCLOUDNATIVEARTIFICIAL, sikhaCloudNativeApplicationDevelopment2023, supritpattanayakIntegratingAIDevOps2024}.
\\

\bottomrule
\end{tabularx}
\end{table}
