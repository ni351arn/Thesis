\begin{table}[!htbp]
\caption{Übersicht zum Muster Intelligente Build-Fehlerdiagnose}
\label{tab:intelligenteBuildFehlerdiagnose}
\footnotesize
\renewcommand{\arraystretch}{1.2}
\setlength{\tabcolsep}{7pt}

\begin{tabularx}{\linewidth}{p{3.5cm} X}
\toprule
\textbf{Intelligente Build-Fehlerdiagnose} & \textbf{Beschreibung} \\
\midrule

Konkrete Plattform-\newline aufgabe &
Früherkennung von Build-Fehlschlägen durch Analyse von CI-Logs (z. B. Jenkins/Git) sowie automatisierte Gruppierung / Filterung von Log-Signaturen zur Ursachenanalyse.
\\

Empfohlene \gls{ki}-Methode &
\gls{sl} (z. B. Random Forest, XGBoost) zur Klassifikation von Fehlerrisiken und Fehlertypen auf Basis historischer Build-Metadaten und Log-Features \cite{kankanalaAIMLDevOps2024,tamanampudiAIEnhancedContinuousIntegration}.
\\

KPI/Mehrwert &
Deployment-Zeit bis zu -30 \%; Vorhersagegenauigkeit ca.\ 87 \%; Reduzierung der Fehlersuchzeit bis zu 40 \% \cite{kankanalaAIMLDevOps2024,tamanampudiAIEnhancedContinuousIntegration}.
\\

Limitationen &
Geringe Anzahl von Fehlerfällen (unausgewogene Trainingsdaten), Konzeptdrift bei Änderungen an Build-Pipelines oder Abhängigkeiten, Regelmäßige Überwachung und erneutes Training erforderlich.\\

Einsatzvoraussetzungen &
Zentral verfügbare und einheitlich strukturierte Build- und Test-Logs, ausreichend große, gelabelte Historie (Erfolg vs. Fehlertyp). Eindeutige Referenz auf Commit, Pipeline-Schritt und Artefaktversion \cite{tamanampudiAIEnhancedContinuousIntegration,kathiresanCybersecurityRiskModeling2025}.
\\

Konkrete Tools &
K8sGPT, Jenkins X (mit AI-Plugins), CircleCI Insights, DeepCode, SonarQube, GitLab CI, GitHub Copilot \cite{zaaloukCLOUDNATIVEARTIFICIAL,enemosahEnhancingDevOpsEfficiency2025,kankanalaAIMLDevOps2024,supritpattanayakIntegratingAIDevOps2024,sikhaCloudNativeApplicationDevelopment2023}.
\\

\bottomrule
\end{tabularx}
\end{table}
