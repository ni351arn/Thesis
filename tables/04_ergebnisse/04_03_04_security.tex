\begin{table}[htbp]
\caption{Intelligent Security \& Threat Detection}
\label{tab:pattern4-intelligent-security-threat-detection}
\footnotesize
\renewcommand{\arraystretch}{1.2}
\setlength{\tabcolsep}{6pt}

\begin{tabularx}{\linewidth}{p{4.2cm} Y p{3.0cm}}
\toprule
\textbf{Kategorie} & \textbf{Details} & \textbf{Quellen (IDs)} \\
\midrule

Plattform-Aufgabe &
Erkennung von Intrusionen (DDoS, Malware), Anomalie-Erkennung im Netzwerkverkehr und Benutzerverhalten. &
\cite{uddohAIBasedThreatDetection2021}, \cite{govindarajanMachineLearningBased2025} \\

Datenquellen &
Netzwerk: IP-Traffic, System-Logs, Cloud-Audit-Logs, Benutzer-Zugriffsmuster. &
\cite{uddohAIBasedThreatDetection2021}, \cite{govindarajanMachineLearningBased2025} \\

Datenverwendung &
Abgleich von Echtzeit-Datenströmen mit Baselines für \glqq normales\grqq{} Verhalten zur Identifikation von Ausreißern. &
\cite{uddohAIBasedThreatDetection2021}, \cite{govindarajanMachineLearningBased2025} \\

KI-Methoden &
Deep Learning (CNN für visuelle Pattern-Analyse); XGBoost/Random Forest für Klassifizierung auf imbalanced Datasets. &
\cite{uddohAIBasedThreatDetection2021}, \cite{govindarajanMachineLearningBased2025} \\

Konkrete Tools &
IBM Watson AI, Microsoft Sentinel, Trivy/Clair (für Scans in CI/CD). &
\cite{kankanalaAIMLDevOps2024} \\

Vorteile (Ziele) &
Detektionsrate von bis zu 95\% bei fortgeschrittenen Bedrohungen; Reduktion von Sicherheitsvorfällen. &
\cite{uddohAIBasedThreatDetection2021}, \cite{tamminediAutomatingKubernetesOperations2024} \\

\bottomrule
\end{tabularx}
\end{table}
