% --- Tabelle 4: Intelligente Sicherheit & Bedrohungserkennung ---
\begin{table}[H]
\caption{Intelligente Sicherheit \& Bedrohungserkennung}
\label{tab:pattern4-intelligent-security-threat-detection}
\footnotesize
\renewcommand{\arraystretch}{1.2}
\setlength{\tabcolsep}{6pt}

\begin{tabularx}{\linewidth}{p{4.2cm} Y p{3.2cm}}
\toprule
\textbf{Kategorie} & \textbf{Beschreibung} & \textbf{Beispielquellen} \\
\midrule

Plattformaufgabe &
Erkennung von Intrusionen (DDoS, Malware), Anomalieerkennung im Netzwerkverkehr sowie im Benutzerverhalten. &
\cite{uddohAIBasedThreatDetection2021}, \cite{govindarajanMachineLearningBased2025} \\

Datenquellen &
Netzwerk: IP-Traffic, System-Logs, Cloud-Audit-Logs, Benutzer-Zugriffsmuster. &
\cite{uddohAIBasedThreatDetection2021}, \cite{govindarajanMachineLearningBased2025} \\

Datennutzung &
Abgleich von Echtzeit-Datenströmen mit Baselines für \glqq normales\grqq{} Verhalten zur Identifikation von Ausreißern. &
\cite{uddohAIBasedThreatDetection2021}, \cite{govindarajanMachineLearningBased2025} \\

KI-/ML-Methoden &
Deep Learning (CNN für visuelle Musteranalyse); XGBoost/Random Forest zur Klassifikation auf unausgewogenen Datensätzen (Klassenungleichgewicht). &
\cite{uddohAIBasedThreatDetection2021}, \cite{govindarajanMachineLearningBased2025} \\

Werkzeuge (Tools) &
IBM Watson AI, Microsoft Sentinel, Trivy/Clair (für Scans in CI/CD). &
\cite{kankanalaAIMLDevOps2024} \\

Ziele/Nutzen &
Detektionsrate von bis zu 95\% bei fortgeschrittenen Bedrohungen; Reduktion von Sicherheitsvorfällen. &
\cite{uddohAIBasedThreatDetection2021}, \cite{tamminediAutomatingKubernetesOperations2024} \\

\bottomrule
\end{tabularx}
\end{table}
