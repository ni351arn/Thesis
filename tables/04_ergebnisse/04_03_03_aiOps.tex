\begin{table}[htbp]
\caption{KI-gestützte Observability \& Ursachenanalyse (AIOps)}
\label{tab:pattern3-aiops-observability-rca}
\footnotesize
\renewcommand{\arraystretch}{1.2}
\setlength{\tabcolsep}{6pt}

\begin{tabularx}{\linewidth}{p{4.2cm} Y p{3.2cm}}
\toprule
\textbf{Kategorie} & \textbf{Beschreibung} & \textbf{Beispielquellen} \\
\midrule

Plattformaufgabe &
Log-basierte Anomalieerkennung; automatisierte Root-Cause-Analyse (RCA); natürlichsprachliche Schnittstelle für Cluster-Abfragen. &
\cite{zaaloukCLOUDNATIVEARTIFICIAL}, \cite{tamminediAutomatingKubernetesOperations2024} \\

Datenquellen &
Telemetrie: System-Logs, Kubernetes-Events, Traces (OpenTelemetry), Warnsignale von Sensoren/Nodes. &
\cite{zaaloukCLOUDNATIVEARTIFICIAL}, \cite{guptaCloudNativeMLArchitecting2024a} \\

Datennutzung &
Mustererkennung in unstrukturierten Log-Daten zur Korrelation von Fehlern über Microservices hinweg. &
\cite{zaaloukCLOUDNATIVEARTIFICIAL} \\

KI-/ML-Methoden &
LLMs (Natural Language Processing) zur Interpretation von Fehlermeldungen; Clustering (K-Means) zur Bündelung ähnlicher Alerts. &
\cite{zaaloukCLOUDNATIVEARTIFICIAL}, \cite{enemosahEnhancingDevOpsEfficiency2025} \\

Werkzeuge (Tools) &
K8sGPT, OpenLLMetry, Prometheus/Grafana mit KI-Erweiterungen. &
\cite{zaaloukCLOUDNATIVEARTIFICIAL}, \cite{tamminediAutomatingKubernetesOperations2024} \\

Ziele/Nutzen &
Reduktion der mittleren Zeit bis zur Fehlerbehebung (MTTR) um bis zu 39\%; Reduktion von Fehlalarmen (False Positives) in der Alarmierung. &
\cite{tamminediAutomatingKubernetesOperations2024} \\

\bottomrule
\end{tabularx}
\end{table}
