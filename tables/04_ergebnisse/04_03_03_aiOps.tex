\begin{table}[htbp]
\caption{AI-driven Observability \& Root Cause Analysis (AIOps)}
\label{tab:pattern3-aiops-observability-rca}
\footnotesize
\renewcommand{\arraystretch}{1.2}
\setlength{\tabcolsep}{6pt}

\begin{tabularx}{\linewidth}{p{4.2cm} Y p{3.0cm}}
\toprule
\textbf{Kategorie} & \textbf{Details} & \textbf{Quellen (IDs)} \\
\midrule

Plattform-Aufgabe &
Log-Anomalie-Erkennung; automatisierte Root Cause Analysis (RCA); Natural Language Interface für Cluster-Abfragen. &
\cite{zaaloukCLOUDNATIVEARTIFICIAL}, \cite{tamminediAutomatingKubernetesOperations2024} \\

Datenquellen &
Telemetrie: System-Logs, Kubernetes Events, Traces (OpenTelemetry), Warnsignale von Sensoren/Nodes. &
\cite{zaaloukCLOUDNATIVEARTIFICIAL}, \cite{guptaCloudNativeMLArchitecting2024a} \\
Datenverwendung &
Mustererkennung in unstrukturierten Log-Daten zur Korrelation von Fehlern über Microservices hinweg. &
\cite{zaaloukCLOUDNATIVEARTIFICIAL} \\

KI-Methoden &
LLMs (Natural Language Processing) zur Interpretation von Fehlermeldungen; Clustering (K-Means) zur Gruppierung ähnlicher Alerts. &
\cite{zaaloukCLOUDNATIVEARTIFICIAL}, \cite{enemosahEnhancingDevOpsEfficiency2025} \\
Konkrete Tools &
K8sGPT, OpenLLMetry, Prometheus/Grafana mit KI-Extensions. &
\cite{zaaloukCLOUDNATIVEARTIFICIAL}, \cite{tamminediAutomatingKubernetesOperations2024} \\

Vorteile (Ziele) &
Erhöhung der MTTR (Mean Time To Resolution) um bis zu 39\%; Reduktion von False Positives in der Alarmierung. &
\cite{tamminediAutomatingKubernetesOperations2024} \\
\bottomrule
\end{tabularx}
\end{table}
