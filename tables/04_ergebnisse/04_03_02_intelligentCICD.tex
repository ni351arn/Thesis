\begin{table}[htbp]
\caption{Intelligent CI/CD (Predictive Failure \& Adaptive Rollbacks)}
\label{tab:pattern2-intelligent-cicd}
\footnotesize
\renewcommand{\arraystretch}{1.2}
\setlength{\tabcolsep}{6pt}

\begin{tabularx}{\linewidth}{p{4.2cm} Y p{3.0cm}}
\toprule
\textbf{Kategorie} & \textbf{Details} & \textbf{Quellen (IDs)} \\
\midrule

Plattform-Aufgabe &
Vorhersage von Build-Fehlern; automatisierte Rollbacks bei Anomalien nach dem Deployment. &
\cite{tamanampudiAIEnhancedContinuousIntegration}, \cite{enemosahEnhancingDevOpsEfficiency2025} \\

Datenquellen &
Logs: Build-Logs, Testresultate, Code-Metadaten (Commit-Historie), Fehler-Raten während des Rollouts. &
\cite{tamanampudiAIEnhancedContinuousIntegration}, \cite{supritpattanayakIntegratingAIDevOps2024} \\
Datenverwendung &
Training von Modellen auf historischen Fehlermustern, um \glqq High-Risk\grqq{}-Änderungen frühzeitig zu stoppen. &
\cite{tamanampudiAIEnhancedContinuousIntegration} \\

KI-Methoden &
Supervised Learning (Random Forest, Decision Trees); Reinforcement Learning (DQN) für optimierte Rollback-Entscheidungen. &
\cite{tamanampudiAIEnhancedContinuousIntegration} \\
Konkrete Tools &
Jenkins X (mit ML-Plugins), Spinnaker, StackStorm (für ereignisbasierte Automatisierung). &
\cite{kankanalaAIMLDevOps2024}, \cite{supritpattanayakIntegratingAIDevOps2024} \\

Vorteile (Ziele) &
Reduzierung der Deployment-Zeit um 30\%; Steigerung der Systemstabilität durch proaktive Fehlervermeidung. &
\cite{kankanalaAIMLDevOps2024}, \cite{tamanampudiAIEnhancedContinuousIntegration} \\
\bottomrule
\end{tabularx}
\end{table}
