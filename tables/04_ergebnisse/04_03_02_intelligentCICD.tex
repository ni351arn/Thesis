% --- Tabelle 2: Intelligente CI/CD (Fehlervorhersage & adaptive Rollbacks) ---
\begin{table}[H]
\caption{Intelligente CI/CD (Fehlervorhersage \& adaptive Rollbacks)}
\label{tab:pattern2-intelligent-cicd}
\footnotesize
\renewcommand{\arraystretch}{1.2}
\setlength{\tabcolsep}{6pt}

\begin{tabularx}{\linewidth}{p{4.2cm} Y p{3.2cm}}
\toprule
\textbf{Kategorie} & \textbf{Beschreibung} & \textbf{Beispielquellen} \\
\midrule

Plattformaufgabe &
Vorhersage von Build-Fehlern; automatisierte Rollbacks bei Anomalien nach dem Deployment. &
\cite{tamanampudiAIEnhancedContinuousIntegration}, \cite{enemosahEnhancingDevOpsEfficiency2025} \\

Datenquellen &
Logs: Build-Logs, Testergebnisse, Code-Metadaten (Commit-Historie), Fehlerraten während des Rollouts. &
\cite{tamanampudiAIEnhancedContinuousIntegration}, \cite{supritpattanayakIntegratingAIDevOps2024} \\

Datennutzung &
Training von Modellen auf historischen Fehlermustern, um risikoreiche Änderungen frühzeitig zu stoppen. &
\cite{tamanampudiAIEnhancedContinuousIntegration} \\

KI-/ML-Methoden &
Supervised Learning (Random Forest, Entscheidungsbäume); Reinforcement Learning (DQN) zur Optimierung von Rollback-Entscheidungen. &
\cite{tamanampudiAIEnhancedContinuousIntegration} \\

Werkzeuge (Tools) &
Jenkins X (mit ML-Plugins), Spinnaker, StackStorm (für ereignisbasierte Automatisierung). &
\cite{kankanalaAIMLDevOps2024}, \cite{supritpattanayakIntegratingAIDevOps2024} \\

Ziele/Nutzen &
Reduzierung der Deployment-Zeit um 30\%; Steigerung der Systemstabilität durch proaktive Fehlervermeidung. &
\cite{kankanalaAIMLDevOps2024}, \cite{tamanampudiAIEnhancedContinuousIntegration} \\

\bottomrule
\end{tabularx}
\end{table}
