\begin{table}[H]
\caption{Bewertungsskalen der Zusatzdimensionen}
\label{tab:zusatzdimensionen}
\footnotesize
\renewcommand{\arraystretch}{1.2}
\setlength{\tabcolsep}{7pt}

\begin{tabularx}{\linewidth}{p{4.2cm} p{2.2cm} X}
\toprule
\textbf{Zusatzdimension} & \textbf{Ausprägung} & \textbf{Beschreibung} \\
\midrule

\makecell[l]{Umsetzbarkeit} & Niedrig &
Daten, Integrationspfade und erforderliches Know-how sind vorhanden; Umsetzung ohne strukturelle Änderungen möglich. \\
& Mittel &
Vorarbeiten bei Datenverfügbarkeit, Integration oder Kompetenzen notwendig. \\
& Hoch &
Zentrale Daten, Schnittstellen oder Fähigkeiten fehlen aktuell und erfordern zusätzliche Vorarbeiten. \\
\midrule

\makecell[l]{Betriebswirksamkeit\\\& Skalierbarkeit} & Niedrig &
Wirkung lokal oder instabil; stark kontextabhängig und nur schwer auf weitere Services oder Teams übertragbar. \\
& Mittel &
Stabile Wirkung in mehreren Szenarien, jedoch mit begrenzter Skalierbarkeit oder erhöhtem operativem Aufwand. \\
& Hoch &
Zuverlässig wirksam im Plattformbetrieb und plattformweit skalierbar ohne individuellen Zusatzaufwand. \\
\midrule

\makecell[l]{Governance\\\& Reifegrad} & Niedrig &
Geringe Governance- und Compliance-Hürden bei produktiv erprobtem Reifegrad. \\
& Mittel &
Zusätzliche Anforderungen an Dokumentation, Compliance oder organisatorische Abstimmung erforderlich. \\
& Hoch &
Hohe regulatorische Anforderungen oder geringer Reifegrad (z.\,B.\ Proof of Concept oder Forschung). \\
\bottomrule
\end{tabularx}
\end{table}
