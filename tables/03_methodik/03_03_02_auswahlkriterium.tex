\begin{table}[htbp]
  \centering
  \caption{Literatur-Auswahlkriterien}
  \label{tab:auswahlkriterien}
  \footnotesize
  \renewcommand{\arraystretch}{1.15}
  \begin{tabularx}{\linewidth}{@{}lY@{}}
    \toprule
    \textbf{Kriterium} & \textbf{Beschreibung} \\
    \midrule
    EK1 & Die Publikation befasst sich mit dem Einsatz von KI oder Machine Learning im Kontext von Platform Engineering, Cloud-Native-Technologien, DevOps oder AIOps. \\
    EK2 & Die Studie beschreibt konkrete KI-Methoden, Anwendungen, Architekturen oder Use Cases, die sich auf Plattformumgebungen beziehen. \\
    EK3 & Die Arbeit ist wissenschaftlich fundiert, z.B. als Konferenz-, Journal- oder White Paper, auch wenn kein Peer-Review-Verfahren vorliegt. \\
    EK4 & Veröffentlichungen sind in englischer oder deutscher Sprache verfasst und nach 2020 erschienen. \\
    \midrule
    AK1 & Arbeiten, die keinen direkten Bezug zu KI im Platform Engineering oder verwandten Domänen aufweisen. \\
    AK2 & Review-Paper oder systematische Übersichtsarbeiten, die keine eigenen empirischen oder technischen Beiträge enthalten. \\
    AK3 & Studien ohne nachvollziehbare methodische Grundlage oder ohne Beschreibung der verwendeten KI-Techniken. \\
    AK4 & Arbeiten mit primärem Fokus auf MLOps, insbesondere auf Infrastruktur, Deployment und Lebenszyklusmanagement von ML-Modellen für Data-Science-Workflows, ohne direkten Bezug zur operativen Unterstützung von Platform Engineers (AI for Ops). \\
    \bottomrule
  \end{tabularx}
\end{table}