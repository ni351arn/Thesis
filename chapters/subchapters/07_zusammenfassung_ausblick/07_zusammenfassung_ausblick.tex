\chapter{Zusammenfassung und Ausblick}
\label{ch:zusammenfassung}
Dieses Kapitel fasst die zentralen Ergebnisse der Arbeit zusammen und ordnet sie in den Gesamtkontext ein. 
Dafür werden zunächst in Kapitel \ref{sec:zusammenfassung} die wichtigsten Erkenntnisse zusammengefasst.
Der Ausblick in Kapitel \ref{sec:ausblick} beschreibt mögliche Ansatzpunkte für weiterführende Forschung und zeigt auf, an welchen Stellen eine vertiefende Untersuchung sinnvoll erscheint.


\section{Zusammenfassung}
\label{sec:zusammenfassung}
Der Einsatz von KI in Cloud-Native Plattformen bietet Potenzial zur Automatisierung und Optimierung betrieblicher Abläufe.
Gleichzeitig fehlen bislang systematische Ansätze, um KI-Technologien im Plattformkontext vergleichbar zu bewerten und gezielt einzusetzen.

Eine systematische Literaturrecherche wurde durchgeführt, um bestehende Lösungsansätze einzuordnen.
Es wurden wiederkehrende Anwendungsfelder sowie typische Herausforderungen beim Einsatz von KI-Technologien festgestellt.
Die Anwendungsfelder beziehen sich vornehmlich auf die Optimierung von Ressourcen und Workload, den Betrieb der Plattform, die Verbesserung von CI/CD-Pipelines sowie sicherheitsrelevante Aspekte.
Besonders im Bereich der IT-Sicherheit, der Qualität und Verfügbarkeit von Daten sowie bei der Integration in bestehende Plattformarchitekturen treten Herausforderungen auf.
Die in der Literatur beschriebenen KI-Tools, Lernparadigmen und Algorithmen wurden erfasst und anschließend klassifiziert.
Basierend darauf wurden generische Use-Case-Muster entwickelt, die typische Nutzungsszenarien und Anforderungen im Plattformbetrieb darstellen.

Für die Bewertung und Auswahl geeigneter KI-Use-Cases wurde ein Bewertungskonzept entwickelt.
Dieses ordnet Anwendungsfälle entlang der Dimensionen Implementierungsaufwand und operativer Mehrwert ein.
Ergänzend werden Aspekte der Umsetzbarkeit, der betrieblichen Wirksamkeit, der Governance sowie des Reifegrads berücksichtigt.
Anhand der Analyse der Bosch Digital Manufacturing Plattform wurden wiederkehrende Probleme in Build- und Bereitstellungsprozessen identifiziert.
Dieses Problemfeld wurde auf die generischen Use-Case-Muster abgebildet und mithilfe des Bewertungskonzepts analysiert.
Auf dieser Grundlage konnten Handlungsempfehlungen einschließlich eines konkreten KI-Tool-Vorschlags abgeleitet werden.


\section{Ausblick}
\label{sec:ausblick}
Der entwickelte Bewertungsansatz ist ein erster Schritt zur Bewertung von KI-Anwendungsfällen. 
Die Bewertungslogik ist allgemein gehalten und ermöglicht eine qualitative Priorisierung entlang klar definierter Leitfragen.
Eine weiterführende Ausarbeitung kann diese Logik durch quantitative Kriterien ergänzen und auf weitere Anwendungsfälle anwenden.
Für eine belastbare Entscheidungsgrundlage ist jedoch eine weitere Vertiefung erforderlich. 
Die Gewichtung der einzelnen Kriterien sollte in zukünftigen Arbeiten systematisch überprüft und kontextabhängig angepasst werden.

Das Bewertungskonzept bietet die Möglichkeit, weitere Anwendungsfälle zu bewerten. 
Zusätzliche Use Cases lassen sich mit dem gleichen Raster bewerten, ohne die Bewertungslogik grundlegend zu verändern. 
Dadurch kann das Konzept schrittweise erweitert werden.

Ein nächster Schritt besteht in der prototypischen Umsetzung einzelner priorisierter Anwendungsfälle im Plattformbetrieb. 
Erst durch den praktischen Einsatz lassen sich Aussagen zur tatsächlichen Wirksamkeit, zum Betriebsaufwand und zu langfristigen Effekten treffen. 
Dies betrifft vor allem Aspekte wie Stabilität, Wartbarkeit und den Umgang mit veränderlichen Datenlagen.

Insgesamt zeigt die Arbeit, dass eine strukturierte Einordnung von KI im Plattformkontext möglich ist, auch ohne detaillierte Kosten- oder Leistungsmodelle. 
Dieser Ansatz bietet eine Grundlage, um zukünftige Analysen gezielter und vergleichbarer durchzuführen und damit sowohl Forschung als auch Praxis weiter zu unterstützen.
