\chapter{Zusammenfassung und Ausblick}
\label{ch:zusammenfassung}
Dieses Kapitel fasst die zentralen Ergebnisse der Arbeit zusammen und ordnet sie in den Gesamtkontext ein. 
Dafür werden zunächst in Kapitel \ref{sec:zusammenfassung} die wichtigsten Erkenntnisse zusammengefasst.
Der Ausblick in Kapitel \ref{sec:ausblick} beschreibt mögliche Ansatzpunkte für weiterführende Forschung und zeigt auf, an welchen Stellen eine vertiefende Untersuchung sinnvoll erscheint.


\section{Zusammenfassung}
\label{sec:zusammenfassung}
Die Arbeit hat gezeigt, dass sich \gls{ki}-Anwendungen entlang klar abgrenzbarer Anwendungsfelder strukturieren lassen und sich dabei wiederkehrende technische und organisatorische Herausforderungen ergeben.
Die Analyse zeigt, dass in der Literatur bislang keine einheitliche Bewertungsgrundlage für \gls{ki}-Anwendungen im Plattformbetrieb existiert.

Auf Basis einer systematischen Literaturrecherche wurden wiederkehrende Anwendungsfelder und typische Herausforderungen beim Einsatz von \gls{ki}-Technologien identifiziert.
Die Anwendungsfelder betreffen insbesondere Ressourcen- und Workload-Optimierung, den Plattformbetrieb, die Optimierung von \gls{ci}/\gls{cd}-Pipelines sowie sicherheitsrelevante Aufgaben.
Über alle Felder hinweg zeigen sich ähnliche Herausforderungen, vor allem in den Bereichen Datenqualität und -verfügbarkeit, Integration in bestehende Plattformarchitekturen sowie IT-Sicherheit.
Die in der Literatur beschriebenen \gls{ki}-Ansätze wurden hinsichtlich eingesetzter Werkzeuge, Lernparadigmen und Algorithmen systematisch erfasst und klassifiziert.
Darauf aufbauend wurden generische Use-Case-Muster abgeleitet, die typische Einsatzszenarien und Anforderungen im Plattformbetrieb zusammenfassen.

Für die Bewertung und Auswahl geeigneter \gls{ki}-Use-Cases wurde ein Bewertungskonzept entwickelt.
Dieses ordnet Anwendungsfälle entlang der Dimensionen Implementierungsaufwand und operativer Mehrwert ein.
Ergänzend werden Aspekte der Umsetzbarkeit, der betrieblichen Wirksamkeit, der Governance sowie des Reifegrads berücksichtigt.
Anhand der Analyse der \gls{bmlp} wurden wiederkehrende Probleme in Build- und Bereitstellungsprozessen identifiziert.
Dieses Problemfeld wurde auf die generischen Use-Case-Muster abgebildet und mithilfe des Bewertungskonzepts analysiert.
Auf dieser Grundlage konnten Handlungsempfehlungen einschließlich eines konkreten \gls{ki}-Tool-Vorschlags abgeleitet werden.


\section{Ausblick}
\label{sec:ausblick}
Der entwickelte Bewertungsansatz ist ein erster Schritt zur Bewertung von \gls{ki}-Anwendungsfällen. 
Die Bewertungslogik ist allgemein gehalten und ermöglicht eine qualitative Priorisierung entlang klar definierter Leitfragen.
Eine weiterführende Ausarbeitung kann diese Logik durch quantitative Kriterien ergänzen und auf weitere Anwendungsfälle anwenden.
Für eine belastbare Entscheidungsgrundlage ist jedoch eine weitere Vertiefung erforderlich. 
Die Gewichtung der einzelnen Kriterien sollte in zukünftigen Arbeiten systematisch überprüft und kontextabhängig angepasst werden.

Das Bewertungskonzept bietet die Möglichkeit, weitere Anwendungsfälle zu bewerten. 
Zusätzliche Use Cases lassen sich mit dem gleichen Raster bewerten, ohne die Bewertungslogik grundlegend zu verändern. 
Dadurch kann das Konzept schrittweise erweitert werden.

Ein nächster Schritt besteht in der prototypischen Umsetzung einzelner priorisierter Anwendungsfälle im Plattformbetrieb. 
Erst durch den praktischen Einsatz lassen sich Aussagen zur tatsächlichen Wirksamkeit, zum Betriebsaufwand und zu langfristigen Effekten treffen. 
Dies betrifft vor allem Aspekte wie Stabilität, Wartbarkeit und den Umgang mit veränderlichen Datenlagen.

Insgesamt zeigt die Arbeit, dass eine strukturierte Einordnung von \gls{ki} im Plattformkontext möglich ist, auch ohne detaillierte Kosten- oder Leistungsmodelle. 
Dieser Ansatz bietet eine Grundlage, um zukünftige Analysen gezielter und vergleichbarer durchzuführen und damit sowohl Forschung als auch Praxis weiter zu unterstützen.
