\section{Anwendung des Bewertungskonzepts}
\label{sec:anwendung_bewertungskonzept}
In diesem Abschnitt wird das in Kapitel 5.1 entwickelte Bewertungskonzept auf ein konkretes Problemfeld der \gls{bmlp} angewendet. 
Ziel ist es, dieses Problemfeld systematisch einzuordnen und anhand der definierten Zusatzdimensionen zu bewerten. 
Die Analyse folgt dabei den generischen Use-Case-Mustern aus Kapitel 4.3.
\par\bigskip

Im Betrieb der \gls{bmlp} treten Abbrüche in Build- und Bereitstellungsprozessen immer wieder auf.
Sie entstehen im Rahmen automatisierter \gls{ci}/\gls{cd}-Pipelines und betreffen sowohl einzelne Module als auch verkettete Abläufe über mehrere Komponenten hinweg.
Es liegen keine konsolidierte Kennzahlen vor.
Aus Betriebserfahrungen ergibt sich, dass diese Abbrüche wiederkehrend auftreten.
Die Auswirkungen sind insbesondere in Umgebungen mit paralleler Entwicklung und hoher Änderungsfrequenz sichtbar.
Fehlerhafte Builds blockieren nachgelagerte Schritte und machen erneute Pipeline-Läufe erforderlich.
Dadurch verschiebt sich die Bereitstellung von Funktionen und beeinträchtigt die Änderungsbereitstellung der Plattform insgesamt.
\par\bigskip

Der zentrale Aufwand entsteht nicht durch den Abbruch eines Builds an sich, sondern durch die anschließende Ursachenanalyse. 
Fehlersituationen lassen sich häufig nicht eindeutig einem einzelnen Schritt oder einer einzelnen Komponente zuordnen. 
Stattdessen müssen Informationen aus mehreren Quellen zusammengeführt werden. 
Dazu zählen Pipeline-Läufe, System- und Applikationsprotokolle sowie technische Ereignisse aus der Laufzeitumgebung.
Eine durchgängige Sicht über den gesamten Ablauf ist dabei nur eingeschränkt gegeben. 
Detailinformationen stehen nicht dauerhaft zur Verfügung, sondern müssen bei Bedarf gezielt aktiviert werden. 
Dies erhöht den manuellen Analyseaufwand und verlangsamt die Eingrenzung der eigentlichen Ursache.

Zusätzlich zeigt sich, dass Build- und Bereitstellungsprozesse in der Praxis nicht vollständig unabhängig voneinander sind. 
Entlang der Build- und Auslieferungsprozesse bestehen Abhängigkeiten zwischen einzelnen Komponenten und Schritten.
Beispiele hierfür sind gemeinsam genutzte Artefakte, feste Reihenfolgen bei der Initialisierung oder gegenseitige Startvoraussetzungen einzelner Komponenten.
Obwohl eine möglichst unabhängige Ausführung einzelner Pipelines angestrebt wird, führen diese Kopplungen dazu, dass Fehlerfolgen über mehrere Schritte hinweg wirken. 
Die Analyse erfordert dadurch ein Verständnis des Gesamtzusammenhangs und einen hohen manuellen Analyseaufwand für die beteiligten Teams.
\par\bigskip

Für die systematische Einordnung dieses Problemfelds liegt in der Plattform eine geeignete Datengrundlage vor. 
Pipeline-Laufinformationen und Build-Protokolle werden erfasst und für Auswertungen bereitgestellt.
Applikations- und Prozessprotokolle werden zusammengeführt und folgen einem einheitlichen, strukturierten Format mit standardisierten Feldern.
Dadurch lassen sich relevante Informationen einheitlich auswerten und vergleichen, ohne dass jede Analyse bei null beginnt.

Zudem ist eine historische Protokollierung vorgesehen. 
Die Einträge enthalten unter anderem Zeitstempel sowie Angaben zu Umgebung, Systemkontext, Version und Instanz. 
Pipeline-Ausführungen und Laufzeitereignisse lassen sich über gemeinsame Kennungen und Kontextinformationen miteinander in Beziehung setzen und im Nachgang nachvollziehen.
Damit sind zentrale Voraussetzungen gegeben, um wiederkehrende Fehlermuster zu identifizieren. 
Die Ursachen lassen sich systematischer eingrenzen, statt rein manuell von Einzelfall zu Einzelfall vorzugehen.
\par\bigskip

Das beschriebene Problemfeld wird dem in Abschnitt 4.3.3 abgeleiteten generischen Use Case der intelligenten Build-Fehlerdiagnose zugeordnet. 
Auf dieser Grundlage erfolgt die Bewertung entlang der in Abschnitt 5.1.2 definierten Zusatzdimensionen.

Bezüglich der Umsetzbarkeit sind die erforderlichen Build-, Pipeline- und Protokolldaten in strukturierter Form und mit historischer Tiefe verfügbar. 
Auch der technische Fit zur bestehenden Plattform ist gegeben, da die Analyse an vorhandene Build- und Auslieferungsprozesse anknüpfen kann. 
Zusätzlicher Aufwand entsteht durch die Einbettung in bestehende Abläufe sowie durch den Umgang mit Abhängigkeiten zwischen Komponenten und Pipelines. 
Insgesamt ergibt sich eine mittlere Umsetzbarkeit.

Durch die Auswertung historischer Build- und Protokolldaten kann der manuelle Analyseaufwand reduziert und die Eingrenzung von Fehlerursachen beschleunigt werden. 
Eine Übertragung auf weitere Pipelines ist möglich, erfordert jedoch Anpassungen, wenn Abläufe eng miteinander verknüpft sind. 
In der Gesamtsicht ergibt sich eine mittlere bis hohe Ausprägung.

Da überwiegend technische Prozess- und Protokolldaten verarbeitet werden, entstehen keine zusätzlichen Anforderungen an Datenschutz oder Regulierung.
Abstimmungsaufwände betreffen vor allem Zuständigkeiten und organisatorische Regelungen, nicht jedoch regulatorische Einschränkungen. 
Die Dimension ist daher als niedrig bis mittel einzuordnen.

Der Reifegrad des Use Case ist insgesamt als mittel einzuordnen. 
In der Literatur wird der Ansatz über die reine Konzeptionsphase hinaus beschrieben und in ersten Werkzeugen sowie Pilotanwendungen praktisch eingesetzt.
Gleichzeitig fehlt jedoch bislang eine breite Etablierung mit klar abgesicherten Betriebsmodellen und Verantwortlichkeiten für den dauerhaften Einsatz.
\par\bigskip

Abbildung \ref{fig:05_03_bewertungsraster} zeigt die Einordnung des Use Case in die Bewertungsmatrix auf Basis der durchgeführten Analyse und Anwendung des Bewertungsrasters.
\begin{figure}[H]
    \centering
    \includegraphics[width=0.7\textwidth]{figures/05_bewertung/05_03_bewertungsraster.png}
    \caption{Einordnung des Use Case in die Bewertungsmatrix}
    \label{fig:05_03_bewertungsraster}
\end{figure}