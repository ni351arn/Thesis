\section{Anwendung des Bewertungskonzepts}
\label{sec:anwendung_bewertungskonzept}
In diesem Abschnitt wird das in \autoref{sec:bewertungskonzept} entwickelte Bewertungskonzept auf ein konkretes Problemfeld der \gls{bmlp} angewendet. 
Das Problemfeld wird systematisch eingeordnet und anhand der definierten Zusatzdimensionen bewertet.
Die Analyse folgt dabei den generischen Use-Case-Mustern aus \autoref{sec:matching-framework}.
\par\bigskip

Im Betrieb der \gls{bmlp} treten wiederkehrend Störungen und Fehlerzustände in der Laufzeitumgebung auf.
Sie betreffen einzelne Module sowie verkettete Abläufe über mehrere Komponenten hinweg und wirken sich direkt auf Stabilität und Verfügbarkeit aus.
Es liegen keine konsolidierten Kennzahlen vor.
Aus Betriebserfahrungen ergibt sich, dass diese Störungen regelmäßig auftreten, insbesondere bei hoher Änderungsfrequenz.
Die Auswirkungen zeigen sich in wiederholten Neustarts, nicht startenden Workloads oder Servicezuständen.
Dadurch entstehen Verzögerungen in operativen Abläufen und ein erhöhter Aufwand im Plattformbetrieb.
\par\bigskip

Der zentrale Aufwand entsteht nicht durch das Auftreten eines Fehlzustands an sich, sondern durch die anschließende Eingrenzung der Ursache.
Fehlersituationen lassen sich häufig nicht eindeutig einer einzelnen Komponente zuordnen.
Stattdessen müssen Informationen aus mehreren Quellen zusammengeführt werden.
Dazu zählen Kubernetes-Events und Zustandsinformationen, Applikations- und Systemprotokolle sowie Metriken und Traces aus der Beobachtbarkeit.
Eine durchgängige Sicht über den gesamten Ablauf ist dabei nur eingeschränkt gegeben.
Detailinformationen stehen nicht dauerhaft in ausreichender Tiefe zur Verfügung, sondern müssen bei Bedarf gezielt herangezogen werden.
Dies erhöht den manuellen Analyseaufwand und verlangsamt die Eingrenzung der eigentlichen Ursache.
\par\bigskip

Zusätzlich zeigt sich, dass Störungen im Plattformbetrieb selten isoliert entstehen.
Abhängigkeiten zwischen Komponenten, Konfigurationen und Plattformdiensten führen dazu, dass sich Fehler über mehrere Komponenten hinweg auswirken.
Beispiele sind fehlende Konfigurationswerte, Ressourcenkonflikte, fehlerhafte Netzwerkrouten oder nicht erfüllte Startvoraussetzungen.
Dadurch reicht eine lokale Betrachtung einzelner Logs häufig nicht aus.
Die Analyse erfordert ein Verständnis des Gesamtzusammenhangs und verursacht hohen manuellen Aufwand für die beteiligten Teams.
\par\bigskip

Für die systematische Einordnung dieses Problemfelds liegt in der Plattform eine geeignete Datengrundlage vor.
Logs, Metriken, Traces sowie technische Ereignisse werden zentral erfasst und folgen einheitlichen Formaten mit standardisierten Feldern.
Die Einträge enthalten Zeitstempel sowie Angaben zu Umgebung, Systemkontext, Version und Instanz.
Abläufe über mehrere Komponenten hinweg lassen sich über Korrelationskennungen miteinander in Beziehung setzen und im Nachgang nachvollziehen.
Damit sind zentrale Voraussetzungen gegeben, um wiederkehrende Fehlerbilder zu erkennen und Ursachen systematischer einzugrenzen.
\par\bigskip

Das beschriebene Problemfeld wird dem in Abschnitt 4.3.1 abgeleiteten generischen Use Case der intelligenten Betriebsautomatisierung und Orchestrierung zugeordnet.
Auf dieser Grundlage erfolgt die Bewertung entlang der in Abschnitt 5.1.2 definierten Zusatzdimensionen.

Bezüglich der Umsetzbarkeit sind die erforderlichen Betriebsdaten in strukturierter Form und mit historischer Tiefe verfügbar.
Der technische Fit zur bestehenden Plattform ist gegeben, da die Analyse an die vorhandene Beobachtbarkeit sowie an Zustands- und Eventdaten aus dem Cluster anknüpfen kann.
Zusätzlicher Aufwand entsteht durch die Integration in bestehende Betriebsabläufe sowie durch die Abgrenzung von Analyse und operativer Umsetzung.
Insgesamt ergibt sich eine mittlere Umsetzbarkeit.

Durch die Auswertung von Betriebszuständen und historischen Verläufen kann der manuelle Analyseaufwand reduziert und die Eingrenzung von Störungen beschleunigt werden.
Eine Übertragung auf weitere Cluster und Module ist grundsätzlich möglich, erfordert jedoch stabile Datenqualität und konsistente Konventionen bei Logging und Event-Erfassung.
Im Bewertungsschema wird die Übertragbarkeit als mittel bis hoch eingeordnet.

Da überwiegend technische Betriebsdaten verarbeitet werden, ergeben sich nur geringe Anforderungen an Datenschutz oder Regulierung.
Abstimmungsaufwände betreffen vor allem Zuständigkeiten, Zugriffsrechte und operative Freigaben im Plattformbetrieb.
Die Dimension ist daher als niedrig bis mittel einzuordnen.

Der Reifegrad des Use Case ist insgesamt als mittel einzuordnen.
In der Literatur wird der Ansatz über die reine Konzeptionsphase hinaus beschrieben und in ersten Werkzeugen sowie Pilotanwendungen beschrieben.
Gleichzeitig fehlt jedoch bislang eine breite Etablierung mit klar abgesicherten Betriebsmodellen und Verantwortlichkeiten für den dauerhaften Einsatz.
\par\bigskip

Abbildung \ref{fig:05_03_bewertungsraster} zeigt die Einordnung des Use Case in die Bewertungsmatrix auf Basis der durchgeführten Analyse und Anwendung des Bewertungsrasters.

\begin{figure}[H]
    \centering
    \includegraphics[width=0.7\textwidth]{figures/05_bewertung/05_03_bewertungsraster.png}
    \caption{Einordnung des Use Case in die Bewertungsmatrix}
    \label{fig:05_03_bewertungsraster}
\end{figure}