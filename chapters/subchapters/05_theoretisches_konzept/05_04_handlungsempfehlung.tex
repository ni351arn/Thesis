\section{Handlungsempfehlung}
\label{sec:handlungsempfehlung}
In Kapitel 5.3 wurden die identifizierten Problemfelder analysiert und bewertet. 
Darauf aufbauend werden in diesem Kapitel konkrete Handlungsempfehlungen für den Betrieb der Plattform abgeleitet.
Die Empfehlungen orientieren sich an der priorisierten Einordnung entlang der in Kapitel 5.1 definierten Kriterien und Zusatzdimensionen.
Ziel ist es, betriebliche Entscheidungen zu unterstützen und strukturelle Schwächen im Plattformbetrieb gezielt zu adressieren.
Eine detaillierte technische Umsetzung oder die Auswahl konkreter Werkzeuge ist nicht Gegenstand dieses Abschnitts.

\subsection{Handlungsempfehlung: Proaktives Ressourcenmanagement}
\label{handlungsempfehlungProaktivesRM}
Die folgende Handlungsempfehlung baut auf der Einordnung in Abschnitt 5.3.1 auf. 
Ziel ist es, Engpässe früher zu erkennen und Entscheidungen im Ressourcenmanagement zu stabilisieren.

Ressourcenentscheidungen sollten nicht erst bei akuten Überlastsituationen getroffen werden. 
Stattdessen ist eine vorausschauende Betrachtung der Kapazitätsentwicklung erforderlich, die Auslastung, Anfragevolumen und wiederkehrende Lastmuster berücksichtigt. 
Dadurch lassen sich absehbare Engpässe frühzeitig erkennen, bevor Stabilität oder Verfügbarkeit betroffen sind.

Voraussetzung ist eine belastbare Datengrundlage aus dem Betrieb. 
Relevant sind insbesondere Auslastungswerte, Nachfrageverläufe und zeitliche Nutzungsmuster. 
Auf dieser Basis können Maßnahmen vorbereitet werden, ohne dass Skalierungsentscheidungen im Störungsfall unter Zeitdruck erfolgen.

Das Ressourcenmanagement sollte dabei nicht isoliert auf einzelne Komponenten begrenzt bleiben. 
Entscheidend ist die Wirkung auf den Plattformbetrieb insgesamt. 
Skalierungs- und Kapazitätsentscheidungen müssen nachvollziehbar in die Betriebsabläufe eingebettet sein, damit sie reproduzierbar angewendet werden können.

Die Empfehlung ist mit mittlerem Implementierungsaufwand verbunden, da sie auf vorhandenen Betriebsdaten und Prozessen aufsetzt. 
Der operative Mehrwert ist hoch, weil Entscheidungen früher getroffen werden und der Betrieb weniger reaktiv agiert. 
Die Empfehlung ist skalierbar, sofern die Datenerfassung plattformweit vergleichbar ist und Verantwortlichkeiten klar geregelt sind.

\subsection{Handlungsempfehlung: Automatisierte Release-Absicherung}
\label{handlungsempfehlungAutomatsierterRA}

\subsection{Handlungsempfehlung: Risikobasiertes Schwachstellen- und Compliance-Management}
\label{handlungsempfehlungRisikobasiertesSuCM}