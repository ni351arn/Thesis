\section{Handlungsempfehlung}
\label{sec:handlungsempfehlung}
Auf Basis der Bewertung in \autoref{sec:anwendung_bewertungskonzept} wird eine Umsetzung des Use Case im Plattformbetrieb empfohlen.
Der Schwerpunkt liegt auf der Unterstützung bei der Analyse und Einordnung von Störungen, wobei automatisierte Eingriffe in den Cluster nicht vorgesehen sind.
\par\bigskip

Im ersten Schritt sollte die vorhandene Datengrundlage gezielt für Analysezwecke genutzt werden.
Dazu zählen Kubernetes-Ereignisse, Zustandsinformationen sowie Applikations- und Systemprotokolle, die im Betrieb bereits zentral erfasst werden.
Die Auswertung erfolgt übergreifend, indem zeitliche und funktionale Zusammenhänge zwischen den Einträgen berücksichtigt werden.
Ergänzend sollten Metriken und weitere Betriebsdaten aus dem bestehenden Monitoring einbezogen werden, um Auffälligkeiten in Last, Latenz und Fehlerraten einzuordnen.
\par\bigskip

Damit die Ergebnisse im Betrieb nutzbar sind, sollten Hinweise in einer einheitlichen Form bereitgestellt werden.
Dazu gehören Ursachenhinweise, betroffene Komponenten sowie priorisierte Handlungshinweise, die sich in bestehende Betriebsabläufe übernehmen lassen.
Entscheidungen und Maßnahmen verbleiben bei den verantwortlichen Teams, sodass die Kontrolle über operative Eingriffe erhalten bleibt.
\par\bigskip

Als beispielhafte Umsetzung wird der Einsatz von K8sGPT\footnote{\url{https://k8sgpt.ai}} empfohlen.
Das Werkzeug wird als Open-Source-Lösung beschrieben, die Kubernetes-Cluster analysiert und Hinweise zur Diagnose und Einordnung von Fehlerzuständen bereitstellt \cite{zaaloukCLOUDNATIVEARTIFICIAL}.
Der Schwerpunkt liegt auf der Auswertung technischer Betriebsinformationen, insbesondere von Ereignissen, Zuständen und Protokolldaten aus dem Cluster. 
Dabei werden direkt verfügbare Zustände und Ereignisse aus dem Cluster mit zentralen Betriebsdaten zusammengeführt. 
Der manuelle Aufwand bei der Ursachenanalyse wird dadurch reduziert.
\par\bigskip

Abbildung \ref{fig:k8sgptEinordnung} ordnet K8sGPT im Plattformbetrieb ein und zeigt die für die Analyse genutzten Informationsquellen.
Im dargestellten Zielbild werden Ereignisse und Protokolle aus dem Kubernetes-Cluster als Analysegrundlage herangezogen.
Das GitOps-Repository dient dabei als Referenz für den gewünschten Systemzustand.
Zur verständlichen Aufbereitung der Analyseergebnisse kann ein Sprachmodell unterstützend eingesetzt werden.
Die notwendige Kontextanreicherung erfolgt über eine vorgeschaltete Schnittstelle, die Hinweise aus K8sGPT sowie optionalen Kontext aus GitOps und Runbooks zusammenführt.
Die Ausgabe ist auf erklärende Handlungshinweise beschränkt.
Kontextanreicherung und Sprachmodell sind dabei ausschließlich als unterstützende Bausteine zu verstehen.
Eine eigenständige Analyse oder Entscheidungslogik ist nicht vorgesehen und wird in dieser Arbeit nicht weiter betrachtet.
\par\bigskip

Der Einsatz von K8sGPT ist als Ergänzung bestehender Betriebs- und Analyseprozesse zu verstehen, da das Werkzeug nicht im kritischen Pfad eingesetzt wird.
Es liefert ausschließlich Hinweise, während Entscheidungen und Maßnahmen bei den verantwortlichen Teams verbleiben.
\par\bigskip

Die Handlungsempfehlung beschreibt keinen vollständigen Zielzustand der Plattform.
Sie zeigt, wie der betrachtete Use Case schrittweise umgesetzt werden kann, ohne bestehende Abläufe grundlegend zu verändern.

\begin{figure}[ht]
  \centering
  \includegraphics[width=\textwidth]{figures/05_bewertung/05_04_k8sgptEinordnung2.pdf}
  \caption{Einordnung von K8sGPT im Plattformbetrieb}
  \label{fig:k8sgptEinordnung}
\end{figure}

