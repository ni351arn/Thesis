\section{Handlungsempfehlung}
\label{sec:handlungsempfehlung}
Auf Basis der Einordnung und Bewertung aus Abschnitt 5.3 wird in diesem Kapitel eine konkrete Handlungsempfehlung abgeleitet.
Der Use Case der intelligenten Build-Fehlerdiagnose wurde im Bewertungsraster als hoch hinsichtlich des operativen Mehrwerts und mit mittlerer Umsetzbarkeit eingestuft.
Daraus ergibt sich die Empfehlung, diesen Use Case priorisiert zu adressieren.
\par\bigskip

Auf Grundlage dieser Bewertung sollte die intelligente Build-Fehlerdiagnose als erster Anwendungsfall weiterverfolgt werden. 
Der erwartete Nutzen ergibt sich insbesondere aus der Reduktion des manuellen Analyseaufwands sowie einer schnelleren Eingrenzung von Fehlerursachen in Build- und Bereitstellungsprozessen. 
Gleichzeitig ist der erforderliche Aufwand überschaubar, da die notwendigen Datenquellen bereits vorhanden sind und keine grundlegenden Änderungen an bestehenden Abläufen erforderlich sind.

Eine priorisierte Umsetzung erlaubt es, das identifizierte Problemfeld gezielt zu adressieren und gleichzeitig Erfahrungen im Umgang mit datengetriebenen Analyseansätzen im Plattformbetrieb zu sammeln. 
Der Use Case eignet sich damit sowohl zur operativen Verbesserung als auch als Ausgangspunkt für weitere, darauf aufbauende Anwendungsfälle.
\par\bigskip

Auf Basis der priorisierten Einordnung bietet sich der Einsatz von K8sGPT als beispielhafte Umsetzung des identifizierten Use Case an \cite{kankanalaAIMLDevOps2024}. 
In der Literatur wird K8sGPT als Open-Source-Projekt beschrieben, das den Einsatz von KI zur Unterstützung des Betriebs containerbasierter Systeme adressiert.

Der Schwerpunkt des Werkzeugs liegt auf der Analyse technischer Betriebsinformationen, insbesondere von Protokolldaten aus Kubernetes-Umgebungen. 
Ziel ist es, Betreiber bei der Ursachenanalyse von Fehlerzuständen zu unterstützen und wiederkehrende Probleme schneller nachvollziehbar zu machen. 
Damit greift der Ansatz direkt das in Abschnitt 5.3 beschriebene Problemfeld der Build- und Fehlerdiagnose auf.

Der Einsatz von K8sGPT ist als ergänzende Unterstützung bestehender Betriebs- und Analyseprozesse zu verstehen. 
Entscheidungen und Maßnahmen verbleiben weiterhin bei den verantwortlichen Teams. 
Das Werkzeug eignet sich somit, um den identifizierten Use Case technisch umzusetzen, ohne bestehende Abläufe grundlegend zu verändern.
\par\bigskip

Architekturskizze (Abbildung):
wo das Tool sitzt
welche Daten es nutzt

\par\bigskip
Die Handlungsempfehlung beschreibt keinen vollständigen Zielzustand der Plattform. 
Sie zeigt lediglich, wie der identifizierte Use Case beispielhaft umgesetzt werden kann.



