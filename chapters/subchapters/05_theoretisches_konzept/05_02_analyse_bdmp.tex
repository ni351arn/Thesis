\section{Analyse der Bosch Digital Manufacturing Plattform}
\label{sec:analyse_bmlp}

Die Bosch Digital Manufacturing Plattform (BMLP) ist eine modular aufgebaute, Cloud-Native-Plattform zur Unterstützung von Fertigungs- und Logistikprozessen.
Anwendungen werden als lose gekoppelte Services bereitgestellt und in verteilten Umgebungen betrieben.
Der Betrieb umfasst Edge-Standorte in den Werken, zentrale Rechenzentren sowie cloudbasierte Instanzen innerhalb der Bosch-internen Cloud-Infrastruktur.
Die Plattform ist auf einen hybriden Betrieb ausgelegt und adressiert Anforderungen an Latenz, Verfügbarkeit und Compliance gleichermaßen.

Technisch basiert die Plattform auf Container-Orchestrierung mit Kubernetes, deklarativen Schnittstellen sowie automatisierten \gls{ci}/\gls{cd}-Prozessen und durchgängiger Beobachtbarkeit.
Diese Eigenschaften bilden die technische Grundlage für eine hohe Änderungsfrequenz und einen standardisierten Plattformbetrieb.
Gleichzeitig entstehen umfangreiche Betriebsdaten, die für \gls{ki}-gestützte Verfahren im Plattformbetrieb nutzbar sind.
Alle Informationen basieren auf internen Plattformdokumentationen und Expertengesprächen (nicht öffentlich zugänglich).

\subsection{Architektur und Betriebsmodell}
\label{sec:architektur}
Fachliche Funktionen in der BMLP werden als eigenständige Module in einer verteilten und modularen Grundstruktur umgesetzt.
Jedes Modul bringt typischerweise eigene Logik und eigene Datenhaltung mit, wodurch direkte Abhängigkeiten zwischen Modulen vermieden werden.
Die Zusammenarbeit zwischen Modulen erfolgt überwiegend ereignisbasiert, während synchrone Schnittstellen nur für spezielle Anwendungsfälle genutzt werden.

In jedem Werk wird ein eigenständiges Kubernetes-Cluster betrieben.
Ergänzend existieren zentrale Cluster für übergreifende Plattformdienste sowie separate Werks-Cluster für latenzkritische Funktionen.
So werden Echtzeitanforderungen sichergestellt und zentrale Cluster im Multi-Tenant-Betrieb genutzt.

Für den werks- und organisationsübergreifenden Einsatz unterstützt die Plattform eine klare Trennung von Zugriffen und Verantwortlichkeiten.
Diese wird durch zentrale Plattformdienste und ein gemeinsames Rollen- und Berechtigungsmodell umgesetzt.
Dadurch können mehrere Organisationseinheiten dieselbe Plattform nutzen, ohne die Zugriffskontrolle zu verlieren.

Im Betrieb werden zentrale Mechanismen möglichst vereinheitlicht.
Dazu zählen gemeinsame Einstiegspunkte für Anwendungen sowie einheitliche Informationen zum Betriebszustand der Module.
Monitoring und Supportprozesse bauen auf diesen Grundlagen auf und werden plattformweit einheitlich umgesetzt.

\subsection{Entwicklungsmodell und Plattformstandards}
\label{sec:entwicklungsmodell}
Die Entwicklung der Module erfolgt intern und ist über mehrere Repositories organisiert.
Zentrale Entwicklungsschritte wie Build, Tests, Container-Erstellung und Sicherheitsprüfungen sind in standardisierten CI-Pipelines automatisiert.
Die daraus entstehenden Artefakte werden versioniert abgelegt und für den weiteren Betrieb bereitgestellt.

Auch die Auslieferung folgt einem weitgehend automatisierten Ansatz.
Ein wiederverwendbares Pipeline-Modell verbindet Artefaktverwaltung, Konfigurationsverwaltung sowie die Verwaltung von Schlüsseln und Zugangsdaten zu einem durchgängigen Prozess.
Die Bereitstellung erfolgt reproduzierbar und nachvollziehbar unter Berücksichtigung regulatorischer Anforderungen.

Ergänzend definieren Plattformstandards zentrale Leitplanken für Entwicklung und Betrieb.
Dazu zählen einheitliche Vorgaben für Container-Images und deklarative Infrastruktur.
Schnittstellen werden nach gemeinsamen Gestaltungsregeln umgesetzt und konsistent dokumentiert, während die Kommunikation zwischen Modulen überwiegend ereignisbasiert erfolgt.
Für den Betrieb werden einheitliche Logformate und durchgängige Referenzen genutzt, sodass Abläufe über mehrere Komponenten hinweg nachvollziehbar bleiben.

\subsection{Operativer Stack und Datenbasis für AIOps}
\label{subsec:operativerStack}
Für \gls{aiops} ist weniger die Tool-Landschaft entscheidend als die Daten, die im Plattformbetrieb entstehen.
In der BMLP werden Betriebsdaten über Logs, Metriken, Traces sowie technische und fachliche Ereignisse erfasst und zentral nutzbar gemacht.
Logs liegen in einem einheitlichen Format vor und enthalten neben Zeitstempel und Systemkontext auch Versions- und Instanzinformationen sowie Korrelationskennungen.
Damit lassen sich Abläufe über mehrere Komponenten hinweg zusammenführen und automatisiert auswerten.

Metriken und Traces ergänzen diese Sicht um messbare Größen wie Last, Latenz und Fehlerraten sowie um Ablaufspuren über Servicegrenzen hinweg.
Ereignisdaten bilden Zustandswechsel und Prozessketten ab und unterstützen dadurch die Einordnung von Störungen.
Zusätzlich entstehen Daten aus Build- und Rollout-Prozessen, etwa Testergebnisse, Sicherheitsprüfungen und Rollout-Verläufe, die Rückschlüsse auf Änderungsrisiken erlauben.
Auch Änderungen an Konfigurationen können Hinweise auf Abweichungen zwischen geplantem und aktuellem Zustand liefern.

Auf dieser Datenbasis lassen sich insbesondere Auffälligkeiten frühzeitig erkennen und Störungen schneller eingrenzen.
Ebenso können Rollouts datenbasiert überwacht und Kapazitätsbedarfe besser abgeschätzt werden.
