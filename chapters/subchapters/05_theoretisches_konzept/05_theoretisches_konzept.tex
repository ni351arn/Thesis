\chapter{Anwendung der Literaturrecherche}
\label{ch:anwendungLiteraturrecherche}

In diesem Kapitel wird das in Kapitel \ref{ch:literaturErgebnisse} entwickelte theoretische Konzept zur Bewertung und Priorisierung von KI-Use-Cases im Cloud-Native Platform Engineering angewendet.
Zunächst wird in Kapitel 5.1 ein Bewertungskonzept vorgstellt, das die Grundlage für die Einordnung und Analyse der identifizierten Use Cases bildet.
Anschließend wird in Kapitel 5.2 die Bosch Digital Manufacturing Plattform (BMLP) analysiert, um einen praxisnahen Kontext für die Anwendung des Bewertungskonzepts zu schaffen.
Dafür werden bekannte Pain-Points innerhalb der Plattform identifiziert.
Abschließend wird in Kapitel 5.3 das zuvor entwickelte Bewertungskonzept konkret auf die identifizierten Pain-Points angwendet und es wird eine Handlungsempfehlung ausgesprochen. 


\section{Bewertungskonzept}
\label{sec:bewertungskonzept}
Aufbauend auf den zuvor identifizierten KI-Use-Cases wird im Folgenden ein Bewertungskonzept vorgestellt, das eine strukturierte Einordnung und Vergleichbarkeit dieser Anwendungsfälle ermöglicht.
Ziel des Frameworks ist es, KI-Anwendungsfälle im Cloud-Native Platform Engineering systematisch hinsichtlich ihres Implementierungsaufwands und ihres operativen Mehrwerts zu bewerten.
Damit soll eine fundierte Entscheidungsgrundlage für die Priorisierung und Implementierung von KI-Lösungen geschaffen werden.

Implementierungsaufwand beschreibt dabei den Aufwand, eine KI-Funktion robust, sicher und wartbar produktiv zu integrieren und zu betreiben.
Operativer Mehrwert beschreibt den erwarteten Beitrag zur messbaren Verbesserung zentraler Betriebsziele, insbesondere der Betriebszuverlässigkeit (z.\,B. MTTR), der Effizienz sowie der Systemstabilität.
Als weitere betriebliche Zielgrößen können Service Level Objectives (SLOs) herangezogen werden, die messbare Zielwerte für Servicequalität (z.\,B. Verfügbarkeit oder Latenz) definieren.
Aspekte wie Datenzugang, Governance und organisatorische Voraussetzungen werden bewusst nicht in der X/Y-Achse abgebildet, sondern in zusätzlichen Dimensionen vertieft betrachtet.

Die erste Bewertungsdimension ist der Implementierungsaufwand, der auf der X-Achse der Bewertungsmatrix abgebildet wird.
Tabelle \ref{tab:xAchse} konkretisiert diese Dimension anhand der qualitativen Ausprägung von niedrig bis hoch.
Berücksichtigt werden dabei insbesondere der Grad der erforderlichen Modellanpassungen, der Trainings- und Anpassungsaufwand sowie die Komplexität des operativen Betriebs.
Darüber hinaus fließen Integrations- und Betriebsaspekte in die Einordnung ein, z.\,B. Anforderungen an Datenpipelines, Security-Mechanismen, Observability, Evaluation und laufende Qualitätssicherung.
Ein niedriger Implementierungsaufwand beschreibt demnach den Einsatz bestehender KI-Funktionen oder Standardwerkzeuge mit geringem Integrations- und Betriebsaufwand.
Ein hoher Aufwand hingegen ist typischerweise mit umfangreicher Integration und eigener Anpassung (z.\,B. RAG mit Datenpipeline und Evaluation oder Modelltraining) sowie kontinuierlicher Überwachung, Wartung und Qualitätssicherung verbunden.
\begin{table}[H]
\caption{X-Achse: Implementierungsaufwand }
\label{tab:xAchse}
\footnotesize
\renewcommand{\arraystretch}{1.2}
\setlength{\tabcolsep}{7pt}

\begin{tabularx}{\linewidth}{p{3.5cm} X}
\toprule
\textbf{Ausprägung} & \textbf{Beschreibung} \\
\midrule

Niedrig &
Einsatz vorhandener \gls{ki}-Funktionen oder Standardtools ohne eigenes Modelltraining.
\\

Mittel &
Anpassung bestehender Modelle, Feature Engineering, begrenztes Retraining. 
\\

Hoch &
Eigenes Modelltraining, kontinuierlicher Betrieb und Überwachung notwendig.
\\


\bottomrule
\end{tabularx}
\end{table}

\par\bigskip

Die zweite Dimension bildet der operative Mehrwert, der auf der Y-Achse dargestellt wird und in Tabelle \ref{tab:yAchse} näher beschrieben ist.
Hier wird der tatsächliche Beitrag eines KI-Use-Cases zur Verbesserung des Plattformbetriebs erfasst.
Ein niedriger operativer Mehrwert umfasst vor allem unterstützende Funktionen mit begrenztem Einfluss auf zentrale Betriebskennzahlen.
Ein hoher Mehrwert hingegen zeigt sich in deutlichen Verbesserungen wesentlicher KPIs und Betriebsziele, z.\,B. MTTR, SLO-Erfüllung, Error-Budget-Verbrauch, Infrastrukturkosten oder Systemstabilität.
\begin{table}[H]
\caption{Y-Achse: Operativer Mehrwert}
\label{tab:yAchse}
\footnotesize
\renewcommand{\arraystretch}{1.2}
\setlength{\tabcolsep}{7pt}

\begin{tabularx}{\linewidth}{p{3.5cm} X}
\toprule
\textbf{Ausprägung} & \textbf{Beschreibung} \\
\midrule

Niedrig &
Geringer Einfluss auf den Plattformbetrieb, primär unterstützende Funktionen wie Log-Analyse oder einfache Anomalieerkennung.
\\

Mittel &
Messbare Effizienzsteigerung oder Zeitersparnis im Plattformbetrieb durch KI-gestützte Automatisierung oder Optimierung.
\\

Hoch &
Deutliche Verbesserung zentraler KPIs (z.B. MTTR, Kostenreduktion, Stabilität)
\\


\bottomrule
\end{tabularx}
\end{table}

\par\bigskip

Abbildung \ref{fig:05_01_bewertungsMatrix} führt beide Dimensionen in einer zweidimensionalen Bewertungsmatrix zusammen.
Hierfür wurde die Matrix in vier Quadranten unterteilt, wodurch eine qualitative Einordnung der KI-Use-Cases in Bezug auf das Verhältnis von Implementierungsaufwand und operativem Mehrwert ermöglicht wird.
Anwendungsfälle mit hohem Mehrwert und niedrigem Aufwand weisen ein günstiges Aufwand-Nutzen-Verhältnis auf und eignen sich als priorisierte Kandidaten für eine erste Umsetzung.
Use Cases mit hohem Mehrwert und hohem Aufwand sind strategische Kandidaten, die typischerweise eine Roadmap, Vorarbeiten (z.\,B. Daten- und Integrationsgrundlagen) sowie ein risikobewusstes Vorgehen erfordern.
Hingegen sollten Anwendungsfälle mit hohem Aufwand und geringem Mehrwert nur weiterverfolgt werden, wenn sie notwendige Vorstufen für höherwertige Use Cases darstellen oder klare Abhängigkeiten bestehen.
\begin{figure}[H]
    \centering
    \includegraphics[width=0.6\textwidth]{figures/05_bewertung/05_01_bewertungsMatrix.png}
    \caption{Bewertungsmatrix zur Einordnung von KI-Use-Cases im Cloud-Native Platform Engineering entlang von Implementierungsaufwand und operativem Mehrwert}
    \label{fig:05_01_bewertungsMatrix}
\end{figure}
\par\bigskip

Die zweidimensionale Bewertungsmatrix ermöglicht eine erste Priorisierung von KI-Use-Cases anhand des Verhältnisses von Implementierungsaufwand und operativem Mehrwertpotenzial.
Sie dient als Ausgangspunkt für eine vertiefende Bewertung entlang zusätzlicher Faktoren, die in einer rein zweidimensionalen Darstellung nur unzureichend abbildbar sind.
Die zusätzlichen Dimensionen erweitern die Matrix dabei nicht um weitere Achsen, sondern dienen der kontextbezogenen Bewertung der Realisierbarkeit und der betrieblichen Einbettung einzelner Use Cases.
\par\bigskip

\textbf{Dimension 1: Umsetzbarkeit}

Die erste zusätzliche Dimension beschreibt die technische und organisatorische Umsetzbarkeit eines KI-Use-Cases.
Sie umfasst (i) die Verfügbarkeit und Qualität geeigneter Datenquellen, (ii) den technischen Fit mit der bestehenden Plattformarchitektur sowie die Integrationsfähigkeit und (iii) das erforderliche Know-how für Entwicklung, Betrieb und Wartung.
Für die Bewertung wird eine einheitliche dreistufige Skala (niedrig/mittel/hoch) verwendet.
Niedrig liegt vor, wenn Datenzugang und Integrationspfade vorhanden sind und das erforderliche Know-how im Team verfügbar ist.
Hoch liegt vor, wenn Datenlücken, fehlende Schnittstellen oder signifikanter Kompetenzaufbau erforderlich sind.
\par\bigskip

\textbf{Dimension 2: Betriebswirksamkeit \& Skalierbarkeit}

Die zweite Dimension bewertet, ob sich der erwartete operative Mehrwert eines KI-Use-Cases im täglichen Plattformbetrieb zuverlässig realisieren lässt.
Zusätzlich wird betrachtet, ob die Lösung über Services und Teams hinweg skalierbar ist.
Bewertet werden (i) die Betriebswirksamkeit, z.\,B. Robustheit der Ergebnisse, Fehlalarme, Drift-Risiken und laufender operativer Aufwand, sowie (ii) die Skalierbarkeit in Bezug auf Wiederverwendbarkeit und Rollout-Fähigkeit.
Für die Bewertung wird eine einheitliche dreistufige Skala (niedrig/mittel/hoch) verwendet.
Niedrig beschreibt einen lokalen, stark kontextabhängigen oder schwer reproduzierbaren Effekt mit begrenzter Skalierbarkeit.
Hoch beschreibt eine im Betrieb verlässlich wirksame Lösung mit breiter Übertragbarkeit über Services und Teams hinweg.
\par\bigskip

\textbf{Dimension 3: Governance \& Reifegrad}

Die dritte Dimension berücksichtigt Governance-Anforderungen im Sinne von Compliance- und Security-Vorgaben.
Bewertet werden (i) Anforderungen an Datenzugriff, Datenschutz und Sicherheitsmaßnahmen sowie (ii) der Reifegrad der jeweiligen Lösung (Konzeptnachweis, Pilotierung oder produktiv erprobt).
Zur Einordnung dienen regulatorische Vorgaben, Sicherheitsanforderungen und der Nachweis operativer Stabilität.
Niedrig liegt vor, wenn geringe Compliance-Hürden bestehen und die Lösung produktiv erprobt ist.
Hoch liegt vor, wenn sensible Daten, strenge Vorgaben oder ein niedriger Reifegrad (z.\,B. Konzeptnachweis) vorliegen.
\par\bigskip

Zur einheitlichen Anwendung der Zusatzdimensionen werden die qualitativen Bewertungen in Tabelle \ref{tab:zusatzdimensionen} zusammengefasst.
\begin{table}[H]
\caption{Bewertungsskalen der Zusatzdimensionen}
\label{tab:zusatzdimensionen}
\footnotesize
\renewcommand{\arraystretch}{1.2}
\setlength{\tabcolsep}{7pt}

\begin{tabularx}{\linewidth}{p{4.2cm} p{2.2cm} X}
\toprule
\textbf{Zusatzdimension} & \textbf{Ausprägung} & \textbf{Beschreibung} \\
\midrule

\makecell[l]{Umsetzbarkeit} & Niedrig &
Daten, Integrationspfade und erforderliches Know-how sind vorhanden; Umsetzung ohne strukturelle Änderungen möglich. \\
& Mittel &
Vorarbeiten bei Datenverfügbarkeit, Integration oder Kompetenzen notwendig. \\
& Hoch &
Zentrale Daten, Schnittstellen oder Fähigkeiten fehlen aktuell und erfordern zusätzliche Vorarbeiten. \\
\midrule

\makecell[l]{Betriebswirksamkeit\\\& Skalierbarkeit} & Niedrig &
Wirkung lokal oder instabil; stark kontextabhängig und nur schwer auf weitere Services oder Teams übertragbar. \\
& Mittel &
Stabile Wirkung in mehreren Szenarien, jedoch mit begrenzter Skalierbarkeit oder erhöhtem operativem Aufwand. \\
& Hoch &
Zuverlässig wirksam im Plattformbetrieb und plattformweit skalierbar ohne individuellen Zusatzaufwand. \\
\midrule

\makecell[l]{Governance\\\& Reifegrad} & Niedrig &
Geringe Governance- und Compliance-Hürden bei produktiv erprobtem Reifegrad. \\
& Mittel &
Zusätzliche Anforderungen an Dokumentation, Compliance oder organisatorische Abstimmung erforderlich. \\
& Hoch &
Hohe regulatorische Anforderungen oder geringer Reifegrad (z.\,B.\ Proof of Concept oder Forschung). \\
\bottomrule
\end{tabularx}
\end{table}

\par\bigskip

In Abschnitt 5.2 werden die identifizierten KI-Use-Cases zunächst in die Bewertungsmatrix eingeordnet (X/Y).
Anschließend werden sie entlang der drei Zusatzdimensionen qualitativ bewertet, um Umsetzbarkeit, Betriebswirksamkeit und Governance-Aspekte kontextbezogen zu prüfen.
Dadurch entsteht eine priorisierte und zugleich praxisnahe Grundlage, um identifizierte Pain Points der Plattform gezielt mit geeigneten KI-Use-Cases zu adressieren und daraus Handlungsempfehlungen abzuleiten.

\clearpage 
\section{Analyse der Bosch Digital Manufacturing Plattform}
\label{sec:analyse_bmlp}

Die Bosch Digital Manufacturing Plattform (BMLP) ist eine modular aufgebaute, Cloud-Native-Plattform zur Unterstützung von Fertigungs- und Logistikprozessen.
Anwendungen werden als lose gekoppelte Services bereitgestellt und in verteilten Umgebungen betrieben.
Dazu zählen Edge-Standorte in den Werken, zentrale Rechenzentren sowie cloudbasierte Instanzen.
Die Plattform ist auf einen hybriden Betrieb ausgelegt und adressiert Anforderungen an Latenz, Verfügbarkeit und Compliance gleichermaßen.

Cloud-Native ist die Plattform, da sie konsequent auf Container-Orchestrierung mit Kubernetes, deklarative Schnittstellen, automatisierte CI/CD-Prozesse sowie durchgängige Beobachtbarkeit setzt.
Diese Eigenschaften bilden die technische Grundlage für eine hohe Änderungsfrequenz und einen standardisierten Plattformbetrieb.
Gleichzeitig entstehen umfangreiche Betriebsdaten, die für KI-gestützte Verfahren im Plattformbetrieb nutzbar sind.

\subsection{Architektur und Betriebsmodell}
\label{sec:architektur}
Pro Werk wird typischerweise eine eigene Plattforminstanz als Kubernetes-Cluster betrieben.
Zentrale und cloudbasierte Instanzen ergänzen die Edge-Workloads.
Ziel ist ein hybrider Multi-Cluster-Betrieb mit standardisierten Lebenszyklen, etwa für Provisionierung, Updates und Patches.

Deployments sind auf häufige Änderungen ausgelegt.
Dazu werden Zero-Downtime-Strategien wie Rolling Updates, Blue-Green oder Canary verwendet.

Der Plattform-Stack wird hier bewusst nur auf hoher Ebene beschrieben.
Die Basis bilden Kubernetes sowie grundlegende Infrastrukturbausteine wie Registry, Netzwerk, Storage und Secrets-Management.
Darüber liegen zentrale Plattformfunktionen für Zugriff, Schnittstellen und Integration.
Fachliche Anwendungen werden als eigenständige Services betrieben und kommunizieren über REST-Schnittstellen oder Events.


\subsection{Entwicklungsmodell und Plattformstandards}
\label{sec:entwicklungsmodell}


\subsection{Operativer Stack und Datenbasis für AI for Ops}
\label{sec:operativerStack}

\clearpage 
\section{Anwendung des Bewertungskonzepts und Handlungsempfehlungen}
\label{sec:handlungsempfehlung}

An Bosch gerichtet
