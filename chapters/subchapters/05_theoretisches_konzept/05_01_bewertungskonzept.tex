\section{Bewertungskonzept}
\label{sec:bewertungskonzept}
Aufbauend auf den zuvor identifizierten KI-Use-Cases wird im Folgenden ein Bewertungskonzept vorgestellt, das eine strukturierte Einordnung und Vergleichbarkeit dieser Anwendungsfälle ermöglicht.
Ziel des Frameworks ist es, KI-Anwendungsfälle im Cloud-Native Platform Engineering systematisch hinsichtlich ihres Implementierungsaufwands und ihres operativen Mehrwerts zu bewerten.
Damit soll eine fundierte Entscheidungsgrundlage für die Priorisierung und Implementierung von KI-Lösungen geschaffen werden.

\subsection{Ziel und Einordnung der Bewertungslogik}
\label{zielBewertungslogik}
Der Implementierungsaufwand beschreibt den Aufwand, eine KI-Funktion robust, sicher und wartbar in den Betrieb zu integrieren.
Der operative Mehrwert beschreibt den erwarteten Beitrag zur messbaren Verbesserung zentraler Betriebsziele.
Dazu zählen insbesondere Betriebszuverlässigkeit (z.\,B. MTTR), Effizienz und Systemstabilität.
Aspekte wie Datenzugang, Steuerung und organisatorische Voraussetzungen werden nicht in der X/Y-Achse abgebildet, sondern über zusätzliche Dimensionen betrachtet.

Die X-Achse bildet den Implementierungsaufwand ab.
Tabelle \ref{tab:xAchse} beschreibt die Ausprägungen von niedrig bis hoch.
Niedrig bedeutet, dass sich der Use Case mit vorhandenen Funktionen oder Standardwerkzeugen umsetzen lässt und der Integrationsaufwand gering bleibt.
Hoch bedeutet, dass zusätzliche Entwicklung, tiefere Integration und ein spürbarer Betriebsaufwand notwendig sind.
\begin{table}[H]
\caption{X-Achse: Implementierungsaufwand }
\label{tab:xAchse}
\footnotesize
\renewcommand{\arraystretch}{1.2}
\setlength{\tabcolsep}{7pt}

\begin{tabularx}{\linewidth}{p{3.5cm} X}
\toprule
\textbf{Ausprägung} & \textbf{Beschreibung} \\
\midrule

Niedrig &
Einsatz vorhandener \gls{ki}-Funktionen oder Standardtools ohne eigenes Modelltraining.
\\

Mittel &
Anpassung bestehender Modelle, Feature Engineering, begrenztes Retraining. 
\\

Hoch &
Eigenes Modelltraining, kontinuierlicher Betrieb und Überwachung notwendig.
\\


\bottomrule
\end{tabularx}
\end{table}

\par\bigskip

Die Y-Achse bildet den operativen Mehrwert ab.
Tabelle \ref{tab:yAchse} konkretisiert auch hier die Ausprägungen von niedrig bis hoch.
Niedrig steht für unterstützende Funktionen, die zentrale Betriebsgrößen nur gering beeinflussen.
Hoch liegt vor, wenn sich messbare Verbesserungen zeigen, etwa bei MTTR, Infrastrukturkosten oder der Systemstabilität.
\begin{table}[H]
\caption{Y-Achse: Operativer Mehrwert}
\label{tab:yAchse}
\footnotesize
\renewcommand{\arraystretch}{1.2}
\setlength{\tabcolsep}{7pt}

\begin{tabularx}{\linewidth}{p{3.5cm} X}
\toprule
\textbf{Ausprägung} & \textbf{Beschreibung} \\
\midrule

Niedrig &
Geringer Einfluss auf den Plattformbetrieb, primär unterstützende Funktionen wie Log-Analyse oder einfache Anomalieerkennung.
\\

Mittel &
Messbare Effizienzsteigerung oder Zeitersparnis im Plattformbetrieb durch KI-gestützte Automatisierung oder Optimierung.
\\

Hoch &
Deutliche Verbesserung zentraler KPIs (z.B. MTTR, Kostenreduktion, Stabilität)
\\


\bottomrule
\end{tabularx}
\end{table}

\par\bigskip

Abbildung \ref{fig:05_01_bewertungsMatrix} führt beide Dimensionen in einer Bewertungsmatrix zusammen.
Die Matrix ist in vier Quadranten unterteilt, wodurch sich Anwendungsfälle qualitativ nach Aufwand und Mehrwert einordnen lassen.
Anwendungsfälle mit hohem Mehrwert und niedrigem Aufwand eignen sich für eine frühe Umsetzung.
Use Cases mit hohem Mehrwert und hohem Aufwand bleiben relevant, erfordern jedoch typischerweise Vorarbeiten, etwa bei Daten- und Integrationsgrundlagen.
Anwendungsfälle mit hohem Aufwand und geringem Mehrwert sollten nur weiterverfolgt werden, wenn klare Abhängigkeiten bestehen.
\begin{figure}[H]
    \centering
    \includegraphics[width=0.6\textwidth]{figures/05_bewertung/05_01_bewertungsMatrix.png}
    \caption{Bewertungsmatrix zur Einordnung von KI-Use-Cases entlang von Implementierungsaufwand und operativem Mehrwert}
    \label{fig:05_01_bewertungsMatrix}
\end{figure}


\subsection{Zusatzdimensionen und Bewertungsstufen}
\label{zusatzdimensionenBewertung}
Die zweidimensionale Bewertungsmatrix priorisiert KI-Use-Cases zunächst über das Verhältnis von Implementierungsaufwand (X) und operativem Mehrwert (Y). 
Sie liefert damit eine erste Einordnung, greift aber zentrale Rahmenbedingungen des Plattformbetriebs nur indirekt auf.

Daher werden ergänzend vier Zusatzdimensionen betrachtet. 
Diese dienen als Bewertungsgrundlage, um Implementierungsaufwand (X) und operativen Mehrwert (Y) nachvollziehbar abzuleiten.
Die Bewertung erfolgt je Unterdimension einheitlich auf einer dreistufigen Skala (niedrig/mittel/hoch).
Die Zusatzdimensionen wirken nicht direkt auf die Achsen der Bewertungsmatrix, fließen jedoch in die spätere Einordnung der Use Cases ein.
\par\bigskip
\textbf{Dimension 1: Umsetzbarkeit}

Die Dimension Umsetzbarkeit bewertet, wie realistisch ein KI-Use-Case technisch und organisatorisch umgesetzt werden kann. 
Sie umfasst (i) Datenverfügbarkeit und -qualität, (ii) die Passung zur bestehenden Plattform- bzw. CI/CD-Architektur (inkl. Integration) sowie (iii) die verfügbaren Fachkenntnisse für Entwicklung, Betrieb und Wartung.

Niedrig bedeutet, dass Daten ausreichend vorhanden sind, die Integration ohne größere Änderungen möglich ist und die nötige Expertise im Team verfügbar ist.

Mittel bedeutet, dass einzelne Vorarbeiten nötig sind (z.B. Datenaufbereitung, kleinere Anpassungen in Schnittstellen/Prozessen oder gezielter Kompetenzaufbau).

Hoch bedeutet, dass wesentliche Voraussetzungen fehlen (z.B. Datenlücken, fehlende Integrationsmöglichkeiten oder deutlicher Kompetenzaufbau), sodass der Use Case nur mit spürbarem Vorlauf realisierbar ist.
\par\bigskip

\textbf{Dimension 2: Betriebswirksamkeit \& Skalierbarkeit}

Diese Dimension bewertet, ob ein KI-Use-Case im laufenden Plattformbetrieb zuverlässig einen operativen Nutzen entfaltet.
Zusätzlich wird betrachtet, ob sich der Ansatz mit vertretbarem Aufwand auf weitere Services oder Teams übertragen lässt.
Berücksichtigt werden (i) die Stabilität und Wirksamkeit im Betrieb sowie (ii) die Skalierbarkeit und Wiederverwendbarkeit der Lösung.

Niedrig bedeutet, dass der Nutzen stark kontextabhängig, instabil oder auf einzelne Services begrenzt ist.

Mittel bedeutet, dass ein erkennbarer Mehrwert im Betrieb entsteht, die Übertragbarkeit jedoch nur eingeschränkt gegeben ist oder zusätzlichen Anpassungsaufwand erfordert.

Hoch bedeutet, dass die Lösung stabil im Betrieb wirkt und mit geringem Zusatzaufwand breit ausgerollt werden kann.
\par\bigskip

\textbf{Dimension 3: Compliance \& Governance}

Diese Dimension erfasst organisatorische und regulatorische Rahmenbedingungen für den Einsatz eines KI-Use-Cases.
Bewertet werden (i) Anforderungen an Datenschutz, Datenklassifikation und Zugriffskontrolle sowie (ii) die organisatorische Verankerung durch klare Zuständigkeiten und Freigabeprozesse.
Die Dimension dient dazu, Risiken und Abstimmungsbedarfe frühzeitig sichtbar zu machen.

Niedrig bedeutet, dass geringe regulatorische Anforderungen bestehen und Verantwortlichkeiten klar geregelt sind.

Mittel bedeutet, dass zusätzliche Abstimmungen oder formale Freigaben erforderlich sind, jedoch gut handhabbar.

Hoch bedeutet, dass strenge Vorgaben, sensible Daten oder unklare Zuständigkeiten zu erhöhtem Abstimmungs- und Umsetzungsaufwand führen.
\par\bigskip

\textbf{Dimension 4: Reifegrad}

Die Dimension Reifegrad bewertet den Entwicklungs- und Betriebsstand eines KI-Use-Cases.
Sie umfasst (i) die Technologiereife der Lösung sowie (ii) die Absicherung des dauerhaften Betriebs.
Berücksichtigt werden unter anderem der Entwicklungsstatus, vorhandene Betriebserfahrungen und die organisatorische Absicherung.

Niedrig bedeutet, dass der Use Case als Konzept oder Prototyp vorliegt und belastbare Erfahrungen im Regelbetrieb fehlen.

Mittel bedeutet, dass der Use Case pilotiert oder eingeschränkt produktiv eingesetzt wurde, mit noch offenen Betriebsfragen.

Hoch bedeutet, dass die Lösung produktiv im Einsatz ist und technisch sowie organisatorisch abgesichert betrieben wird.
\par\bigskip

Zur einheitlichen Anwendung der Zusatzdimensionen werden die qualitativen Bewertungen in Tabelle \ref{tab:zusatzdimensionen} zusammengefasst.
\begin{table}[H]
\caption{Bewertungsskalen der Zusatzdimensionen}
\label{tab:zusatzdimensionen}
\footnotesize
\renewcommand{\arraystretch}{1.2}
\setlength{\tabcolsep}{7pt}

\begin{tabularx}{\linewidth}{p{4.2cm} p{2.2cm} X}
\toprule
\textbf{Zusatzdimension} & \textbf{Ausprägung} & \textbf{Beschreibung} \\
\midrule

\makecell[l]{Umsetzbarkeit} & Niedrig &
Daten, Integrationspfade und erforderliches Know-how sind vorhanden; Umsetzung ohne strukturelle Änderungen möglich. \\
& Mittel &
Vorarbeiten bei Datenverfügbarkeit, Integration oder Kompetenzen notwendig. \\
& Hoch &
Zentrale Daten, Schnittstellen oder Fähigkeiten fehlen aktuell und erfordern zusätzliche Vorarbeiten. \\
\midrule

\makecell[l]{Betriebswirksamkeit\\\& Skalierbarkeit} & Niedrig &
Wirkung lokal oder instabil; stark kontextabhängig und nur schwer auf weitere Services oder Teams übertragbar. \\
& Mittel &
Stabile Wirkung in mehreren Szenarien, jedoch mit begrenzter Skalierbarkeit oder erhöhtem operativem Aufwand. \\
& Hoch &
Zuverlässig wirksam im Plattformbetrieb und plattformweit skalierbar ohne individuellen Zusatzaufwand. \\
\midrule

\makecell[l]{Governance\\\& Reifegrad} & Niedrig &
Geringe Governance- und Compliance-Hürden bei produktiv erprobtem Reifegrad. \\
& Mittel &
Zusätzliche Anforderungen an Dokumentation, Compliance oder organisatorische Abstimmung erforderlich. \\
& Hoch &
Hohe regulatorische Anforderungen oder geringer Reifegrad (z.\,B.\ Proof of Concept oder Forschung). \\
\bottomrule
\end{tabularx}
\end{table}


\subsection{Ableitung der X- und Y-Achse und Quadrantenzuordnung}
\label{ableitungXYAchse}
Die Einordnung eines KI-Use-Cases in der Bewertungsmatrix erfolgt auf Basis eines Bewertungsrasters.
Das Raster umfasst die Unterdimensionen der vier Zusatzdimensionen aus Abschnitt \ref{zusatzdimensionenBewertung}.
Für jede Unterdimension wird die zugehörige Leitfrage beantwortet und auf einer dreistufigen Skala bewertet (niedrig/mittel/hoch).
Zur Auswertung werden die Stufen in Zahlenwerte überführt (niedrig = 1, mittel = 2, hoch = 3).

Die Werte fpr X und Y ergeben sich aus den Einzelbewertungen der Unterdimensionen.
Kriterien mit Bezug zum Implementierungsaufwand gehen in den X-Wert ein, Kriterien mit Bezug zum operativen Nutzen in den Y-Wert.
Unterdimensionen, die beide Perspektiven betreffen, werden entsprechend in beiden Achsen berücksichtigt.

Der Achsenwert ergibt sich je Use Case als Mittelwert der zugeordneten Unterdimensionen.
Dadurch entsteht eine Position in der Bewertungsmatrix, die in das X/Y-Diagramm übertragen wird.
Da Mittelwerte verwendet werden, kann die Position auch zwischen Quadranten liegen.
Dies bildet gemischte Bewertungen ab, etwa wenn einzelne Voraussetzungen erfüllt sind, andere jedoch noch Vorarbeiten erfordern.

Das Bewertungsraster ist als Excel-Vorlage umgesetzt und im Anhang dokumentiert.
Es ermöglicht eine einheitliche, nachvollziehbare und reproduzierbare Einordnung der Use Cases.
