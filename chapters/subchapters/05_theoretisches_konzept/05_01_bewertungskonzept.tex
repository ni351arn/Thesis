\section{Bewertungskonzept/ Framework}
\label{sec:framework}
Aufbauend auf den zuvor identifizierten KI-Use-Cases wird im Folgenden ein Bewertungskonzept vorgestellt, das eine strukturierte Einordnung und Vergleichbarkeit dieser Anwendungsfälle ermöglicht.
Ziel des Frameworks ist es, KI-Anwendungsfälle im Cloud-Native Plattform Engineering systematisch hinsichtlich ihres Implementierungsaufwands und ihres operativen Mehrwerts zu bewerten.
Damit soll eine fundierte Entscheidungsgrundlage für die Priorisierung und Implementierung von KI-Lösungen geschaffen werden.
Aspekte wie Datenzugang, Governance und organisatorische Voraussetzungen werden bewusst nicht in der X/Y-Achse abgebildet, sondern in den Zusatzdimensionen bewertet.

Die erste Bewertungsdimension ist der Implementierungsaufwand, der auf der X-Achse der Bewertungsmatrix abgebildet wird.
Tabelle \ref{tab:xAchse} konkretisiert diese Dimension anhand der qualitativen Ausprägung von niedrig bis hoch.
Berücksichtigt werden dabei insbesondere der Grad der erforderlichen Modellanpassungen, der Trainingsaufwand sowie die Komplexität des operativen Betriebs.
Ein niedriger Implementierungsaufwand beschreibt demnach den Einsatz bestehender KI-Funktionen oder Standardwerkzeuge. 
Ein hoher Aufwand hingegen ist mit eigenem Modelltraining sowie kontinuierlicher Überwachung und Wartung verbunden.
\begin{table}[H]
\caption{X-Achse: Implementierungsaufwand }
\label{tab:xAchse}
\footnotesize
\renewcommand{\arraystretch}{1.2}
\setlength{\tabcolsep}{7pt}

\begin{tabularx}{\linewidth}{p{3.5cm} X}
\toprule
\textbf{Ausprägung} & \textbf{Beschreibung} \\
\midrule

Niedrig &
Einsatz vorhandener \gls{ki}-Funktionen oder Standardtools ohne eigenes Modelltraining.
\\

Mittel &
Anpassung bestehender Modelle, Feature Engineering, begrenztes Retraining. 
\\

Hoch &
Eigenes Modelltraining, kontinuierlicher Betrieb und Überwachung notwendig.
\\


\bottomrule
\end{tabularx}
\end{table}

\par\bigskip

Die zweite Dimension bildet der operative Mehrwert, der auf der Y-Achse dargestellt wird und in Tabelle \ref{tab:yAchse} näher beschrieben ist.
Hier wird der tatsächliche Beitrag eines KI-Use-Cases zur Verbesserung des Plattformbetriebs erfasst.
Ein niedriger operativer Mehrwert umfasst vor allem unterstützende Funktionen mit begrenztem Einfluss auf zentrale Betriebskennzahlen. 
Ein hoher Mehrwert hingegen zeigt sich in deutlichen Verbesserungen wesentlicher KPIs wie MTTR, Kostenreduktion oder Systemstabilität.
\begin{table}[H]
\caption{Y-Achse: Operativer Mehrwert}
\label{tab:yAchse}
\footnotesize
\renewcommand{\arraystretch}{1.2}
\setlength{\tabcolsep}{7pt}

\begin{tabularx}{\linewidth}{p{3.5cm} X}
\toprule
\textbf{Ausprägung} & \textbf{Beschreibung} \\
\midrule

Niedrig &
Geringer Einfluss auf den Plattformbetrieb, primär unterstützende Funktionen wie Log-Analyse oder einfache Anomalieerkennung.
\\

Mittel &
Messbare Effizienzsteigerung oder Zeitersparnis im Plattformbetrieb durch KI-gestützte Automatisierung oder Optimierung.
\\

Hoch &
Deutliche Verbesserung zentraler KPIs (z.B. MTTR, Kostenreduktion, Stabilität)
\\


\bottomrule
\end{tabularx}
\end{table}

\par\bigskip

Abbildung \ref{fig:05_01_bewertungsMatrix} führt beide Dimensionen in einer zweidimensionalen Bewertungsmatrix zusammen. 
Hierfür wurde die Matrix in vier Quadranten unterteilt, wodurch eine qualitative Einordnung der KI-Use-Cases in Bezug auf das Verhältnis von Implementierungsaufwand und operativem Mehrwert ermöglicht wird.
Anwendungsfälle mit hohem Mehrwert und niedrigem Aufwand stellen besonders attraktive Kandidaten für eine kurzfristige Umsetzung dar.
Hingegen sollten Anwendungsfälle mit hohem Aufwand und geringem Mehrwert kritisch hinterfragt werden.
\begin{figure}[H]
    \centering
    \includegraphics[width=\textwidth]{figures/05_bewertung/05_01_bewertungsMatrix.png}
    \caption{Bewertungsmatrix zur Einordnung von KI-Use-Cases im Cloud-Native Platform Engineering entlang von Implementierungsaufwand und operativem Mehrwert}
    \label{fig:05_01_bewertungsMatrix}
\end{figure}
\par\bigskip
Die zweidimensionale Bewertungsmatrix ermöglicht eine erste Priorisierung von KI-Use-Cases basierend auf dem Verhältnis von Implementierungsaufwand und operativem Mehrwert.
Sie zeigt, welche Anwendungsfälle ein günstiges Verhältnis zwischen Implementierungsaufwand und operativem Mehrwert aufweisen.

Für eine fundiertere Bewertung sollten jedoch weitere Faktoren berücksichtigt werden.
Die zusätzlichen Dimensionen erweitern die Matrix nicht um weitere Achsen.
Stattdessen dienen sie der vertiefenden Analyse einzelner Anwendungsfälle innerhalb der bestehenden Struktur.
Sie adressieren insbesondere Aspekte der Umsetzbarkeit, der operativen Wirkung sowie regulatorische und organisatorische Rahmenbedingungen.
Diese Faktoren lassen sich in einer rein zweidimensionalen Darstellung nur unzureichend abbilden, sind jedoch für die praktische Bewertung entscheidend.
\par\bigskip

\textbf{Dimension 1: Umsetzbarkeit}
Die erste zusätzliche Dimension beschreibt die technische und organisatorische Umsetzbarkeit eines KI-Use-Cases.
Sie umfasst die Verfügbarkeit und Qualität geeigneter Datenquellen, den technischen Fit mit der bestehenden Plattformarchitektur sowie das erforderliche Know-how für Entwicklung, Betrieb und Wartung.
Die Einordnung erfolgt qualitativ (niedrig/mittel/hoch) auf Basis der in Abschnitt 5.2 betrachteten Datenquellen, Architekturartefakte sowie Rollenkompetenzen.
\par\bigskip

\textbf{Dimension 2: Operative Wirkung}
Die zweite Dimension fokussiert auf die erwartete Wirkung eines KI-Use-Cases im operativen Betrieb.
Hierbei werden Aspekte wie der direkte Einfluss auf den Plattformbetrieb und die Skalierbarkeit der Lösung über Services und Teams hinweg betrachtet.
Die Bewertung erfolgt anhand des erwarteten Beitrags zu zentralen Betriebskennzahlen sowie der Übertragbarkeit auf weitere Services und Teams.
\par\bigskip

\textbf{Dimension 3: Governance \& Reifegrad}
Die dritte Dimension berücksichtigt regulatorische und organisatorische Rahmenbedingungen.
Hier wird insbesondere auf Anforderungen an Compliance, Security und den Reifegrad der jeweiligen KI-Lösung eingegangen.
Die Einordnung erfolgt anhand regulatorischer Vorgaben, Sicherheitsanforderungen sowie des Reifegrads (PoC vs. produktiv erprobt) der jeweiligen Lösung.

\par\bigskip
In Abschnitt 5.2 werden die identifizierten KI-Use-Cases zunächst in die Bewertungsmatrix eingeordnet. 
Anschließend werden sie entlang der drei Zusatzdimensionen qualitativ bewertet, um Umsetzbarkeit, Wirkung und Governance-Aspekte kontextbezogen einzuordnen. 
Dadurch entsteht eine priorisierte und zugleich praxisnahe Grundlage für die Ableitung von Pain Points und Handlungsempfehlungen.