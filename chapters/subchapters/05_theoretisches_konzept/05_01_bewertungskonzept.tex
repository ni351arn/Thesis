\section{Bewertungskonzept}
\label{sec:bewertungskonzept}
Aufbauend auf den zuvor identifizierten KI-Use-Cases wird im Folgenden ein Bewertungskonzept vorgestellt, das eine strukturierte Einordnung und Vergleichbarkeit dieser Anwendungsfälle ermöglicht.
Ziel des Frameworks ist es, KI-Anwendungsfälle im Cloud-Native Platform Engineering systematisch hinsichtlich ihres Implementierungsaufwands und ihres operativen Mehrwerts zu bewerten.
Damit soll eine fundierte Entscheidungsgrundlage für die Priorisierung und Implementierung von KI-Lösungen geschaffen werden.

Implementierungsaufwand beschreibt dabei den Aufwand, eine KI-Funktion robust, sicher und wartbar produktiv zu integrieren und zu betreiben.
Operativer Mehrwert beschreibt den erwarteten Beitrag zur messbaren Verbesserung zentraler Betriebsziele, insbesondere der Betriebszuverlässigkeit (z.\,B. MTTR), der Effizienz sowie der Systemstabilität.
Als weitere betriebliche Zielgrößen können Service Level Objectives (SLOs) herangezogen werden, die messbare Zielwerte für Servicequalität (z.\,B. Verfügbarkeit oder Latenz) definieren.
Aspekte wie Datenzugang, Governance und organisatorische Voraussetzungen werden bewusst nicht in der X/Y-Achse abgebildet, sondern in zusätzlichen Dimensionen vertieft betrachtet.

Die erste Bewertungsdimension ist der Implementierungsaufwand, der auf der X-Achse der Bewertungsmatrix abgebildet wird.
Tabelle \ref{tab:xAchse} konkretisiert diese Dimension anhand der qualitativen Ausprägung von niedrig bis hoch.
Berücksichtigt werden dabei insbesondere der Grad der erforderlichen Modellanpassungen, der Trainings- und Anpassungsaufwand sowie die Komplexität des operativen Betriebs.
Darüber hinaus fließen Integrations- und Betriebsaspekte in die Einordnung ein, z.\,B. Anforderungen an Datenpipelines, Security-Mechanismen, Observability, Evaluation und laufende Qualitätssicherung.
Ein niedriger Implementierungsaufwand beschreibt demnach den Einsatz bestehender KI-Funktionen oder Standardwerkzeuge mit geringem Integrations- und Betriebsaufwand.
Ein hoher Aufwand hingegen ist typischerweise mit umfangreicher Integration und eigener Anpassung (z.\,B. RAG mit Datenpipeline und Evaluation oder Modelltraining) sowie kontinuierlicher Überwachung, Wartung und Qualitätssicherung verbunden.
\begin{table}[H]
\caption{X-Achse: Implementierungsaufwand }
\label{tab:xAchse}
\footnotesize
\renewcommand{\arraystretch}{1.2}
\setlength{\tabcolsep}{7pt}

\begin{tabularx}{\linewidth}{p{3.5cm} X}
\toprule
\textbf{Ausprägung} & \textbf{Beschreibung} \\
\midrule

Niedrig &
Einsatz vorhandener \gls{ki}-Funktionen oder Standardtools ohne eigenes Modelltraining.
\\

Mittel &
Anpassung bestehender Modelle, Feature Engineering, begrenztes Retraining. 
\\

Hoch &
Eigenes Modelltraining, kontinuierlicher Betrieb und Überwachung notwendig.
\\


\bottomrule
\end{tabularx}
\end{table}

\par\bigskip

Die zweite Dimension bildet der operative Mehrwert, der auf der Y-Achse dargestellt wird und in Tabelle \ref{tab:yAchse} näher beschrieben ist.
Hier wird der tatsächliche Beitrag eines KI-Use-Cases zur Verbesserung des Plattformbetriebs erfasst.
Ein niedriger operativer Mehrwert umfasst vor allem unterstützende Funktionen mit begrenztem Einfluss auf zentrale Betriebskennzahlen.
Ein hoher Mehrwert hingegen zeigt sich in deutlichen Verbesserungen wesentlicher KPIs und Betriebsziele, z.\,B. MTTR, SLO-Erfüllung, Error-Budget-Verbrauch, Infrastrukturkosten oder Systemstabilität.
\begin{table}[H]
\caption{Y-Achse: Operativer Mehrwert}
\label{tab:yAchse}
\footnotesize
\renewcommand{\arraystretch}{1.2}
\setlength{\tabcolsep}{7pt}

\begin{tabularx}{\linewidth}{p{3.5cm} X}
\toprule
\textbf{Ausprägung} & \textbf{Beschreibung} \\
\midrule

Niedrig &
Geringer Einfluss auf den Plattformbetrieb, primär unterstützende Funktionen wie Log-Analyse oder einfache Anomalieerkennung.
\\

Mittel &
Messbare Effizienzsteigerung oder Zeitersparnis im Plattformbetrieb durch KI-gestützte Automatisierung oder Optimierung.
\\

Hoch &
Deutliche Verbesserung zentraler KPIs (z.B. MTTR, Kostenreduktion, Stabilität)
\\


\bottomrule
\end{tabularx}
\end{table}

\par\bigskip

Abbildung \ref{fig:05_01_bewertungsMatrix} führt beide Dimensionen in einer zweidimensionalen Bewertungsmatrix zusammen.
Hierfür wurde die Matrix in vier Quadranten unterteilt, wodurch eine qualitative Einordnung der KI-Use-Cases in Bezug auf das Verhältnis von Implementierungsaufwand und operativem Mehrwert ermöglicht wird.
Anwendungsfälle mit hohem Mehrwert und niedrigem Aufwand weisen ein günstiges Aufwand-Nutzen-Verhältnis auf und eignen sich als priorisierte Kandidaten für eine erste Umsetzung.
Use Cases mit hohem Mehrwert und hohem Aufwand sind strategische Kandidaten, die typischerweise eine Roadmap, Vorarbeiten (z.\,B. Daten- und Integrationsgrundlagen) sowie ein risikobewusstes Vorgehen erfordern.
Hingegen sollten Anwendungsfälle mit hohem Aufwand und geringem Mehrwert nur weiterverfolgt werden, wenn sie notwendige Vorstufen für höherwertige Use Cases darstellen oder klare Abhängigkeiten bestehen.
\begin{figure}[H]
    \centering
    \includegraphics[width=0.6\textwidth]{figures/05_bewertung/05_01_bewertungsMatrix.png}
    \caption{Bewertungsmatrix zur Einordnung von KI-Use-Cases im Cloud-Native Platform Engineering entlang von Implementierungsaufwand und operativem Mehrwert}
    \label{fig:05_01_bewertungsMatrix}
\end{figure}
\par\bigskip

Die zweidimensionale Bewertungsmatrix priorisiert KI-Use-Cases zunächst über das Verhältnis von Implementierungsaufwand (X) und operativem Mehrwert (Y). 
Sie liefert damit eine erste Einordnung, greift aber zentrale Rahmenbedingungen des Plattformbetriebs nur indirekt auf.

Daher werden ergänzend vier Zusatzdimensionen betrachtet. 
Diese erweitern die Matrix nicht um weitere Achsen, sondern beschreiben Faktoren, die über Erfolg, Risiko und realistische Umsetzbarkeit eines Use Cases entscheiden. 
Die Bewertung erfolgt je Unterdimension einheitlich auf einer dreistufigen Skala (niedrig/mittel/hoch).
\par\bigskip
\textbf{Dimension 1: Umsetzbarkeit}

Die Dimension Umsetzbarkeit bewertet, wie realistisch ein KI-Use-Case technisch und organisatorisch umgesetzt werden kann. 
Sie umfasst (i) Datenverfügbarkeit und -qualität, (ii) die Passung zur bestehenden Plattform- bzw. CI/CD-Architektur (inkl. Integration) sowie (iii) die verfügbaren Fachkenntnisse für Entwicklung, Betrieb und Wartung.

Niedrig bedeutet, dass Daten ausreichend vorhanden sind, die Integration ohne größere Änderungen möglich ist und die nötige Expertise im Team verfügbar ist.

Mittel bedeutet, dass einzelne Vorarbeiten nötig sind (z.B. Datenaufbereitung, kleinere Anpassungen in Schnittstellen/Prozessen oder gezielter Kompetenzaufbau).

Hoch bedeutet, dass wesentliche Voraussetzungen fehlen (z.B. Datenlücken, fehlende Integrationsmöglichkeiten oder deutlicher Kompetenzaufbau), sodass der Use Case nur mit spürbarem Vorlauf realisierbar ist.
\par\bigskip

\textbf{Dimension 2: Betriebswirksamkeit \& Skalierbarkeit}

Diese Dimension bewertet, ob ein KI-Use-Case im laufenden Plattformbetrieb zuverlässig einen operativen Nutzen entfaltet.
Zusätzlich wird betrachtet, ob sich der Ansatz mit vertretbarem Aufwand auf weitere Services oder Teams übertragen lässt.
Berücksichtigt werden (i) die Stabilität und Wirksamkeit im Betrieb sowie (ii) die Skalierbarkeit und Wiederverwendbarkeit der Lösung.

Niedrig bedeutet, dass der Nutzen stark kontextabhängig, instabil oder auf einzelne Services begrenzt ist.

Mittel bedeutet, dass ein erkennbarer Mehrwert im Betrieb entsteht, die Übertragbarkeit jedoch nur eingeschränkt gegeben ist oder zusätzlichen Anpassungsaufwand erfordert.

Hoch bedeutet, dass die Lösung stabil im Betrieb wirkt und mit geringem Zusatzaufwand breit ausgerollt werden kann.
\par\bigskip

\textbf{Dimension 3: Compliance \& Governance}

Diese Dimension erfasst organisatorische und regulatorische Rahmenbedingungen für den Einsatz eines KI-Use-Cases.
Bewertet werden (i) Anforderungen an Datenschutz, Datenklassifikation und Zugriffskontrolle sowie (ii) die organisatorische Verankerung durch klare Zuständigkeiten und Freigabeprozesse.
Die Dimension dient dazu, Risiken und Abstimmungsbedarfe frühzeitig sichtbar zu machen.

Niedrig bedeutet, dass geringe regulatorische Anforderungen bestehen und Verantwortlichkeiten klar geregelt sind.

Mittel bedeutet, dass zusätzliche Abstimmungen oder formale Freigaben erforderlich sind, jedoch gut handhabbar.

Hoch bedeutet, dass strenge Vorgaben, sensible Daten oder unklare Zuständigkeiten zu erhöhtem Abstimmungs- und Umsetzungsaufwand führen.
\par\bigskip

\textbf{Dimension 4: Reifegrad}

Die Dimension Reifegrad bewertet den Entwicklungs- und Betriebsstand eines KI-Use-Cases.
Sie umfasst (i) die Technologiereife der Lösung sowie (ii) die Absicherung des dauerhaften Betriebs.
Berücksichtigt werden unter anderem der Entwicklungsstatus, vorhandene Betriebserfahrungen und die organisatorische Absicherung.

Niedrig bedeutet, dass die Lösung sich im Produktivbetrieb befindet und organisatorisch sowie technisch abgesichert ist.

Mittel bedeutet, dass der Use Case pilotiert oder eingeschränkt produktiv eingesetzt wurde, mit noch offenen Betriebsfragen.

Hoch bedeutet, dass die Lösung als Konzept oder Prototyp vorliegt und belastbare Erfahrungen im Regelbetrieb fehlen.
\par\bigskip

Zur einheitlichen Anwendung der Zusatzdimensionen werden die qualitativen Bewertungen in Tabelle \ref{tab:zusatzdimensionen} zusammengefasst.
\begin{table}[H]
\caption{Bewertungsskalen der Zusatzdimensionen}
\label{tab:zusatzdimensionen}
\footnotesize
\renewcommand{\arraystretch}{1.2}
\setlength{\tabcolsep}{7pt}

\begin{tabularx}{\linewidth}{p{4.2cm} p{2.2cm} X}
\toprule
\textbf{Zusatzdimension} & \textbf{Ausprägung} & \textbf{Beschreibung} \\
\midrule

\makecell[l]{Umsetzbarkeit} & Niedrig &
Daten, Integrationspfade und erforderliches Know-how sind vorhanden; Umsetzung ohne strukturelle Änderungen möglich. \\
& Mittel &
Vorarbeiten bei Datenverfügbarkeit, Integration oder Kompetenzen notwendig. \\
& Hoch &
Zentrale Daten, Schnittstellen oder Fähigkeiten fehlen aktuell und erfordern zusätzliche Vorarbeiten. \\
\midrule

\makecell[l]{Betriebswirksamkeit\\\& Skalierbarkeit} & Niedrig &
Wirkung lokal oder instabil; stark kontextabhängig und nur schwer auf weitere Services oder Teams übertragbar. \\
& Mittel &
Stabile Wirkung in mehreren Szenarien, jedoch mit begrenzter Skalierbarkeit oder erhöhtem operativem Aufwand. \\
& Hoch &
Zuverlässig wirksam im Plattformbetrieb und plattformweit skalierbar ohne individuellen Zusatzaufwand. \\
\midrule

\makecell[l]{Governance\\\& Reifegrad} & Niedrig &
Geringe Governance- und Compliance-Hürden bei produktiv erprobtem Reifegrad. \\
& Mittel &
Zusätzliche Anforderungen an Dokumentation, Compliance oder organisatorische Abstimmung erforderlich. \\
& Hoch &
Hohe regulatorische Anforderungen oder geringer Reifegrad (z.\,B.\ Proof of Concept oder Forschung). \\
\bottomrule
\end{tabularx}
\end{table}

\par\bigskip

In Abschnitt 5.2 werden die identifizierten KI-Use-Cases zunächst in die Bewertungsmatrix eingeordnet (X/Y).
Anschließend werden sie entlang der drei Zusatzdimensionen qualitativ bewertet, um Umsetzbarkeit, Betriebswirksamkeit und Governance-Aspekte kontextbezogen zu prüfen.
Dadurch entsteht eine priorisierte und zugleich praxisnahe Grundlage, um identifizierte Pain Points der Plattform gezielt mit geeigneten KI-Use-Cases zu adressieren und daraus Handlungsempfehlungen abzuleiten.
