\section{Vorgehen der Mapping Study}
\label{sec:mapping-study}

Für diese Arbeit wird eine Systematic Mapping Study (SMS) nach den Richtlinien von Petersen, Vakkalanka und Kuzniarz \cite{petersenGuidelinesConductingSystematic2015} durchgeführt. Diese Methodik dient dazu, den aktuellen Forschungsstand zu einem Themengebiet systematisch zu erfassen, zu kategorisieren und bestehende Forschungslücken zu identifizieren.

Das Vorgehen umfasst die Phasen Planung, Durchführung und Auswertung. 
In der Planungsphase werden die Forschungsfragen definiert und die Suchstrategie entiwckelt, einschließlich der Auswahl relevanter wissenschaftlicher Datenbanken. Dabei wird gezielt nach bestimmten Keywords gesucht, um Publikationen zu finden, die KI-Anwendungen im Kontext von Platform Engineering adressieren.\par

Um eine fundierte Datenerhebung und Auswertung sicherzustellen, werden einzelne Prinzipien einer Systematic Literature Review (SLR) nach Kitchenham und Charters \cite{kitchenhamPDFGuidelinesPerforming} berücksichtigt.
In der Durchführungsphase werden identifizierte Studien anhand festgelegter Ein- und Ausschlusskriterien geprüft. Zusätzlich wird das Schneeballverfahren nach Wohlin \cite{wohlinGuidelinesSnowballingSystematic2014} eingesetzt, um die Literatursammlung zu erweitern. Alle Schritte werden dokumentiert, um die Nachvollziehbarkeit und Reproduzierbarkeit sicherzustellen. 
Die Auswertung erfolgt durch eine systematische Kategorisierung der Studien entlang zentraler Themenfelder des Platform Engineerings. Darauf aufbauend werden Muster, Trends und Forschungslücken identifiziert.   

Die Ergebnisse der Mapping Study bilden die Grundlage für die anschließende Analyse der Bosch Digital Manufacturing Platform sowie für die Entwicklung eines Frameworks zur Bewertung von KI-Potenzialen in Cloud-Native Plattformumgebungen. 
