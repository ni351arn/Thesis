\section{Forschungsfragen}
\label{sec:forschungsfragen}

Die Forschungsfragen werden durch ein strukturiertes methodisches Vorgehen beantwortet. Jede Forschungsphase ist darauf ausgelegt, das Verständnis über den Einsatz von Künstlicher Intelligenz im Platform Engineering schrittweise zu vertiefen.
Die Mapping Study dient der Beantwortung der ersten beiden Forschungsfragen, indem sie eine systematische Übersicht über bestehende KI-Ansätze und deren Anwendungsfelder liefert. 
Auf Grundlage der Ergebnisse der Mapping Study zielt die dritte Forschungsfrage darauf ab, ein praxisorientiertes Framework zur Beantwortung und Übertragbarkeit von KI-Lösungen zu entwickeln. 
Die Arbeit ist entlang der folgenden Forschungsfragen strukturiert: 

\begin{description}[labelwidth=3em,labelsep=1em,leftmargin=!,font=\bfseries]
  \item[RQ1:] Welche typische Anwendungsfelder und Herausforderungen bestehen im Platform Engineering, in denen KI-Technologien potenziell Mehrwert bieten können?
  \\ Ziel: Systematische Erfassung und Kategorisierung relevanter Use Cases.

  \item[RQ2:] Welche KI-Technologien (einschließlich Frameworks und Tools) finden derzeit im Platform Engineering Anwendung, und welche Formen des maschinellen Lernens und Algorithmen kommen dabei zum Einsatz?
  \\ Ziel: Erstellung einer Übersicht über vorhandene KI-Ansätze und deren typische Einsatzkontexte.

  \item[RQ3:] Wie lassen sich die identifizierten KI-Lösungen auf typische Anwendungsfälle in Cloud-Native-Platform-Umgebungen übertragen und hinsichtlich ihres Mehrwertes und ihrer Umsetzbarkeit bewerten?
  \\ Ziel: Entwicklung eines Bewertungsschemas, das die Passung zwischen KI-Lösungen und spezifischen Platform-Use-Cases beschreibt und die praktische Umsetzbarkeit aufzeigt.
\end{description}