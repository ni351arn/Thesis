\section{Literaturanalyse-Prozess}
\label{sec:literatur-prozess}
Der folgende Abschnitt beschreibt den Ablauf der Literaturanalyse im Rahmen der durchgeführten Mapping Study. Ziel ist es, den methodischen Prozess transparent darzustellen. Dazu werden zunächst die Suchstrategie und die Auswahlkriterien erläutert, gefolgt von der Anwendung der Schneeballmethode. Danach wird die Datenextraktion sowie die Datensynthese beschrieben.

\subsection{Suchstrategie}
\label{subsec:suchstrategie}

Zur Durchführung der Mapping Study wurde eine systematische Suchstrategie angewendet, um relevante wissenschaftliche Publikationen zu identifizieren. Die Literaturrecherche erfolgte in den Datenbanken Google Scholar, SpringerLink, ScienceDirect und IEEE Xplore, da diese eine breite Abdeckung im Bereich Software Engineering, Cloud Native Technologien und KI bieten. 
Ziel der Suche war, eine möglichst vollständige Übersicht aktueller Forschungsarbeiten zu KI-Anwendungen im Platform Engineering zu erhalten. 
Dafür wurden gezielt Suchbegriffe und Kombinationen von Suchstrings verwendet, die zentrale Themen der Arbeit abbilden. 
Die Suchbegriffe waren hierbei: "Platform Engineering“, "Cloud-Native“, "AIOps“, "Artificial Intelligence“, "Machine Learning“, "MLOps“, "DevOps“ und "Kubernetes Cluster“. 

Zur Transparenz und Vollständigkeit sind die exakten Suchstrings sowie ihre logischen Verknüpfungen im Anhang dokumentiert. 

\subsection{Auswahlkriterien}
\label{subsec:auswahlkriterien}

Um relevante Studien und Publikationen zu identifizieren, wurde ein systematischer Auswahlprozess durchgeführt, der auf klar definierten Ein- und Ausschlusskriterien basiert. Eine Studie wurde in die Analyse aufgenommen, wenn sie alle Einschlusskriterien erfüllt und zugleich keinem der Ausschlusskriterien unterlag. Die vollständige Übersicht der Kriterien ist in Tabelle \ref{tab:auswahlkriterien} dargestellt.

\begin{table}[htbp]
  \centering
  \caption{Auswahlkriterien}
  \label{tab:auswahlkriterien}
  \footnotesize
  \renewcommand{\arraystretch}{1.15}
  \begin{tabularx}{\linewidth}{@{}lY@{}}
    \toprule
    \textbf{Kriterium} & \textbf{Beschreibung} \\
    \midrule
    EK1 & Die Publikation befasst sich mit dem Einsatz von KI oder Machine Learning im Kontext von Platform Engineering, Cloud-Native-Technologien, DevOps oder AIOps. \\
    EK2 & Die Studie beschreibt konkrete KI-Methoden, Anwendungen, Architekturen oder Use Cases, die sich auf Plattformumgebungen beziehen. \\
    EK3 & Nur peer-reviewte Veröffentlichungen werden berücksichtigt. \\
    \midrule
    AK1 & Arbeiten, die keinen klaren Bezug zu KI im Platform Engineering oder verwandten Domänen aufweisen. \\
    AK2 & Studien, die keine empirische Analyse, Evaluation oder methodische Beschreibung ihrer Ansätze enthalten, werden ausgeschlossen. \\
    AK3 & Bücher, kommerzielle Reports, technische Dokumentationen und graue Quellen werden ausgeschlossen. \\
    AK4 & Studien, die vor 2020 veröffentlicht wurden. \\
    \bottomrule
  \end{tabularx}
\end{table}

\subsection{Schneeballmethode}
\label{subsec:schneeballmethode}

Zur Ergänzung der systematischen Suche wurde eine Schneeballmethode nach den Leitlinien von Wohlin \cite{wohlinGuidelinesSnowballingSystematic2024} angewendet. 
Dabei erfolgte sowohl eine Rückwärtssuche als auch eine Vorwärtssuche. Als Ausgangspunkt dienten vier relevante Paper, auf deren Basis zwei Iterationen der Vorwärts- und Rückwärtssuche durchgeführt wurden. 
Die neu gefundenen Publikationen wurden nach denselben Ein- und Ausschlusskriterien geprüft. Der Prozess wurde beendet, sobald keine weiteren relevanten Studien identifiziert werden konnten.


\subsection{Datenerhebung aus den Studien}
\label{subsec:datenerhebung}

Zur Sicherstellung von Konsistenz und Nachvollziehbarkeit wurde ein strukturierter Prozess zur Datenerhebung aus den Studien umgesetzt. 
Hierzu wurde eine eigene Extraktionsvorlage entwickelt, die die wesentlichen Merkmale der identifizierten Studien erfasst. 
Diese Merkmale wurden anschließend entlang von fünf zentralen Dimensionen kategorisiert, wie in Tabelle \ref{tab:datenextraktion} dargestellt.
\begin{table}[htbp]
  \centering
  \caption{Kategorisierung der Datenerhebung}
  \label{tab:datenextraktion}
  \footnotesize
  \renewcommand{\arraystretch}{1.15}
  \begin{tabularx}{\linewidth}{@{}lY@{}}
    \toprule
    \textbf{Dimension} & \textbf{Beschreibung} \\
    \midrule
    Forschungskontext & Beschreibt Ziel, Umfang und Art der Studie. \\
    KI-Ansatz und Methode & Erfasst die verwendeten KI- oder ML-Verfahren. \\
    Platform-Domänen & Ordnet den Beitrag einem Bereich des Platform Engineerings zu. \\
    Ergebnisse und Use-Cases & Fasst die zentralen Erkenntnisse, Anwendungsfälle oder Evaluationsergebnisse zusammen. \\
    Forschungslücke & Dokumentiert identifizierte Limitationen und Ansätze für zukünftige Arbeiten. \\
    \bottomrule
  \end{tabularx}
\end{table}
\FloatBarrier 

Die Extraktion erfolgte qualitativ, wobei relevante Textpassagen und zentrale Aussagen aus jeder Publikation manuell erfasst und den entsprechenden Dimensionen zugeordnet wurden

\subsection{Datenanalyse und Kategorisierung}
\label{subsec:datenanalyse}

Nach der Datenerhebung wurden die Ergebnisse zusammengeführt und ausgewertet, um zentrale Themen, Muster und Forschungslücken zu erkennen. Die ausgewählten Studien wurden nach ihren Inhalten und Schwerpunkten strukturiert und den Forschungsfragen zugeordnet. 
Zur besseren Übersicht erfolgte die Kategorisierung der Arbeit entlang wichtiger Bereiche des Platform Engineerings. 
Innerhalb dieser Kategorien wurden die identifizierten KI-Ansätze, Anwendungsfälle und Herausforderungen miteinander verglichen, um wiederkehrende Trends sichtbar zu machen.
Die Ergebnisse der Datenanalyse und Kategorisierung werden anschließend in Form einer Mapping Study dargestellt. 
Diese Übersicht zeigt, in welchen Themenfelder bereits Forschungsschwerpunkte existieren und wo noch Forschungslücken bestehen. 