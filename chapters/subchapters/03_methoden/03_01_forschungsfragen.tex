\section{Forschungsfragen}
\label{sec:forschungsfragen}

Die Forschungsfragen werden durch ein strukturiertes methodisches Vorgehen beantwortet. Jede Forschungsphase ist darauf ausgelegt, das Verständnis über den Einsatz von \gls{ki} im Platform Engineering schrittweise zu vertiefen.
Die Mapping-Studie dient der Beantwortung der ersten beiden Forschungsfragen, indem sie eine systematische Übersicht über bestehende \gls{ki}-Ansätze und deren Anwendungsfelder liefert. 
Auf Grundlage der Ergebnisse der Mapping-Studie zielt die dritte Forschungsfrage darauf ab, ein praxisorientiertes Framework zur Beantwortung und Übertragbarkeit von \gls{ki}-Lösungen zu entwickeln. 
Die Arbeit ist entlang der folgenden Forschungsfragen strukturiert: 
\vspace{1em}

\begin{description}[labelwidth=3em,labelsep=1em,leftmargin=!,font=\bfseries,itemsep=1em]
  \item[RQ1:] Welche typischen Anwendungsfelder (Use Cases) und Herausforderungen bestehen im Platform Engineering, in denen \gls{ki}-Technologien potenziell Mehrwert bieten können?
  \\ Ziel: Systematische Erfassung und Kategorisierung relevanter Use Cases.

  \item[RQ2:] Welche \gls{ki}-Technologien (einschließlich Frameworks und Tools) finden derzeit im Platform Engineering Anwendung, und welche Formen des maschinellen Lernens und Algorithmen kommen dabei zum Einsatz?
  \\ Ziel: Erstellung einer Übersicht über vorhandene \gls{ki}-Ansätze und deren typische Einsatzkontexte.

  \item[RQ3:] Wie lassen sich die identifizierten \gls{ki}-Lösungen auf typische Anwendungsfälle in Cloud-Native-Plattform-Umgebungen übertragen und hinsichtlich ihres Mehrwertes und ihrer Umsetzbarkeit bewerten?
  \\ Ziel: Entwicklung eines Bewertungsschemas, das die Passung zwischen \gls{ki}-Lösungen und spezifischen Plattform-Use-Cases beschreibt und die praktische Umsetzbarkeit aufzeigt.
\end{description}

