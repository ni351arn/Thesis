\section{Vorgehen der Mapping-Studie}
\label{sec:mapping-study}

Für diese Arbeit wird eine Mapping-Studie nach den Richtlinien von \textcite{petersenGuidelinesConductingSystematic2015} durchgeführt. 
Diese Methodik dient dazu, den aktuellen Forschungsstand zu einem Themengebiet systematisch zu erfassen, zu kategorisieren und bestehende Forschungslücken zu identifizieren.

Das Vorgehen umfasst die Phasen Planung, Durchführung und Auswertung. 
In der Planungsphase werden die Forschungsfragen definiert und die Suchstrategie entwickelt, einschließlich der Auswahl relevanter wissenschaftlicher Datenbanken. 
Dabei wird gezielt nach bestimmten Keywords gesucht, um Publikationen zu finden, die \gls{ki}-Anwendungen im Kontext von Platform Engineering adressieren.

Um eine fundierte Datenerhebung und Auswertung sicherzustellen, werden einzelne Prinzipien einer Systematic Literature Review (SLR) nach \textcite{kitchenhamPDFGuidelinesPerforming2007} berücksichtigt.
In der Durchführungsphase werden identifizierte Studien anhand festgelegter Ein- und Ausschlusskriterien geprüft. Zusätzlich wird das Schneeballverfahren nach \textcite{wohlinGuidelinesSnowballingSystematic2024} eingesetzt, um die Literatursammlung zu erweitern. Alle Schritte werden dokumentiert, um die Nachvollziehbarkeit und Reproduzierbarkeit sicherzustellen. 
Die Auswertung erfolgt durch eine systematische Kategorisierung der Studien entlang zentraler Themenfelder des Platform Engineerings. Darauf aufbauend werden Muster, Trends und Forschungslücken identifiziert.

Die Analyse umfasst dabei Mappings zwischen Anwendungsfeldern und Herausforderungen, zwischen Lernparadigmen und verwendeten Algorithmen sowie zwischen Anwendungsfeldern und genutzten Datenquellen.

Die Ergebnisse der Mapping-Studie bilden die Grundlage für die anschließende Analyse der \gls{bmlp} sowie für die Entwicklung eines Frameworks zur Bewertung von \gls{ki}-Potenzialen in Cloud-Native Plattformumgebungen. 
