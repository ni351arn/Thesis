\chapter{Methodisches Vorgehen}
\label{ch:methoden}

Dieses Kapitel beschreibt das methodische Vorgehen dieser Arbeit. 
Es erläutert die zugrunde liegende Forschungslogik, die Auswahl des methodischen Ansatzes sowie die Verfahren zur Datenerhebung und -auswertung.
Ziel ist es, ein wissenschaftlich fundiertes und zugleich praxisorientiertes Vorgehen aufzuzeigen, das eine systematische Untersuchung der Forschungsfragen ermöglicht. 
In \autoref{sec:forschungsfragen} werden die spezifischen Ziele und Forschungsfragen dieser Arbeit definiert.
\autoref{sec:literatur-prozess} beschreibt den Prozess der Literaturanalyse, einschließlich Suchstrategie, Ein- und Ausschlusskriterien, Schneeballmethode sowie Datenerhebung und -analyse.
In \autoref{sec:mapping-study} wird das Vorgehen der Mapping-Studie beschrieben und die Auswertung der identifizierten Studien erläutert.

% Abschnittsdateien einbinden
\section{Forschungsfragen}
\label{sec:forschungsfragen}

Die Forschungsfragen werden durch ein strukturiertes methodisches Vorgehen beantwortet. Jede Forschungsphase ist darauf ausgelegt, das Verständnis über den Einsatz von \gls{ki} im Platform Engineering schrittweise zu vertiefen.
Die Mapping-Studie dient der Beantwortung der ersten beiden Forschungsfragen, indem sie eine systematische Übersicht über bestehende \gls{ki}-Ansätze und deren Anwendungsfelder liefert. 
Auf Grundlage der Ergebnisse der Mapping-Studie zielt die dritte Forschungsfrage darauf ab, ein praxisorientiertes Framework zur Beantwortung und Übertragbarkeit von \gls{ki}-Lösungen zu entwickeln. 
Die Arbeit ist entlang der folgenden Forschungsfragen strukturiert: 
\vspace{1em}

\begin{description}[labelwidth=3em,labelsep=1em,leftmargin=!,font=\bfseries,itemsep=1em]
  \item[RQ1:] Welche typischen Anwendungsfelder (Use Cases) und Herausforderungen bestehen im Platform Engineering, in denen \gls{ki}-Technologien potenziell Mehrwert bieten können?
  \\ Ziel: Systematische Erfassung und Kategorisierung relevanter Use Cases.

  \item[RQ2:] Welche \gls{ki}-Technologien (einschließlich Frameworks und Tools) finden derzeit im Platform Engineering Anwendung, und welche Formen des maschinellen Lernens und Algorithmen kommen dabei zum Einsatz?
  \\ Ziel: Erstellung einer Übersicht über vorhandene \gls{ki}-Ansätze und deren typische Einsatzkontexte.

  \item[RQ3:] Wie lassen sich die identifizierten \gls{ki}-Lösungen auf typische Anwendungsfälle in Cloud-Native-Plattform-Umgebungen übertragen und hinsichtlich ihres Mehrwertes und ihrer Umsetzbarkeit bewerten?
  \\ Ziel: Entwicklung eines Bewertungsschemas, das die Passung zwischen \gls{ki}-Lösungen und spezifischen Plattform-Use-Cases beschreibt und die praktische Umsetzbarkeit aufzeigt.
\end{description}


\input{chapters/subchapters/03_methoden/03_02_literatur_analyse_prozess}
\section{Vorgehen der Mapping-Studie}
\label{sec:mapping-study}

Für diese Arbeit wird eine Mapping-Studie nach den Richtlinien von \textcite{petersenGuidelinesConductingSystematic2015} durchgeführt. 
Diese Methodik dient dazu, den aktuellen Forschungsstand zu einem Themengebiet systematisch zu erfassen, zu kategorisieren und bestehende Forschungslücken zu identifizieren.

Das Vorgehen umfasst die Phasen Planung, Durchführung und Auswertung. 
In der Planungsphase werden die Forschungsfragen definiert und die Suchstrategie entwickelt, einschließlich der Auswahl relevanter wissenschaftlicher Datenbanken. 
Dabei wird gezielt nach bestimmten Keywords gesucht, um Publikationen zu finden, die \gls{ki}-Anwendungen im Kontext von Platform Engineering adressieren.

Um eine fundierte Datenerhebung und Auswertung sicherzustellen, werden einzelne Prinzipien einer Systematic Literature Review (SLR) nach \textcite{kitchenhamPDFGuidelinesPerforming2007} berücksichtigt.
In der Durchführungsphase werden identifizierte Studien anhand festgelegter Ein- und Ausschlusskriterien geprüft. Zusätzlich wird das Schneeballverfahren nach \textcite{wohlinGuidelinesSnowballingSystematic2024} eingesetzt, um die Literatursammlung zu erweitern. Alle Schritte werden dokumentiert, um die Nachvollziehbarkeit und Reproduzierbarkeit sicherzustellen. 
Die Auswertung erfolgt durch eine systematische Kategorisierung der Studien entlang zentraler Themenfelder des Platform Engineerings. Darauf aufbauend werden Muster, Trends und Forschungslücken identifiziert.

Die Analyse umfasst dabei Mappings zwischen Anwendungsfeldern und Herausforderungen, zwischen Lernparadigmen und verwendeten Algorithmen sowie zwischen Anwendungsfeldern und genutzten Datenquellen.

Die Ergebnisse der Mapping-Studie bilden die Grundlage für die anschließende Analyse der \gls{bmlp} sowie für die Entwicklung eines Frameworks zur Bewertung von \gls{ki}-Potenzialen in Cloud-Native Plattformumgebungen. 
