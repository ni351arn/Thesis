\section{Ergebnisse der Mapping Study}
\label{sec:mapping-study-ergebnisse}

Die Ergebnisse der Mapping Studie wurden in 
Hierzu wurden die Ergebnisse in Form von Bubble Chart Diagrammen visualisiert, in denen Blasengröße die Häufigkeit der entsprechenden Zuordnung widerspieglt.


\subsection{Zusammenspiel der Anwendungsfelder und Herausforderungen}
\label{subsec:mappingEins}
Die Abbildung \ref{fig:04_02_01_mapping} zeigt das Zusammenspiel zwischen den ausgewählten Use cases und zentralen Herausforderungen im Platform Engineering.
Im Gegensatz zur Abbildung \ref{fig:04_01_01_useCases}, wurden hier alle Anwendungsbereiche pro Paper einbezogen und den angesprochenen Herausforderungen zugeordnet.
Die in der Mapping Study extrahierten Häufigkeiten verdeutlichen, dass bestimmte Use cases besonders stark mit einer Vielzahl technischer und organisatorischer Herausforderungen zusammenhängen.

\begin{figure}[H]
    \centering
    \includegraphics[width=\textwidth]{figures/04_ergebnisse/04_02_mapping/04_02_01_mapping.png}
    \caption{Korrelation zwischen Anwendungsbereichen und Herausforderungen}
    \label{fig:04_02_01_mapping}
\end{figure}

Am stärksten ausgeprägt sind die Zusammenhänge in den Use Cases Ressource und Workload Optimierung sowie Self-Healing und Predictive Maintenance, die in allen fünf Herausforderungskategorien hohe Werte erreichen.
Diese Cluster zeigen, dass beide Anwendungsbereiche eng mit betrieblichen und technischen Anforderungen zusammenhängen und häufig als zentrale Ansatzpunkte für KI-basierte Optimierung diskutiert werden.

Der Anwendungsbereich Security und Threat Detection weist insbesondere in der Hausforderung Governance und Security eine hohe Ausprägung auf, während die übrigen Kategorien leicht geringere, aber dennoch hohe Werte zeigen.
Dies deuetet darauf hin, dass sicherbetsbezogene KI-Ansätze in der Liiteratur vor allem im Kontext von Compliance, Risikoerkennung und Schutzmechanismen betrachtet werden.
Allerdings spielt diese Herausforderung auch im Monitoring sowie im Ressourcenmanagemment eine Rolle.

Vergleichsweise etwas weniger thematisiert, aber dennoch deutlich erkennbar, sind die Werte im Bereich Intelligente Bereitstellung (Edge-AI und Serverless).
Diese Verteilung spiegelt wider dass diese Use Cases weniger stark im Mittelpunkt der untersuchten Literatur stehen als die zuvor genannten Anwendungsbereiche. 

Insgesamt zeigt die Verteilung der Häufigkeiten, dass KI-Anwendungen im Platform Engineeering vor allem dort adressiert werden, wo Effizienz, Stabilität und Sicherheitsaspekte zusammenspielen.
Die identifzierten Zahlen verdeutlichen bevorzugte Kombinationen aus Anwendungsbereiche und Herausforderungen und ermöglichen damit eine strukturierte Einordnung besteheneder Forschungsschwerpunkte. 


\subsection{Zusammenspiel der Lernparadimgen und Algorithmen}
\label{subsec:mappingzwei}

\begin{figure}[H]
    \centering
    \includegraphics[width=\textwidth]{figures/04_ergebnisse/04_02_mapping/04_02_02_mappingZwei.png}
    \caption{Korrelation zwischen Lernparadigmen und Algorithmen}
    \label{fig:04_02_02_mappingZwei}
\end{figure}



\subsection{Mapping von Anwendungsfelder und Lernparadigmen}
\label{subsec:mappingDrei}


\begin{figure}[H]
    \centering
    \includegraphics[width=\textwidth]{figures/04_ergebnisse/04_02_mapping/04_02_03_mappingDrei.png}
    \caption{Mapping von Anwendungsfeldern und Lernparadigmen}
    \label{fig:04_02_03_mappingDrei}
\end{figure}

\subsection{Mapping von Herausforderungen und Lernparadigmen}
\label{subsec:mappingVier}

\begin{figure}[H]
    \centering
    \includegraphics[width=\textwidth]{figures/04_ergebnisse/04_02_mapping/04_02_04_mappingVier.png}
    \caption{Mapping von Herausforderungen und Lernparadigmen}
    \label{fig:04_02_04_mappingVier}
\end{figure}