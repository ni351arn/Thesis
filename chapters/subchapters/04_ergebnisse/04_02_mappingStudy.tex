\section{Ergebnisse der Mapping Study}
\label{sec:mapping-study-ergebnisse}

Text für Mapping Study


\subsection{Zusammenspiel der Anwendungsfelder und Herausforderungen}
\label{subsec:mappingEins}
Die Analyse des Zusammenspiels zwischen typischen Use cases und zentralen Herausforderungen im Platform Engineering zeigt deutliche Muster hinsichtlich der Bereiche, in denen KI-Ansätze in der Letratur adressiert werden.
Die Ergebnisse sind in Form eines Bubble Charts visualisiert, in dem die Größe der Blasen die Häufigkeit der entsprechenden Zuordnung widerspieglt.
Im Gegensatz zur \ref{fig:04_01_01_useCases}

Über alle Arbeiten hinweg zeigt sich, dass insbesondere die Ressourcen-  und Workload-Optimierung sowie Slef-Healing und Preditive Maintenance in enegem Zusammenhang mit nahezu allen Heruasforderungen stehen.
Beide Anwendungsbereiche weisen durchgehend hohe Häufikgeitswerte auf, was darauf hinweist, dass sie in der Forschung als besonders relevant wahrgenommen werden. 


\begin{figure}[H]
    \centering
    \includegraphics[width=\textwidth]{figures/04_ergebnisse/04_02_01_mapping.png}
    \caption{Korrelation zwischen Anwendungsbereichen und Herausforderungen}
    \label{fig:04_02_01_mapping}
\end{figure}

\subsection{Zusammenspiel der Lernparadimgen und Algorithmen}
\label{subsec:mappingzwei}

\begin{figure}[H]
    \centering
    \includegraphics[width=\textwidth]{figures/04_ergebnisse/04_02_02_mappingZwei.png}
    \caption{Korrelation zwischen Lernparadigmen und Algorithmen}
    \label{fig:04_02_02_mappingZwei}
\end{figure}

\subsection{Zusammenspiel der Algorithmen und eingesetzten Tools}
\label{subsec:mappingDrei}

\subsection{Mapping von Anwendungsfelder und Lernparadigmen}
\label{subsec:mappingVier}