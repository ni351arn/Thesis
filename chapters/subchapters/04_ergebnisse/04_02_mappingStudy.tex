\section{Ergebnisse der Mapping Study}
\label{sec:mapping-study-ergebnisse}

Um die Beziehungen innerhalb der KI-Forschung im Platform Engineering weiter zu verdeutlichen, baut dieser Abschnitt auf den vorherigen Erkenntnissen auf und präsentiert detaillierte Kreuztabellenanalysen.
Diese Zuordnungen untersuchen Beziehungen zwischen Anwendungsfeldern und Herausforderungen (Abschnitt 4.2.1), Lernparadigmen und Algorithmen (Abschnitt 4.2.2), Herausforderungen und KI-Technologien (Abschnitt 4.2.3) sowie Lernparadigmen, Algorithmen und Anwendungsfeldern (Abschnitt 4.2.4).
Die Ergebnisse werden als Bubble-Chart-Diagramme visualisiert, in denen die Blasengröße die Häufigkeit der jeweiligen Zuordnung (n) widerspiegelt.
Die Häufigkeiten (n) geben an, in wie vielen der betrachteten Studien die jeweilige Zuordnung vorkommt. 
Pro Studie wird eine Zuordnung je Kombination höchstens einmal gezählt, unabhängig davon, wie häufig sie im Text erwähnt wird.
Ziel ist es, durch diese systematischen Mappings tiefere Einblicke in Struktur und Schwerpunkte der aktuellen Forschung zu gewinnen.

\subsection{Zusammenspiel der Anwendungsfelder und Herausforderungen}
\label{subsec:mappingEins}
Die Abbildung \ref{fig:04_02_01_mapping} zeigt das Zusammenspiel zwischen den identifizierten Use Cases und den zentralen Herausforderungen im Platform Engineering. 
Im Gegensatz zur Abbildung \ref{fig:04_01_01_useCases} wurden hier alle in den analysierten Studien genannten Anwendungsbereiche berücksichtigt und den jeweils adressierten Herausforderungen zugeordnet. 
Die Häufigkeiten geben an, wie oft eine bestimmte Kombination in der Literatur thematisiert wurde.

\begin{figure}[H]
    \centering
    \includegraphics[width=\textwidth]{figures/04_ergebnisse/04_02_mapping/04_02_01_mapping.png}
    \caption{Korrelation zwischen Anwendungsfelderen (Use Cases) und Herausforderungen}
    \label{fig:04_02_01_mapping}
\end{figure}

Besonders deutlich wird die starke Verknüpfung der Use Cases Ressourcen- und Workload-Optimierung sowie Resilienz, Self-Healing und Predictive Maintenance mit nahezu allen Herausforderungskategorien. 
Die Ressourcen- und Workload-Optimierung weist in allen Bereichen hohe Werte auf, je nach Herausforderung liegen diese zwischen n=13 und n=17.
Dieser Anwendungsfall verdeutlicht ein Spannungsfeld zwischen hohem Ressourcenbedarf, Datenmanagement und Pipeline-Komplexität (bedingt durch große und heterogene Datenmengen) sowie Trade-offs zwischen Skalierbarkeit und Latenz, wie es auch in \cite{bajwaCLOUDNATIVEARCHITECTURESLARGESCALE2025} beschrieben wird.
Ein ähnlich breites Muster zeigt sich im Bereich Self-Healing und Predictive Maintenance, der über alle Herausforderungskategorien hinweg hohe Häufigkeiten erreicht (zwischen n= 12 und n=17).
Insgesamt deutet dies darauf hin, dass beide Use Cases in der Literatur als besonders relevante Einsatzfelder für KI-basierte Optimierungen betrachtet werden, da sie auf zentrale betriebliche Ziele wie Effizienz, Zuverlässigkeit und einen höheren Automatisierungsgrad in DevOps-Prozessen einzahlen \cite{kankanalaAIMLDevOps2024}.

Der Use Case Sicherheits- und Bedrohungserkennung zeigt seine höchste Ausprägung im Bereich AI Governance, Datenschutz und Compliance (n=14) sowie bei Ressourcenverbrauch und Kosten (n=14). 
Auch die übrigen Herausforderungen weisen hohe Werte auf (n=13, 11, 10). 
Dies spiegelt wider, dass sicherheitsbezogene KI-Ansätze nicht nur regulatorische und organisatorische Fragen berühren, sondern ebenso Einfluss auf Monitoring, Ressourcenmanagement und Integrationsprozesse haben.

Die CI/CD- und Pipeline-Optimierung zeigt im Vergleich moderate, aber dennoch durchgehend hohe Werte über alle Herausforderungen hinweg (n=10, 9, 9, 9, 8). 
Dies deutet darauf hin, dass KI-Anwendungen zur Optimierung von Build-, Test- und Deployment-Prozessen zwar breit diskutiert werden, aber weniger stark im Zentrum systemübergreifender Herausforderungen stehen als andere Anwendungsbereiche.

Etwas weniger dominant, aber klar erkennbar, ist die Verteilung im Bereich Intelligente Bereitstellung (Edge-AI und Serverless). 
Die Häufigkeiten liegen zwischen n=14 und n=11 und zeigen damit ein insgesamt konsistentes, jedoch weniger dominantes Muster als in den zuvor genannten Bereichen. 
Dies legt nahe, dass dieses Anwendungsfeld im aktuellen Forschungsstand zwar relevant ist, jedoch nicht denselben Schwerpunkt bildet wie die großen Themenfelder Optimierung, Resilienz und Sicherheit.

Insgesamt zeigt die Verteilung der Häufigkeiten, dass KI-Anwendungen im Platform Engineering besonders dort adressiert werden, wo Effizienz, Stabilität, Automatisierung und Sicherheitsaspekte zusammenwirken. 
Die Häufungen bei einzelnen Anwendungsbereichen zeigen, welche Themen in der Forschung besonders im Fokus stehen.
Dadurch wird verdeutlicht, in welchen Bereichen KI-Lösungen aktuell als besonders relevant angesehen werden.


\subsection{Zusammenspiel der Lernparadigmen und Algorithmen}
\label{subsec:mappingzwei}
Die Abbildung \ref{fig:04_02_02_mappingZwei} visualisiert die Korrelation zwischen den in den analysierten Studien verwendeten Lernparadigmen des maschinellen Lernens und den eingesetzten Algorithmen.
Einige Kombinationen werden in der Darstellung nicht ausgewiesen (n=0 bzw. ohne Blase), da sie in den betrachteten Studien nicht eindeutig als eigenständige Zuordnung berichtet wurden.

\begin{figure}[H]
    \centering
    \includegraphics[width=\textwidth]{figures/04_ergebnisse/04_02_mapping/04_02_02_mappingZwei.png}
    \caption{Korrelation zwischen Lernparadigmen und Algorithmen}
    \label{fig:04_02_02_mappingZwei}
\end{figure}

Supervised Learning weist die größte Bandbreite an möglichen Algorithmen auf. 
Dies liegt daran, dass überwachte Lernverfahren sowohl tiefen neuronalen Netzen als auch klassischen Klassifikations- und Regressionsmodellen sowie ensemblebasierten Ansätzen zugrunde liegen. 
Der hohe Anteil an SL-Kombinationen (n=14 DL/NN, n=8 Klassifikation/Regression, n=7 Ensemble/baum-basiert) zeigt, dass SL sehr flexibel einsetbar ist.
Solange gelabelte Daten vorliegen, können verschiedene Algorithmen angewendet werden. 

Unsupervised Learning zeigt hingegen ein engeres Spektrum. 
Die Ergebnisse verdeutlichen, dass UL hauptsächlich in Kombination mit Deep Learning eingesetzt wird (n=9) oder mit Clustering-Methoden (n=6). 
Dies ist erwartungsgemäß, da UL in den untersuchten Studien vor allem zur Mustererkennung und Anomalieanalyse eingesetzt wird und Clustering hierbei eine zentrale Rolle einnimmt, wie in \cite{enemosahEnhancingDevOpsEfficiency2025} beschrieben.

Reinforcement Learning tritt zwar in mehreren Studien auf, wird jedoch selten algorithmisch konkretisiert. 
RL wird vor allem für dynamische Optimierungsaufgaben eingesetzt. 
Durch das Lernen auf Basis beobachteter Systemzustände und Feedback können RL-Modelle geeignete Aktionen auswählen. 
Dadurch können sie zur Stabilisierung des Betriebs und zur Verbesserung der Systemleistung beitragen.
Einzelne Arbeiten nutzen hierfür Deep-RL-Ansätze wie Deep Q-Networks \cite{jossonpaulkalapparambathAdvancingDistributedSystems2025}, insbesondere wenn komplexe Zustandsräume berücksichtigt werden müssen. 
Insgesamt bleibt die Zuordnung zu spezifischen Algorithmen jedoch begrenzt, weshalb viele Kombinationen in der Tabelle nicht belegt sind.


\subsection{Zusammenspiel von Herausforderungen und KI-Technologien (Tools und Frameworks)}
\label{subsec:mappingDrei}
Im Folgenden wird das Zusammenspiel von Herausforderungen und KI-Technologien (Tools \& Frameworks) anhand der Abbildung \ref{fig:04_02_03_mappingDrei} analysiert.
Die Werte geben an, in wie vielen Publikationen (n) eine bestimmte Herausforderung gemeinsam mit einer Tool- bzw. Technologiekategorie adressiert wird.

\begin{figure}[H]
    \centering
    \includegraphics[width=\textwidth]{figures/04_ergebnisse/04_02_mapping/04_02_03_mappingDrei.png}
    \caption{Korrelation zwischen Herausforderungen und KI-Technologie}
    \label{fig:04_02_03_mappingDrei}
\end{figure}
Deutlich wird, dass MLOps-Tools \& Orchestrierung über alle Herausforderungen hinweg am stärksten vertreten ist (zwischen n=11 und n=14).
Damit zeigt sich, dass MLOps-nahe Technologien in den betrachteten Publikationen besonders häufig im Zusammenhang mit den Herausforderungen adressiert werden.

Für Ressourcenverbrauch und Kosten sowie Skalierbarkeit, Latenz und Monitoring ergeben sich die höchsten Gesamtwerte über alle Tool-Kategorien hinweg.
In beiden Fällen treten ML/DL Frameworks \& Bibliotheken sowie Monitoring vergleichbar häufig auf (jeweils n=11).
Das spricht dafür, dass diese Herausforderungen häufig zusammen mit Entwicklungs- und Betriebsaspekten diskutiert werden.

Bei AI Governance, Datenschutz und Compliance zeigt sich ein ähnlicher Schwerpunkt.
MLOps-Tools \& Orchestrierung werden hier am häufigsten zusammen mit dieser Herausforderung genannt (n=12), gefolgt von Monitoring (n=10).
ML/DL Frameworks \& Bibliotheken treten etwas seltener auf (n=8).
Insgesamt werden Governance- und Compliance-Themen damit eher auf Betriebs- als nur auf Modell- oder Trainingsebene verortet.

Integrationskomplexität \& Abhängigkeiten wird zwar häufig als Herausforderung gennant (vgl. \ref{subsec:herausforderungen}), jedoch seltener explizit mit konkreten Toolkategorien in Verbindung gebracht.
Auch hier liegt der Schwerpunkt bei MLOps-Tools \& Orchestrierung (n=12) und Monitoring (n=9).
ML/DL Frameworks \& Bibliotheken (n=8) und Vektordatenbanken (n=4) treten seltener auf.
Das verweist darauf, dass Integrations- und Abhängigkeitsfragen vor allem auf System- und Pipeline-Ebene diskutiert werden.

Für Datenqualität, Datenverfügbarkeit und Heterogenität liegen die Werte im Mittelfeld.
MLOps wird am häufigsten gemeinsam adressiert (n=11), während ML/DL Frameworks und Monitoring jeweils bei n=9 liegen.
Vektordatenbanken sind hier vergleichsweise stärker vertreten (n=6).

Zusammenfassend ist MLOps die am häufigsten vertretene Tool-Kategorie über alle Herausforderungen hinweg.
Vektordatenbanken treten insgesamt seltener auf, sind jedoch insbesondere in datenbezogenen Kontexten sichtbar.

\subsection{Mapping der zentralen Dimensionen}
\label{subsec:mappingVier}
In diesem Abschnitt wird das Zusammenspiel von Lernparadigmen, Algorithmen und den identifizierten Anwendungsfeldern betrachtet.
In der Abbildung \ref{fig:04_02_04_mappingVier} wurden alle Kategorien miteinander dargestellt.
Kombinationen ohne Nachweis in den betrachteten Studien werden als n=0 dargestellt (ohne Blase)

\begin{figure}[H]
    \centering
    \includegraphics[width=\textwidth]{figures/04_ergebnisse/04_02_mapping/04_02_04_mappingVier.png}
    \caption{Mapping von Lernparadigmen, Algorithmen und Anwendungsfeldern}
    \label{fig:04_02_04_mappingVier}
\end{figure}

Im Supervised Learning dominiert die Kombination aus DL \& neuronalen Netzen in allen Anwendungsfeldern. 
Besonders häufig wird sie bei Ressourcen- und Workload-Optimierung sowie Self-Healing/Predictive Maintenance genannt (jeweils n=12), gefolgt von Sicherheits- und Bedrohungserkennung (n=11) und Intelligenter Bereitstellung (n=10).
Auch die Optimierung von CI/CD-Pipelines wird häufig mit DL adressiert (n=7). 
Insgesamt zeigt sich damit, dass Supervised Learning in der betrachteten Literatur vor allem in Verbindung mit DL und neuronalen Netzen eingesetzt wird.

Neben DL werden im Supervised Learning auch Ensemble- und baum-basierte Methoden sowie klassische Klassifikations- und Regressionsverfahren erwähnt, jedoch mit deutlich geringeren Häufigkeiten. 
Ensemble- und baum-basierte Ansätze liegen je nach Anwendungsfeld zwischen n=3 und n=5, während klassische Klassifikation/Regression insbesondere bei Ressourcen- und Workload-Optimierung (n=6) etwas häufiger vorkommt. 
Ansonsten sind die Werte ebenfalls geringer und liegen zwischen n=2 und n=5. 
Clustering wird im Kontext von Supervised Learning nicht beobachtet, sodass diese Zuordnungen nicht belegt sind (n=0).

Für Unsupervised Learning zeigt sich ein anderes Muster. Auch hier wird die Kombination aus DL \& neuronalen Netzen über alle Anwendungsfelder hinweg am häufigsten genannt (n=6–8). 
Clustering-Verfahren treten als zweite zentrale Kategorie auf, werden jedoch deutlich seltener thematisiert (n=3–5).
Insgesamt deutet dies darauf hin, dass Unsupervised Learning hauptsächlich dort eingesetzt wird, wo Abweichungen und Muster ohne gelabelte Daten erkannt werden sollen (z. B. bei Anomalien oder Wartungsvorhersagen).
Andere Algorithmusklassen (z. B. Ensemble- bzw. klassische Klassifikations-/Regressionsverfahren) sind im Unsupervised-Learning-Kontext nicht belegt (n=0), wodurch sich eine klare Fokussierung auf DL/NN und Clustering ergibt.

Im Reinforcement Learning konzentrieren sich die Zuordnungen ausschließlich auf Deep Learning und neuronale Netze. 
Die Häufigkeiten sind insgesamt geringer als im Supervised- und Unsupervised Learning. 
Die höchsten Werte zeigen sich bei der Sicherheits- und Bedrohungserkennung (n=7) sowie bei der Ressourcen- und Workload-Optimierung (n=6). Für die Optimierung von CI/CD-Pipelines und Self-Healing/Predictive Maintenance liegen die Werte jeweils bei n=5, während Intelligente Bereitstellung am seltensten genannt wird (n=4).
Andere Algorithmusklassen sind im RL-Kontext nicht belegt (n=0), was die enge Kopplung von RL an DL-basierte (Deep-RL) Ansätze in den betrachteten Publikationen unterstreicht.
\par\bigskip
Die Kreuztabellenanalysen machen sichtbar, welche Zusammenhänge in der Forschung wiederkehrend sind.
Darauf aufbauend wird in Abschnitt 4.3 ein Matching-Framework abgeleitet, das die Ergebnisse strukturiert zusammenfasst.
Ziel ist es, daraus eine nachvollziehbare Grundlage für die Bewertung bzw. Einordnung von KI-Ansätzen im Platform Engineering zu schaffen.