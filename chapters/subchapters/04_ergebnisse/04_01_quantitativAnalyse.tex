\section{Quantitative Analyse}
\label{sec:quantitativeAnalyse}

Zur Einordnung des untersuchten Forschungsfeldes wurden zunächst grundlegende Merkmale der insgesamt 21 Studien analysiert. Abbildung \ref{fig:04_01_quantitativeAnalyse} a) zeigt die jährliche Verteilung der Publikationen sowie die Zuordnung zu verschiedenen Publikationstypen. Zwischen 2021 und 2023 erscheinen nur wenige Arbeiten (insgesamt fünf), während ab 2024 ein deutlicher Anstieg sichtbar wird. 
In den Jahren 2024 und 2025 wurden jeweils acht Publikationen identifiziert, überwiegend Journalartikel, ergänzt durch einzelne Konferenzbeiträge, ArXiv-Papers und ein Whitepaper. Dies weist auf ein zunehmendes wissenschaftliches Interesse am Einsatz von KI in Cloud-Native- und Platform-Engineering-Kontexten hin.

Ergänzend dazu zeigt eine Word Cloud (Abbildung \ref{fig:04_01_quantitativeAnalyse} b) die am häufigsten vorkommenden Keywords aus allen Publikationen. Die Visualisierung bietet einen schnellen Überblick über zentrale thematische Schwerpunkte der Literatur und unterstützt die anschließende inhaltliche Analyse.

\begin{figure}[H]
    \centering
    \includegraphics[width=\textwidth]{figures/04_ergebnisse/04_01_quantitativeAnalyse.png}
    \caption{Jährliche und thematische Verteilung der DevOps AI-Forschung}
    \label{fig:04_01_quantitativeAnalyse}
\end{figure}

Diese Betrachtung bildet die Grundlage für die folgenden Unterkapitel, in denen die Studien hinsichtlich ihrer inhaltlichen Merkmale detaillierter ausgewertet werden.

\subsection{Anwendungsbereiche der KI im Platform Engineering}
\label{subsec:anwendungsbereiche}
Im ersten Schritt der quantitativen Analyse wurden die insgesamt 21 identifizierten Studien hinsichtlich ihrer Hauptanwendungsbereiche untersucht. 
Dazu wurden die ausgewählten Bereiche in fünf Kategorien zusammengfasst: Optimierung von CI/CD Pipelines, Ressourcen- und Workload Optimierung, Sicherheits- und Bedrohungserkennung, Resilienz, Self-Healing und Predictive Maintenance sowie Intelligente Bereitstellung (Edge-AI und Serverless Deployment).

Die Auswertung zeigt, dass CI/CD-Automatisierung mit acht Publikationen am häufigsten als primärer Anwendungsbereich beschrieben wird. Darauf folgt die Ressourcen- und Workloadoptimierung, was in vier Paper als primärerer Anwendun identifiziert wurde.
Ebenfalls wird die intelligente Bereitstellung mittels Edge-AI oder Serverless Deployment in vier Publikationen als zentraler Anwendungsbereich beschrieben.
Der vierte Bereich, Resilienz, Self-Healing und Predictive Maintenance wird insgesamt drei mal als Hauptanwendungsbereich definiert.
Sicherheits- und Bedrohungserkennung steht in zwei Studien im Mittelpunkt. 
Abbildung \ref{fig:04_01_01_useCases} zeigt diese Verteilung.
Zu beachten ist, dass viele Studien mehr als einen Bereich ansprechen. Für die Vergleichbarkeit wurde jedoch jeweils der dominante Use Case pro Publikation ausgewählt. In der Praxis überschneiden sich die Anwendungsbereiche häufig, da KI-Lösungen oft mehrere Aufgaben gleichzeitig unterstützen.

\begin{figure}[H]
    \centering
    \includegraphics[width=\textwidth]{figures/04_ergebnisse/04_01_01_useCases.png}
    \caption{Verteilung der Anwendungsbereiche}
    \label{fig:04_01_01_useCases}
\end{figure}


\subsection{Herausforderungen der KI-Integration im Platform Engineering}
\label{subsec:herausforderungen}
Im nächsten Schritt wurden die in den Publikationen beschriebenen Herausforderungen analysiert, die beim Einsatz von KI im Platform Engineering auftreten. 
Die analysierten Paper wurden fünf Kategorien zugeordnet. Die Auswertung zeigt, dass 95 \% der Studien Herausforderungen im Bereich Ressourcenverbrauch und Kosten nennen. 
Ebenfalls häufig angesprochen werden Skalierbarkeit, Latenz und Monitoring mit 90 \%. 
AI Governance, Datenschutz und Compliance findet sich in 81 \% der Arbeiten wieder, ebenso wie die Kategorie Integrationskomplexität  und Abhängigkeiten.
Die fünfte Kategorie, Datenqualität, Datenverfügbarkeit und Heterogenität, wird in insgesamt 71 \% der Arbeiten als Herausforderung thematisiert.  

Diese Ergebnisse verdeutlichen, dass KI-Anwendungen im Platform Engineering häufig mit mehreren technischen, operativen und organisatorischen Hürden verbunden sind. 
Viele der Herausforderungen treten zudem gleichzeitig auf, da beispielswiese Skalierungsprobleme oft in engem Zusammenhang mit Ressourcenkosten oder Datenqualität stehen. 
Die prozentuale Verteilung ist in Abbildung \ref{fig:04_01_02_herausforderungen} dargestellt.

\begin{figure}[H]
    \centering
    \includegraphics[width=\textwidth]{figures/04_ergebnisse/04_01_02_herausforderungen.png}
    \caption{Relative Häufigkeit der Herausforderungen}
    \label{fig:04_01_02_herausforderungen}
\end{figure}


\subsection{Formen des maschinellen Lernens}
\label{subsec:formenDesMaschinellenLernens}
Ein weiterer Bestandteil der quantitativen Analyse umfasst die in den Publikationen verwendeten Formen bzw. Lernparadigmen des maschinellen Lernens. 
Dabei wurde unterschieden, ob die jeweilige Form explizit oder implizit genannt wird.
Insgesamt wurden die drei grundlegenden Paradigmen identifiziert: Supervised Learning, Unsupervised Learning und Reinforcement Learning.
In Kapitel \ref{subsec:lernparadigmen} werden diese Paradigmen detaillierter erläutert und durch typische Einsatzformen und Methoden veranschaulicht. 

Die Auswertung zeigt, dass Supervised Learning in den meisten Arbeiten eine zentrale Rolle spielt. Es wird es in acht Publikationen ausdrücklich erwähnt und ist in weiteren elf Arbeiten erkennbar. 
Reinforcement Learning wird ausschließlich explizit genannt und in insgesamt elf Publikationen thematisiert. 
Unsupervised Learning tritt mit drei expliziten und acht impliziten Nennungen seltener auf, ist jedoch ebenfalls präsent.


Die Verteilung ist in der folgenden Abbildung \ref{fig:04_01_03_lernparadigmen} dargestellt.
\begin{figure}[H]
    \centering
    \includegraphics[width=\textwidth]{figures/04_ergebnisse/04_01_03_lernparadigmen.png}
    \caption{Formen des maschinellen Lernens}
    \label{fig:04_01_03_lernparadigmen}
\end{figure}

Auffällig ist die deutliche Differenz zwischen explizit und implizit erkennbaren Anwendungen, insbesondere bei Supervised und Unsupervised Learning. 
Dies zeigt, dass viele Publikationen die entsprechenden Paradigmen beschreiben, ohne die zugrunde liegende Lernform ausdrücklich zu benennen.



\subsection{Verwendete Algorithmen}
\label{subsec:methodenAlgorithmen}
Die in den Publikationen verwendeten Algorithmen lassen sich den jeweiligen Lernparadigmen zuordnen. 
Die Analyse zeigt, dass eine Vielfalt an KI- und ML-Verfahren im Platform-Engineering-Kontext eingesetzt wird. 
Für eine systematische Bewertung wurde eine eigene Kategorisierung entwickelt.
Den Ausgangspunkt bildet die Tabelle aus \textcite{enemosahEnhancingDevOpsEfficiency2025}, in der verschiedene AI-Techniken für Testfallpriorisierung beschrieben werden. 
Die Einteilung wurde für den breiteren Kontext dieser Arbeit angepasst.

Auf dieser Basis wurden vier Kategorien definiert, die in \ref{tab:tabelleMethodenAlgorithmen} kurz beschrieben sind. 
Anschließend folgt die quantitative Auswertung der identifizierten Methoden und Algorithmen. 
Dies zeigt, wie häufig die jeweiligen Verfahren in den Publikationen genannt werden. 
\begin{table}[htbp]
  \centering
  \caption{Beschreibung Algorithmen und Methoden}
  \label{tab:tabelleMethodenAlgorithmen}
  \footnotesize
  \renewcommand{\arraystretch}{1.15}
  \begin{tabularx}{\linewidth}{@{}lY@{}}
    \toprule
    \textbf{Algorithmen und Methoden} & \textbf{Beschreibung} \\
    \midrule
    Deep Learning (DL)/ Neuronale Netze (NN) & Erfassen komplexe Muster in Daten und eignen sich besonders für unstrukturierte Eingaben wie Log- oder Monitoring-Daten. \\
    Ensemble und Baum-basiert & Kombinieren mehrere Modelle zur Steigerung der Vorhersagegenaugikeit und verarbeiten große sowie semi-strukturiterte Datensätze effizient. \\
    Klassische Klassifikation/ Regression & Traditionelle ML-Modelle zur Vorhersage von Mustern oder Ereignnissen auf Basis strukturierter Daten. \\
    Clustering & Gruppieren Datenpunkte ohne Labels in inhaltlich ähnliche CLuster und unterstützen dadurch Musterekrennung und Anomaliedetektion. \\
    \bottomrule
  \end{tabularx}
\end{table}

\FloatBarrier 

Die Ergebnisse zeigen ein deutliches Übergewicht von Deep-Learning-Verfahren, die in insgesamt 86 \% der Publikationen eingesetzt oder thematisiert werden. 
Klassische Klassifikations- und Regressionsmethoden treten mit 52 \% ebenfalls häufig auf. 
Demgegenüber werden Clustering-Verfahren und Ensemble- bzw. baumbasierte Modelle in 33 \% der Arbeiten angesprochen.
Insgesamt zeigt sich, dass insbesondere neuronale Netze und Deep-Learning-Verfahren die dominierenden Methoden sind.

Diese Verteilung ist in der Abbildung \ref{fig:04_01_04_algorithmen} dargestellt.
\begin{figure}[H]
    \centering
    \includegraphics[width=\textwidth]{figures/04_ergebnisse/04_01_04_algorithmen.png}
    \caption{Verteilung der Algorithmen und angesporchenen Methoden}
    \label{fig:04_01_04_algorithmen}
\end{figure}
