\section{Quantitative Analyse}
\label{sec:quantitativeAnalyse}

Zur Einordnung des untersuchten Forschungsfeldes wurden zunächst grundlegende bibliographische Merkmale der insgesamt 21 Studien analysiert. Abbildung X zeigt die jährliche Verteilung der Publikationen sowie die Zuordnung zu verschiedenen Publikationstypen. Zwischen 2021 und 2023 erscheinen nur wenige Arbeiten (insgesamt fünf), während ab 2024 ein deutlicher Anstieg sichtbar wird: In den Jahren 2024 und 2025 wurden jeweils acht Publikationen identifiziert, überwiegend Journalartikel, ergänzt durch einzelne Konferenzbeiträge, ArXiv-Papers und ein Whitepaper. Dies weist auf ein zunehmendes wissenschaftliches Interesse am Einsatz von KI in Cloud-Native- und Platform-Engineering-Kontexten hin.

Ergänzend dazu zeigt eine Word Cloud (Abbildung X) die am häufigsten vorkommenden Keywords aus allen Publikationen. Die Visualisierung bietet einen schnellen Überblick über zentrale thematische Schwerpunkte der Literatur und unterstützt die anschließende inhaltliche Analyse.

Abbildung.

Diese Betrachtung bildet die Grundlage für die folgenden Unterkapitel, in denne die Studien hinsichtlich ihrere inhaltlichen Merkmale detaillierter ausgewertet werden. 