\section{Matching-Framework}
\label{sec:matching-framework}

Aufbauend auf den quantitativen Ergebnissen aus Abschnitt 4.1 sowie den Mustern der Mapping Study in Abschnitt 4.2 wird in diesem Abschnitt ein konzeptionelles Matching-Framework abgeleitet.
Ziel ist es, die identifizierten KI-Ansätze in übertragbare Anwendungsmuster für generische Use Cases im Platform Engineering zu strukturieren.
Damit entsteht eine nachvollziehbare Grundlage, um KI-Lösungen später hinsichtlich operativem Nutzen und Umsetzbarkeit einzuordnen.

\subsection{Muster 1: Prädiktives Auto-Scaling und Ressourcenoptimierung}
\label{subsec:muster1-autoscaling-ressourcen}
Das Muster Prädiktives Auto-Scaling und Ressourcenoptimierung beschreibt KI-gestützte Verfahren, um Ressourcen vorausschauend zu skalieren und damit Kosten sowie Performance-Risiken zu optimieren. 
Im Unterschied zu klassischem Auto-Scaling auf Basis statischer Schwellenwerte werden historische Betriebsdaten genutzt, um Lastspitzen frühzeitig zu antizipieren und Skalierungsentscheidungen rechtzeitig auszuführen.
Die Tabelle \ref{tab:predictiveAutoScaling} fasst die charakteristischen Merkmale dieses Musters zusammen.

% (Im Dokument an der Stelle, wo die Tabelle hin soll)

\clearpage % wichtig: neue Seite starten

\begin{landscape}
\begin{table}[p]
\begin{flushleft} % startet links (nicht zentriert)

\caption{Übersicht der KI-Anwendungsfelder: Tasks, Daten, Methoden und Ziele}
\label{tab:anwendungsfelder-5x5}

\footnotesize
\renewcommand{\arraystretch}{1.15}
\setlength{\tabcolsep}{4pt}

\begin{tabularx}{\linewidth}{p{3.6cm} Y Y Y Y}
\toprule
\textbf{Anwendungsfelder} &
\textbf{Konkrete Plattform-Task} &
\textbf{Genutzte Daten} &
\textbf{KI-Methode} &
\textbf{Ziel} \\
\midrule

Ressourcenoptimierung
& \emph{Vorausschauendes Auto-Scaling von Cloud-Ressourcen (z. B. AWS EC2, RDS, oder Kubernetes-Cluster), Dynamische Anpassung der VM-Konfigurationen und Right-sizing.")}
& \emph{"Datenquelle: Historische Leistungsdaten (CPU-Auslastung, Netzwerk-Traffic, Disk I/O, Speicherkapazität). , Nutzung: Gesammelt über AWS CloudWatch oder Prometheus/Grafana (für Kubernetes). , Die Daten werden zur Vorhersage zukünftiger Ressourcenbedarfstrends und Workload-Muster genutzt."}
& \emph{Prädiktive Analytik (Zeitreihenanalyse). Spezifische Algorithmen: Long Short-Term Memory (LSTM) neuronale Netze. 
RL zur dynamischen Anpassung der Ressourcenzuteilung.}
& \emph{"Reduzierung der Betriebskosten. 
Steigerung der Ressourceneffizienz und -auslastung (z. B. 25\% Verbesserung der Ressourcenauslastung). 
Wartung der optimalen Anwendungs-Performance."} \\
\addlinespace[2pt]

Zuverlässigkeit / CI/CD Automation
& \emph{Prädiktive Fehlererkennung in Build- und Deployment-Pipelines. 
Automatisierte Rollbacks bei erkannten Anomalien. 
Adaptive Deployment-Strategien (z. B. dynamische Anpassung von Blue/Green- oder Canary-Rollouts). }
& \emph{Datenquelle: Historische Build-Logs (mit Zeitstempeln, Fehlermeldungen, Warnungen). 
Test-Ergebnisse (Pass/Fail-Raten). 
System-Performance-Metriken (Latenz, Fehlerraten, Ressourcennutzung während des Deployments). 
Nutzung: Analyse von Mustern, die auf bevorstehende Fehler hindeuten. Kontinuierliches Monitoring zur Auslösung von automatisierten Korrekturmaßnahmen.}
& \emph{SL für Fehler-Vorhersage(Random Forests, Entscheidungsbäume, Neuronale Netze). 
UL für Anomalieerkennung ( Autoencoder, Isolation Forests, K-Means Clustering).
RL für optimierte Rollbacks und Testpriorisierung. (Deep Q-Networks (DQN), PPO, Q-Learning)"}
& \emph{Reduzierung der Deployment-Zeit (z. B. 40\% Reduzierung der Testzeit). Reduzierung der Fehlerraten (Error Rate Reduction, z. B. 15\%). Minimierung der Downtime (z. B. 30\% Reduzierung durch proaktive Rollbacks). Verbesserung der Software-Zuverlässigkeit.} \\
\addlinespace[2pt]

Sicherheit / Bedrohungserkennung
& \emph{(eintragen: z.B. Detection, Alert Triage, Policy Violations, Secret/Leak Scans)}
& \emph{(eintragen: z.B. Security-Logs, Netzwerk-Flows, Audit-Logs, IAM-Events)}
& \emph{(eintragen: z.B. Anomalieerkennung, Clustering, Klassifikation)}
& \emph{(eintragen: z.B. Angriffe erkennen, False Positives senken, Compliance unterstützen)} \\
\addlinespace[2pt]

Betrieb / Orchestrierung
& \emph{(eintragen: z.B. Incident Response, Self-Healing, Runbook Automation, Triage)}
& \emph{(eintragen: z.B. Alerts, Tickets, SLO/SLI, Logs/Traces, Topologie-Daten)}
& \emph{(eintragen: z.B. Root-Cause-Ansätze, Empfehlungssysteme, Agenten/Workflow-Automation)}
& \emph{(eintragen: z.B. Stabilität erhöhen, Operator entlasten, schnellere Entstörung)} \\
\bottomrule
\end{tabularx}

\end{flushleft}
\end{table}
\end{landscape}

\clearpage % wichtig: danach wieder zurück ins Hochformat


Die Daten stammen typischerweise aus Cloud- und Cluster-Monitoring, z. B. AWS CloudWatch oder Prometheus. 
Relevante Metriken sind CPU-Auslastung, Arbeitsspeicher, Disk-I/O und Netzwerkverkehr \cite{poudelAIDrivenIntelligentAutoScaling2025a, zaaloukCLOUDNATIVEARTIFICIAL}.

Die Metriken werden als Zeitreihen ausgewertet. 
Ziel ist es, wiederkehrende Muster (z. B. tages-/wochenbasierte Zyklen) und Lastspitzen zu erkennen, um daraus Prognosen für den zukünftigen Ressourcenbedarf abzuleiten \cite{poudelAIDrivenIntelligentAutoScaling2025a, zaaloukCLOUDNATIVEARTIFICIAL}.

Für die Prognose kommen vor allem überwachte Lernverfahren zum Einsatz, insbesondere LSTM für Zeitreihen-Vorhersagen. 
Ergänzend wird Reinforcement Learning genutzt (z. B. Q-Learning, PPO), um eine Strategie zu lernen, welche Skalierungsaktion unter bestimmten Zuständen langfristig am besten ist \cite{poudelAIDrivenIntelligentAutoScaling2025a, tamminediAutomatingKubernetesOperations2024}.

Für die technische Ausführung werden Cloud- und Plattform-Tools eingebunden, z. B. AWS Boto3 zur Anpassung von Instanzen und KEDA zur ereignis- bzw. metrikbasierten Pod-Skalierung in Kubernetes. 
Für Modelltraining und -betrieb werden typische ML-Frameworks wie TensorFlow/PyTorch genutzt \cite{poudelAIDrivenIntelligentAutoScaling2025a, zaaloukCLOUDNATIVEARTIFICIAL, kankanalaAIMLDevOps2024}.


\subsection{Muster 2: Intelligente CI/CD (Fehlervorhersage und adaptive Rollbacks)}
\label{subsec:muster2-intelligente-cicd}
Das zweite Architekturpattern konzentriert sich auf die intelligente CI/CD-Pipeline (Intelligent CI/CD). 
Dabei werden die Prozesse der kontinuierlichen Integration und Bereitstellung durch prädiktive Fehlererkennung und adaptive Rollback-Mechanismen optimiert. 
Während klassische Pipelines auf statischen Regeln basieren, ermöglichen KI-Modelle eine dynamische Reaktion auf Anomalien während des Release-Prozesses \cite{enemosahEnhancingDevOpsEfficiency2025,tamanampudiAIEnhancedContinuousIntegration,supritpattanayakIntegratingAIDevOps2024}.
Die Tabelle \ref{tab:pattern2-intelligent-cicd} gibt einen detaillierten Überblick über die Komponenten dieses Patterns:

\begin{table}[htbp]
\caption{Intelligent CI/CD (Predictive Failure \& Adaptive Rollbacks)}
\label{tab:pattern2-intelligent-cicd}
\footnotesize
\renewcommand{\arraystretch}{1.2}
\setlength{\tabcolsep}{6pt}

\begin{tabularx}{\linewidth}{p{4.2cm} Y p{3.0cm}}
\toprule
\textbf{Kategorie} & \textbf{Details} & \textbf{Quellen (IDs)} \\
\midrule

Plattform-Aufgabe &
Vorhersage von Build-Fehlern; automatisierte Rollbacks bei Anomalien nach dem Deployment. &
\cite{tamanampudiAIEnhancedContinuousIntegration}, \cite{enemosahEnhancingDevOpsEfficiency2025} \\

Datenquellen &
Logs: Build-Logs, Testresultate, Code-Metadaten (Commit-Historie), Fehler-Raten während des Rollouts. &
\cite{tamanampudiAIEnhancedContinuousIntegration}, \cite{supritpattanayakIntegratingAIDevOps2024} \\
Datenverwendung &
Training von Modellen auf historischen Fehlermustern, um \glqq High-Risk\grqq{}-Änderungen frühzeitig zu stoppen. &
\cite{tamanampudiAIEnhancedContinuousIntegration} \\

KI-Methoden &
Supervised Learning (Random Forest, Decision Trees); Reinforcement Learning (DQN) für optimierte Rollback-Entscheidungen. &
\cite{tamanampudiAIEnhancedContinuousIntegration} \\
Konkrete Tools &
Jenkins X (mit ML-Plugins), Spinnaker, StackStorm (für ereignisbasierte Automatisierung). &
\cite{kankanalaAIMLDevOps2024}, \cite{supritpattanayakIntegratingAIDevOps2024} \\

Vorteile (Ziele) &
Reduzierung der Deployment-Zeit um 30\%; Steigerung der Systemstabilität durch proaktive Fehlervermeidung. &
\cite{kankanalaAIMLDevOps2024}, \cite{tamanampudiAIEnhancedContinuousIntegration} \\
\bottomrule
\end{tabularx}
\end{table}


Die prädiktive Fehlererkennung handelt konkret davon, KI-Modelle darauf zu trainieren, Muster in historischen Daten zu erkennen, die in der Vergangneheit, höufig zu Build- oder Deployment Fehler geführt haben. 
Hierbei werden Code-Änderungen und Testresultate analysiert, wodurch die KI ein hohe Risiko vorhersagen kann bevor der Code in die Produktionsumgebung gelangt.
Dies ermöglicht es Plattform-Engineerings, risikoreiche Änderungen frühzeitig zu stoppen und die Erfolgsrate von Buid signifikant zu steigern. 

Anomalie-basierte Rollbacks ergänzen diese Funktionalität, indem die KI Echtzeit-Metriken unmittelbar nach einem Release überwacht. 
Algorithmen wie Isolation FOrests oder Autoencoder identifizieren Abweichungen vom normalen Systemverhalten, wie etwas ungewöhnliche Latensptizen oder erhöhte Fehlerraten. 
Wird eine Anomalie erkannt, triggert das System autonom einen Rollback auf einen stabilen Zustand, um die Auswirkungen auf Endnutzer zu minimieren \cite{tamanampudiAIEnhancedContinuousIntegration,supritpattanayakIntegratingAIDevOps2024,reddygopireddyIntegratingAIDevOps2022,enemosahEnhancingDevOpsEfficiency2025}.

KI wird zudem eingesetzt um Testsequenz innerhalb der Pipeline intelligent zu priorisieren.
Mittels Reinforcement Learning lernt das System, welche Tests die höchste Wahrscheinlichkeit haben, kritische Fehler in einer spezifischen Code-Änderung aufzudecken. 
Dies reduziert die Gesamtlaufzeit der Pipeline, ohne die Qualitätssicherung zu beeinträchtigen \cite{kathiresanCybersecurityRiskModeling2025,reddygopireddyIntegratingAIDevOps2022}.

Der Einsatz dieses Pttern führt zu einer Reduktion des manuellen Intervemtionsaufwands bei fehlgeschlagenen Deployments.
Die Deployment-Zyklen werden durch automatisierte Entscheidungfindung um etwas 30\% verkürzt \cite{kankanalaAIMLDevOps2024}.
Zudem verbessert sich die Systemstabilität, da die KI als "Sicherheitsnetz" fungiert und fehlerhafte Releases abfängt, bevor sie kritische Systemzustände verursachen \cite{tamanampudiAIEnhancedContinuousIntegration,reddygopireddyIntegratingAIDevOps2022}.

\subsection{Muster 3: KI-gestützte Observability und Ursachenanalyse (AIOps)}
\label{subsec:muster3-observability-rca}
Das dritte Muster im Matching-Framework fokussiert sich auf KI-gestützte Observability und Ursachenanalyse, auch bekannt als AIOps (Artificial Intelligence for IT Operations).
In modernen Cloud-native Umgebungen übersteigt die Menge der genreirten Telemetriedaten oft die Kapazitäten manueller Analyse.
Klassische Monitoring-Ansätze sind häufig reaktiv und führen zu einer hohen Anzahl von Fehlalarmen.
Durch den Einsatz von KI können Plattform-Engineers unstrukturierte Log-Daten ud komplexe Systemzustände in Echtzeit interpetieren.
Die Tabelle \ref{tab:pattern3-aiops-observability-rca} fasst die wesentlichen Merkmale dieses Musters zusammen:

\begin{table}[htbp]
\caption{AI-driven Observability \& Root Cause Analysis (AIOps)}
\label{tab:pattern3-aiops-observability-rca}
\footnotesize
\renewcommand{\arraystretch}{1.2}
\setlength{\tabcolsep}{6pt}

\begin{tabularx}{\linewidth}{p{4.2cm} Y p{3.0cm}}
\toprule
\textbf{Kategorie} & \textbf{Details} & \textbf{Quellen (IDs)} \\
\midrule

Plattform-Aufgabe &
Log-Anomalie-Erkennung; automatisierte Root Cause Analysis (RCA); Natural Language Interface für Cluster-Abfragen. &
\cite{zaaloukCLOUDNATIVEARTIFICIAL}, \cite{tamminediAutomatingKubernetesOperations2024} \\

Datenquellen &
Telemetrie: System-Logs, Kubernetes Events, Traces (OpenTelemetry), Warnsignale von Sensoren/Nodes. &
\cite{zaaloukCLOUDNATIVEARTIFICIAL}, \cite{guptaCloudNativeMLArchitecting2024a} \\
Datenverwendung &
Mustererkennung in unstrukturierten Log-Daten zur Korrelation von Fehlern über Microservices hinweg. &
\cite{zaaloukCLOUDNATIVEARTIFICIAL} \\

KI-Methoden &
LLMs (Natural Language Processing) zur Interpretation von Fehlermeldungen; Clustering (K-Means) zur Gruppierung ähnlicher Alerts. &
\cite{zaaloukCLOUDNATIVEARTIFICIAL}, \cite{enemosahEnhancingDevOpsEfficiency2025} \\
Konkrete Tools &
K8sGPT, OpenLLMetry, Prometheus/Grafana mit KI-Extensions. &
\cite{zaaloukCLOUDNATIVEARTIFICIAL}, \cite{tamminediAutomatingKubernetesOperations2024} \\

Vorteile (Ziele) &
Erhöhung der MTTR (Mean Time To Resolution) um bis zu 39\%; Reduktion von False Positives in der Alarmierung. &
\cite{tamminediAutomatingKubernetesOperations2024} \\
\bottomrule
\end{tabularx}
\end{table}


Im Kern adressiert das Pattern operative Plattformaufgaben im AIOps-Kontext: Es erkennt Auffälligkeiten in Logs, unterstützt eine automatisierte Ursachenanalyse und ermöglicht zusätzlich natürlichsprachliche Abfragen an den Cluster, um Fehler schneller einzugrenzen \cite{zaaloukCLOUDNATIVEARTIFICIAL, tamminediAutomatingKubernetesOperations2024}.

Als Datengrundlage werden verschiedene Telemetriequellen zusammengeführt, insbesondere System-Logs, Kubernetes-Events und verteilte Traces (z. B. OpenTelemetry) sowie Warnsignale aus der Infrastruktur. 
Dadurch entsteht ein konsistenteres Bild des Systemzustands über mehrere Komponenten hinweg \cite{zaaloukCLOUDNATIVEARTIFICIAL, guptaCloudNativeMLArchitecting2024a}.

Die Datennutzung fokussiert unstrukturierte Log-Daten: Muster werden erkannt und so korreliert, dass Zusammenhänge von Fehlern über Microservices hinweg sichtbar werden \cite{zaaloukCLOUDNATIVEARTIFICIAL}.

Methodisch werden große Sprachmodelle genutzt, um Fehlermeldungen verständlich zu interpretieren, während Clustering (z. B. K-Means) ähnliche Alarme bündelt und Alarmfluten reduziert \cite{zaaloukCLOUDNATIVEARTIFICIAL, enemosahEnhancingDevOpsEfficiency2025}.

Auf Werkzeugebene wird das Pattern typischerweise durch K8sGPT (natürlichsprachliche Unterstützung), OpenLLMetry (herstellerneutrale Integration) sowie Prometheus/Grafana mit KI-Erweiterungen umgesetzt \cite{zaaloukCLOUDNATIVEARTIFICIAL, tamminediAutomatingKubernetesOperations2024}.

Der operative Nutzen liegt vor allem in schnellerer Fehlerbehebung und besserer Alarmqualität; die Quellen berichten u. a. eine MTTR-Reduktion von bis zu 39 \% sowie weniger Fehlalarme in der Alarmierung \cite{tamminediAutomatingKubernetesOperations2024}.

\subsection{Muster 4: Intelligente Sicherheit und Bedrohungserkennung}
\label{subsec:muster4-security}
Intelligente Sicherheit und Bedrohungserkennung beschreibt KI-gestützte Verfahren, um Sicherheitsereignisse in Cloud-Native-Plattformen frühzeitig zu erkennen und schneller einzuordnen. 
Im Unterschied zu rein regelbasierten Ansätzen werden dabei Muster aus Betriebs- und Sicherheitsdaten gelernt, um Auffälligkeiten automatisch zu identifizieren. 
Die Tabelle \ref{tab:pattern4-intelligent-security-threat-detection} fasst die Kernelemente dieses Patterns zusammen.

% --- Tabelle 4: Intelligente Sicherheit & Bedrohungserkennung ---
\begin{table}[H]
\caption{Intelligente Sicherheit \& Bedrohungserkennung}
\label{tab:pattern4-intelligent-security-threat-detection}
\footnotesize
\renewcommand{\arraystretch}{1.2}
\setlength{\tabcolsep}{6pt}

\begin{tabularx}{\linewidth}{p{4.2cm} Y p{3.2cm}}
\toprule
\textbf{Kategorie} & \textbf{Beschreibung} & \textbf{Beispielquellen} \\
\midrule

Plattformaufgabe &
Erkennung von Intrusionen (DDoS, Malware), Anomalieerkennung im Netzwerkverkehr sowie im Benutzerverhalten. &
\cite{uddohAIBasedThreatDetection2021}, \cite{govindarajanMachineLearningBased2025} \\

Datenquellen &
Netzwerk: IP-Traffic, System-Logs, Cloud-Audit-Logs, Benutzer-Zugriffsmuster. &
\cite{uddohAIBasedThreatDetection2021}, \cite{govindarajanMachineLearningBased2025} \\

Datennutzung &
Abgleich von Echtzeit-Datenströmen mit Baselines für \glqq normales\grqq{} Verhalten zur Identifikation von Ausreißern. &
\cite{uddohAIBasedThreatDetection2021}, \cite{govindarajanMachineLearningBased2025} \\

KI-/ML-Methoden &
Deep Learning (CNN für visuelle Musteranalyse); XGBoost/Random Forest zur Klassifikation auf unausgewogenen Datensätzen (Klassenungleichgewicht). &
\cite{uddohAIBasedThreatDetection2021}, \cite{govindarajanMachineLearningBased2025} \\

Werkzeuge (Tools) &
IBM Watson AI, Microsoft Sentinel, Trivy/Clair (für Scans in CI/CD). &
\cite{kankanalaAIMLDevOps2024} \\

Ziele/Nutzen &
Detektionsrate von bis zu 95\% bei fortgeschrittenen Bedrohungen; Reduktion von Sicherheitsvorfällen. &
\cite{uddohAIBasedThreatDetection2021}, \cite{tamminediAutomatingKubernetesOperations2024} \\

\bottomrule
\end{tabularx}
\end{table}


Im Kern adressiert das Pattern die Erkennung von Angriffen (z. B. DDoS, Malware) sowie Anomalien im Netzwerkverkehr und im Benutzerverhalten. 
Dazu werden IP- und Netzwerk-Traffic, System- und Cloud-Audit-Logs sowie Zugriffsmuster als Datenbasis zusammengeführt \cite{uddohAIBasedThreatDetection2021, govindarajanMachineLearningBased2025}.

Die Datennutzung erfolgt häufig über den Abgleich von Echtzeit-Datenströmen mit Baselines für „normales“ Verhalten, sodass Ausreißer priorisiert und schneller untersucht werden können. 
Als Methoden werden Deep-Learning-Verfahren (z. B. CNNs zur Mustererkennung) sowie Modelle wie XGBoost oder Random Forest eingesetzt, insbesondere um trotz Klassenungleichgewicht robuste Klassifikationen zu ermöglichen \cite{uddohAIBasedThreatDetection2021, govindarajanMachineLearningBased2025}.

Auf Werkzeugebene werden Security-Plattformen wie IBM Watson AI oder Microsoft Sentinel genutzt und durch Scanner wie Trivy/Clair ergänzt, die Sicherheitsprüfungen in CI/CD-Pipelines integrieren. 
Der operative Nutzen liegt in höherer Erkennungsleistung und weniger Sicherheitsvorfällen; die Literatur berichtet Detektionsraten von bis zu 95 \% bei fortgeschrittenen Bedrohungen \cite{kankanalaAIMLDevOps2024, uddohAIBasedThreatDetection2021, tamminediAutomatingKubernetesOperations2024}.
