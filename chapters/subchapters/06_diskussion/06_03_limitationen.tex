\section{Limitationen}
\label{sec:limitationen}

Diese Arbeit basiert auf einer systematischen Literaturanalyse und einer darauf aufbauenden Anwendung auf eine konkrete Plattformumgebung. 
Daraus ergeben sich Grenzen, die bei der Einordnung der Ergebnisse zu beachten sind.

Die Literaturanalyse deckt nicht das gesamte Themenfeld ab. 
Als Ausgangspunkt wurden vier Basispublikationen genutzt und über eine Vorwärts- und Rückwärtssuche erweitert. 
Dadurch können relevante Arbeiten fehlen, etwa wenn sie nicht in den Zitationspfaden dieser Startmenge liegen oder andere Begriffe verwenden. 
Zusätzlich sind die Einordnung und Kategorisierung der Publikationen nicht vollständig objektiv. 
Auch bei klaren Kriterien bleibt eine gewisse Interpretationsabhängigkeit, zum Beispiel bei der Zuordnung zu Anwendungsfeldern oder bei der Ableitung der in Kapitel 4 beschriebenen Muster.

Die in Kapitel 4.3 abgeleiteten Use-Case-Muster sind bewusst generisch. 
Sie fassen wiederkehrende Aufgaben im Plattformbetrieb zusammen, bilden aber nicht alle Varianten und Randbedingungen ab, die in der Praxis auftreten. 
Das Bewertungskonzept in Kapitel 5 unterstützt eine strukturierte Priorisierung, bleibt jedoch qualitativ. 
Der tatsächliche Aufwand und der tatsächliche Nutzen hängen stark von der jeweiligen Datenlage, der Integrationsfähigkeit und den bestehenden Betriebsprozessen ab und wurden im Rahmen dieser Arbeit nicht durch Umsetzung oder Messungen validiert.

Die Anwendung des Bewertungskonzepts erfolgt anhand einer einzelnen Plattformumgebung und dient primär der Plausibilisierung des Vorgehens. 
Eine prototypische Implementierung war nicht Bestandteil der Arbeit, da insbesondere Datenqualität und Datenverfügbarkeit einen erheblichen Aufwand dargestellt hätten. 
Aussagen zu Effekten auf Kennzahlen wie Stabilität, Reaktionszeiten oder Kosten stellen daher begründete Einschätzungen dar und keine empirisch nachgewiesenen Ergebnisse.