\section{Limitationen}
\label{sec:limitationen}

Diese Arbeit basiert auf einer systematischen Mapping-Studie sowie auf der Anwendung der Ergebnisse in einem konkreten Plattformkontext.
Daraus ergeben sich methodische Grenzen, die bei der Einordnung der Ergebnisse zu berücksichtigen sind.

Eine systematische Mapping-Studie liefert keinen vollständigen Überblick über alle existierenden Arbeiten, sondern ordnet den aktuellen Forschungsstand strukturiert ein.
Der Fokus liegt auf der Identifikation von Trends, thematischen Schwerpunkten und wiederkehrenden Mustern.

Für ein schnelllebiges Themenfeld wie den Einsatz von \gls{ki} im Platform Engineering ist dies mit Einschränkungen verbunden.
Ein wesentlicher Teil der Entwicklung findet außerhalb klassischer wissenschaftlicher Publikationen statt, etwa in technischen Berichten, Projektblogs, Herstellerdokumentationen oder Open-Source-Projekten.
Diese graue Literatur ist im Rahmen der Mapping-Studie nur eingeschränkt berücksichtigt.

Die Ergebnisse bilden daher vor allem den aktuellen wissenschaftlichen Forschungsstand ab.
Praktische Umsetzungen können in einzelnen Bereichen bereits weiterentwickelt sein, als es die ausgewerteten Publikationen zeigen.

Die in Kapitel 4.3 abgeleiteten Use-Case-Muster sind bewusst generisch. 
Sie fassen wiederkehrende Aufgaben im Plattformbetrieb zusammen, bilden aber nicht alle Varianten und Randbedingungen ab, die in der Praxis auftreten. 
Das Bewertungskonzept in Kapitel 5 unterstützt eine strukturierte Priorisierung, bleibt jedoch qualitativ. 
Der tatsächliche Aufwand und Nutzen hängen stark von der jeweiligen Datenlage, der Integrationsfähigkeit und den bestehenden Betriebsprozessen ab.
Eine Validierung durch Umsetzung oder Messungen war nicht Bestandteil dieser Arbeit.

Die Anwendung des Bewertungskonzepts erfolgt anhand einer einzelnen Plattformumgebung und dient primär der Plausibilisierung des Vorgehens. 
Eine prototypische Implementierung war nicht Bestandteil der Arbeit, da insbesondere Datenqualität und Datenverfügbarkeit einen erheblichen Aufwand dargestellt hätten. 
Aussagen zu Effekten auf Kennzahlen wie Stabilität, Reaktionszeiten oder Kosten stellen daher begründete Einschätzungen dar und keine empirisch nachgewiesenen Ergebnisse.