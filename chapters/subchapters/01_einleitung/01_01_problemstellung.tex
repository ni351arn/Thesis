\section{Problemstellung}
\label{sec:problemstellung}

Trotz der flächendeckenden Verbreitung von Cloud-Native Architekturen und dem klaren Fokus vieler Unternehmen auf Automatisierung und Effizienzsteigerung, bleibt die Frage offen, wie Plattform-Teams konkret von fortgeschrittenen Automatisierungs- und KI-gestützten Ansätzen profitieren.
Während zahlreiche Unternehmen bereits erste KI-gestützte Tools in ihren Cloud-Native Umgebungen einsetzen, existiert bislang keine systematische und evidenzbasierte Analyse, welche Lösungen tatsächlich Mehrwert für Plattform-Teams schaffen. Die Forschung zu AIOps und KI im Software-Engineering ist zwar umfangreich, doch deren Bezug zu spezifischen Domänen des Plattform Engineerings wie CI/CD-Automatisierung, Infrastruktur-Management und Monitoring bleibt häufig unscharf. Zudem fehlen praxisnahe Untersuchungen, die auf realen Plattform-Stacks aufbauen und konkrete Pain Points der beteiligten Teams berücksichtigen. Daraus ergibt sich die Notwendigkeit, den aktuellen Stand der Forschung systematisch zu erfassen, mit industriellen Anforderungen abzugleichen und daraus Handlungsempfehlungen für die Integration von KI in bestehenden Plattformlandschaften abzuleiten. 
