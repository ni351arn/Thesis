\section{Problemstellung}
\label{sec:problemstellung}

Im Plattformbetrieb treffen Teams heute auf eine schnell wachsende Zahl technischer Optionen. 
Plattform-Stacks entwickeln sich laufend weiter und werden durch neue Komponenten, Schnittstellen und Betriebsmodelle erweitert. 
Branchenberichte beschreiben diese Entwicklung als anhaltenden Trend und zeigen zugleich die zunehmende Breite des Cloud-Native-Ökosystems \cite{bollmann-doddStateCloudNative2025}.

Zusätzlich werden zunehmend Werkzeuge auf Basis Künstlicher Intelligenz (KI) im Plattformkontext eingesetzt.
Diese Ansätze unterscheiden sich jedoch stark im Reifegrad und erfordern je nach Lösung unterschiedliche Integrations- und Betriebsaufwände. 
Fachanalysen zeigen, dass Teams dadurch vermehrt vor schwer vergleichbaren Tool- und Strategieentscheidungen stehen \cite{InfoQCloudDevOps2025}.

In der Praxis fehlt Plattformen als Service damit eine belastbare Grundlage, um relevante Technologieoptionen systematisch zu vergleichen.
Eine Evaluierung über viele Prototypen ist meist nicht realistisch. 
Dadurch entstehen Unsicherheit und Reibungsverluste bei Auswahl- und Priorisierungsentscheidungen.