\chapter{Einleitung}
\label{ch:einleitung}
Cloud-Native Technologien und Platform Engineering-Ansätze prägen zunehmend den Aufbau und Betrieb moderner Softwareplattformen. 
In den vergangenen Jahren hat sich ein breites Spektrum kommerzieller Produkte und Open-Source-Lösungen etabliert, das zentrale Aufgaben rund um Cloud-Native Plattformen adressiert.
Diese Entwicklungen verlaufen in hohen Innovationszyklen und führen zu stetig neuen Anforderungen an Architektur und Betrieb.

\gls{ki}-gestützte Werkzeuge werden zunehmend zur Unterstützung betrieblicher Plattformprozesse eingesetzt.
Sie versprechen, komplexe Betriebsdaten auszuwerten, Abläufe zu automatisieren und Entscheidungen im Plattformbetrieb zu unterstützen. 
Die Dynamik und Vielfalt dieser Ansätze ist hoch und wächst zum Teil schneller als die reinen Plattformtechnologien selbst.

Marktanalysen zeigen, wie rasant sich diese Felder entwickeln.
Laut Fortune Business Insights werden bis 2025 etwa 95 \% aller neuen digitalen Workloads auf Cloud-Native Plattformen betrieben, verglichen mit rund 30 \% im Jahr 2021 \cite{nothbaumCloudnativeErklaertArchitektur2025}.

Vor dem Hintergrund dieses Trends stellt sich die Frage, wie der Einsatz von \gls{ki}-Technologien im Cloud-Native Platform Engineering sinnvoll eingeordnet und bewertet werden kann.

\section{Problemstellung}
\label{sec:problemstellung}

Trotz der flächendeckenden Verbreitung von Cloud-Native Architekturen und dem klaren Fokus vieler Unternehmen auf Automatisierung und Effizienzsteigerung, bleibt die Frage offen, wie Plattform-Teams konkret von fortgeschrittenen Automatisierungs- und KI-gestützten Ansätzen profitieren.
Während zahlreiche Unternehmen bereits erste KI-gestützte Tools in ihren Cloud-Native Umgebungen einsetzen, existiert bislang keine systematische und evidenzbasierte Analyse, welche Lösungen tatsächlich Mehrwert für Plattform-Teams schaffen. Die Forschung zu AIOps und KI im Software-Engineering ist zwar umfangreich, doch deren Bezug zu spezifischen Domänen des Plattform Engineerings wie CI/CD-Automatisierung, Infrastruktur-Management und Monitoring bleibt häufig unscharf. Zudem fehlen praxisnahe Untersuchungen, die auf realen Plattform-Stacks aufbauen und konkrete Pain Points der beteiligten Teams berücksichtigen. Daraus ergibt sich die Notwendigkeit, den aktuellen Stand der Forschung systematisch zu erfassen, mit industriellen Anforderungen abzugleichen und daraus Handlungsempfehlungen für die Integration von KI in bestehenden Plattformlandschaften abzuleiten. 

\section{Zielsetzung der Arbeit}
\label{sec:zielsetzung}

Ziel dieser Arbeit ist es, eine wissenschaftlich fundierte Grundlage für die Integration von Künstlicher Intelligenz in Cloud-Native-Plattform-Engineering-Umgebung zu schaffen. Hierzu wird zunächst im Rahmen einer systematischen Mapping Study untersucht, welche aktuellen KI-Ansätze im Plattform Engineering existieren und wo genau sie Anwendung finden. Darauf aufbauend erfolgt eine Analyse der Bosch Digital Manufacturing Plattform, um bestehende Herausforderungen zu identifizieren und potenzielle Einsatzfelder von KI-Technologien zu evaluieren. 
Das Ende der Arbeit setzt sich aus der Entwicklung eines praxistauglichen Frameworks zur Bewertung von KI-Potenzialen in Plattformumgebung sowie mit einer prototypischen Implementierung zusammen. Damit leistet die Arbeit sowohl einen wissenschaftlichen als auch einen praktischen Beitrag zur Weiterentwicklung moderner Plattform-Engineering Praktiken. 

\section{Aufbau der Arbeit}
\label{sec:aufbau}

\autoref{ch:grundlagenUndVerwandteArbeiten} stellt die theoretischen Grundlagen sowie relevante verwandte Arbeiten aus den Bereichen Cloud-Native Technologien, Platform Engineering und \gls{ki} vor. 
In \autoref{ch:methoden} wird das methodische Vorgehen beschrieben, insbesondere die Durchführung der systematischen Mapping-Studie.

Die Ergebnisse der Literaturanalyse werden in \autoref{ch:literaturErgebnisse} dargestellt und in einem konzeptionellen Matching-Framework zusammengeführt. 
\autoref{ch:anwendungLiteraturrecherche} überträgt dieses Framework auf die \gls{bmlp} und wendet es auf identifizierte Problemfelder an.

Abschließend werden die Ergebnisse in \autoref{ch:diskussion} diskutiert.
\autoref{ch:zusammenfassung} fasst die Arbeit zusammen und gibt einen Ausblick auf weiterführende Fragestellungen.
