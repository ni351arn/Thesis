\chapter{Einleitung}
\label{ch:einleitung}

Die Einleitung führt in das Thema der Arbeit ein und gibt einen Überblick über die behandelten Inhalte.

\section{Problemstellung}
\label{sec:problemstellung}

Hier wird die Problemstellung beschrieben, die den Anlass der Arbeit bildet.

\section{Zielsetzung der Arbeit}
\label{sec:zielsetzung}

Die Ziele der Arbeit werden hier konkret formuliert und gegenüber der Problemstellung abgegrenzt.

\section{Aufbau der Arbeit}
\label{sec:aufbau}

Dieses Abschnitt gibt einen Überblick über den strukturellen Aufbau der Arbeit. 

In Kapitel~\ref{ch:grundlagen} werden die theoretischen Grundlagen sowie verwandte Arbeiten erläutert. 
Kapitel~\ref{ch:methoden} beschreibt das methodische Vorgehen. 
Kapitel~\ref{ch:ergebnisse} stellt die Ergebnisse dar, gefolgt von Kapitel~\ref{ch:konzept} (theoretisches Konzept) und Kapitel~\ref{ch:bdmp} (Analyse der Bosch Digital Manufacturing Platform). 
Kapitel~\ref{ch:diskussion} diskutiert die Ergebnisse, Kapitel~\ref{ch:handlungsempfehlung} gibt Handlungsempfehlungen und Kapitel~\ref{ch:zusammenfassung} schließt mit Zusammenfassung und Ausblick.
