\chapter{Grundlagen und verwandte Arbeiten}
\label{ch:grundlagen}

In diesem Kapitel werden die theoretischen Grundlagen dargestellt, die für das Verständnis der Arbeit notwendig sind, sowie verwandte Arbeiten eingeordnet.

\section{Grundlagen}
\label{sec:grundlagen}

\subsection{Cloud Native Technologien und Platform Engineering}
\label{subsec:cloud-native-platform-engineering}

Ein Überblick über Cloud-Native-Prinzipien (Containment, Orchestrierung, Microservices) und die Rolle von Platform Engineering.

\subsection{DevOps, CI/CD und Plattformbetrieb}
\label{subsec:devops-cicd-plattformbetrieb}

DevOps-Praktiken, CI/CD-Pipelines und Betriebsaspekte für zuverlässige Softwarebereitstellung.

\subsection{Künstliche Intelligenz im Software-Engineering}
\label{subsec:ki-im-se}

Anwendungsfelder und Nutzen von KI innerhalb des Software-Engineering-Lebenszyklus.

\subsection{AIOps und verwandte Konzepte}
\label{subsec:aiops}

Definition und Abgrenzung von AIOps, Monitoring, Observability und autonomen Betriebsansätzen.

\section{Verwandte Arbeiten}
\label{sec:verwandte-arbeiten}

Relevante wissenschaftliche Arbeiten, Industriereports und State-of-the-Art-Übersichten werden beschrieben und kritisch eingeordnet.
