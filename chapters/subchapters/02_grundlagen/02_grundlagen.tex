\chapter{Grundlagen und verwandte Arbeiten}
\label{ch:grundlagen}

In diesem Kapitel werden die theoretischen Grundlagen dargestellt, die für das Verständnis der Arbeit notwendig sind, sowie verwandte Arbeiten eingeordnet.

\section{Grundlagen}
\label{sec:grundlagen}

\subsection{Cloud Native Technologien und Platform Engineering}
\label{subsec:cloud-native-platform-engineering}
Cloud Native (CN) gewann ab 2013 insbesondere durch die Etablierung von Containertechnologien bis hin zu Kubernetes an Bedeutung. 
Im Kern beschreibt CN heute weniger eine einzelne Technologie, sondern ein Zielbild für modular aufgebaute Systeme (z. B. Microservices), die sich gut bereitstellen, skalieren und resilient betreiben lassen. 
Die CNCF fasst Cloud Native als Ansatz für skalierbare Anwendungen in dynamischen Cloud-Umgebungen zusammen \cite{zaaloukCLOUDNATIVEARTIFICIAL}.

Kubernetes hat sich dabei als zentrale Plattform etabliert. Es dient als portable, erweiterbare Open-Source-Lösung zur Verwaltung containerisierter Workloads und unterstützt insbesondere deklarative Konfiguration und Automatisierung \cite{Overview}.

Im Platform Engineering werden diese Grundlagen genutzt, um Entwicklerteams über eine interne Plattform (Internal Developer Platform) standardisierte, sichere Self-Service-Capabilities bereitzustellen und damit Lieferfähigkeit, Compliance und Betrieb zu verbessern \cite{juliakmWhatPlatformEngineering}.

\subsection{DevOps und CI/CD}
\label{subsec:devops-ci-cd}
Development und Operations (DevOps) ist ein kultureller und organisatorischer Ansatz, bei dem Entwicklung und Betrieb gemeinsam Verantwortung für den gesamten Software-Lebenszyklus tragen. 
Ziel ist es, Zusammenarbeit und Kommunikation zwischen den beteiligten Rollen zu stärken und Abläufe so zu unterstützen, dass Änderungen zuverlässig und kontinuierlich bereitgestellt werden können \cite{WasIstDevOps, atlassianWasIstDevOps}.

Continuous Integration (CI) und Continuous Delivery/Deployment (CD) sind zentrale Praktiken innerhalb von DevOps.
CI beschreibt die regelmäßige und automatisierte Integration von Codeänderungen in ein gemeinsames Repository inklusive Build und Tests.
CD baut darauf auf und automatisiert die Auslieferung. 
Continuous Delivery endet vor dem automatischen Produktiv-Deployment, während Continuous Deployment diesen Schritt ebenfalls automatisiert \cite{WasIstCI, rzigEmpiricalAnalysisCI2024}.


\subsection{Lernparadigmen des maschinellen Lernens}
\label{subsec:lernparadigmen}
Die grundlegenden Lernparadigmen des maschinellen Lernens lassen sich in Supervised Learning (SL), Unsupervised Learning (UL) und Reinforcement Learning (RL) einteilen.

SL trainiert ein Modell anhand gelabelter Daten, sodass es eine Abbildung von Eingaben auf eine Ziel- bzw. Antwortvariable lernt \cite{enemosahEnhancingDevOpsEfficiency2025,valkenborgSupervisedLearning2023}.

UL arbeitet mit ungelabelten Daten und zielt darauf ab, darin Strukturen, Cluster oder Auffälligkeiten zu erkennen, ohne dass eine explizite Zielvariable vorgegeben ist \cite{enemosahEnhancingDevOpsEfficiency2025,valkenborgUnsupervisedLearning2023}.

RL beschreibt Verfahren, bei denen ein Agent durch Interaktion mit einer Umgebung Handlungen lernt, um eine langfristige (kumulative) Belohnung zu maximieren \cite{enemosahEnhancingDevOpsEfficiency2025,ghasemiIntroductionReinforcementLearning2024}.

Die drei Lernparadigmen bilden die Grundlage, während die konkrete Umsetzung in der Praxis typischerweise über spezifische Algorithmen und Methoden erfolgt.
Tabelle \ref{tab:tabelleMethodenAlgorithmen} fasst die in dieser Arbeit betrachteten Algorithmusklassen zusammen und schafft damit eine einheitliche begriffliche Grundlage für die spätere Auswertung.

\begin{table}[htbp]
  \centering
  \caption{Beschreibung Algorithmen und Methoden}
  \label{tab:tabelleMethodenAlgorithmen}
  \footnotesize
  \renewcommand{\arraystretch}{1.15}
  \begin{tabularx}{\linewidth}{@{}lY@{}}
    \toprule
    \textbf{Algorithmen und Methoden} & \textbf{Beschreibung} \\
    \midrule
    \glsreset{dl}\gls{dl} / \glsreset{nn}\gls{nn} & Erfassen komplexe Muster in Daten und eignen sich besonders für unstrukturierte Eingaben wie Log- oder Monitoring-Daten. \\
    Ensemble und Baum-basiert & Kombinieren mehrere Modelle zur Steigerung der Vorhersagegenauigkeit und verarbeiten große sowie semi-strukturierte Datensätze effizient. \\
    Klassische Klassifikation / Regression & Traditionelle ML-Modelle zur Vorhersage von Mustern oder Ereignissen auf Basis strukturierter Daten. \\
    Clustering & Gruppieren Datenpunkte ohne Labels in inhaltlich ähnliche Cluster und unterstützen dadurch Mustererkennung und Anomaliedetektion. \\
    \bottomrule
  \end{tabularx}
\end{table}

\FloatBarrier 

In der Analyse wurden Lernparadigmen zudem auch dann zugeordnet, wenn sie in den Publikationen nicht explizit benannt waren, sondern nur über typische Aufgabenstellungen erkennbar wurden. 
Solche Aufgabenstellungen werden in den in diesem Abschnitt zitierten Quellen ausführlicher beschrieben.

\subsection{AIOps und verwandte Konzepte}
\label{subsec:aiops}

Definition und Abgrenzung von AIOps, Monitoring, Observability und autonomen Betriebsansätzen.

\section{Verwandte Arbeiten}
\label{sec:verwandte-arbeiten}

Relevante wissenschaftliche Arbeiten, Industriereports und State-of-the-Art-Übersichten werden beschrieben und kritisch eingeordnet.
