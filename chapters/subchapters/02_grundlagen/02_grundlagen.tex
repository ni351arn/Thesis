\chapter{Grundlagen und verwandte Arbeiten}
\label{ch:grundlagen}

In diesem Kapitel werden die theoretischen Grundlagen dargestellt, die für das Verständnis der Arbeit notwendig sind, sowie verwandte Arbeiten eingeordnet.

\section{Grundlagen}
\label{sec:grundlagen}

\subsection{Cloud Native Technologien und Platform Engineering}
\label{subsec:cloud-native-platform-engineering}

Ein Überblick über Cloud-Native-Prinzipien (Containment, Orchestrierung, Microservices) und die Rolle von Platform Engineering.

\subsection{Zentrale Anwendungsbereiche und Herausforderungen}
\label{subsec:useCasesUndHerausforderungen}

DevOps-Praktiken, CI/CD-Pipelines und Betriebsaspekte für zuverlässige Softwarebereitstellung.

\subsection{Lernparadigmen des maschinellen Lernens}
\label{subsec:lernparadigmen}
Die drei Formen des maschinellen Lernens setzen sich aus Supervised Learning, Unsupervised Learning und Reinforcement Learning zusammen.
Diese drei Lernparadigmen werden nachfolgend genauer beschrieben auf Basis der Defintionen von \textcite{enemosahEnhancingDevOpsEfficiency2025}.

Supervised Learning beschreibt Verfahren, bei denen Modelle anhand markierter Daten (labeled data) trainiert werden. 
Ziel ist es, auf Basis bekannter Eingabe-Ausgabe-Paare präzise Vorhersagen zu treffen, etwa für Klassifikations- oder Priorisierungsaufgaben.

Unsupervised Learning arbeitet mit unmarkierten Daten und dient der Erkennung verborgener Muster oder Strukturen. 
Typische Verfahren wie Clustering gruppieren ähnliche Datenpunkte und unterstützen so beispielsweise die Analyse von Systemlogs oder die Identifikation von Anomalien.

Reinforcement Learning nutzt ein belohnungsbasiertes Lernprinzip, bei dem ein Agent durch Interaktion mit seiner Umgebung optimale Strategien erlernt. Im Kontext der betrachteten Literatur wird RL u. a. zur dynamischen Optimierung von Ressourcenallokationen eingesetzt, beispielsweise für adaptive Skalierungs- oder Rollback-Entscheidungen \cite{enemosahEnhancingDevOpsEfficiency2025}.

\begin{table}[htbp]
  \centering
  \caption{Beschreibung Algorithmen und Methoden}
  \label{tab:tabelleMethodenAlgorithmen}
  \footnotesize
  \renewcommand{\arraystretch}{1.15}
  \begin{tabularx}{\linewidth}{@{}lY@{}}
    \toprule
    \textbf{Algorithmen und Methoden} & \textbf{Beschreibung} \\
    \midrule
    Deep Learning (DL)/ Neuronale Netze (NN) & Erfassen komplexe Muster in Daten und eignen sich besonders für unstrukturierte Eingaben wie Log- oder Monitoring-Daten. \\
    Ensemble und Baum-basiert & Kombinieren mehrere Modelle zur Steigerung der Vorhersagegenaugikeit und verarbeiten große sowie semi-strukturiterte Datensätze effizient. \\
    Klassische Klassifikation/ Regression & Traditionelle ML-Modelle zur Vorhersage von Mustern oder Ereignnissen auf Basis strukturierter Daten. \\
    Clustering & Gruppieren Datenpunkte ohne Labels in inhaltlich ähnliche CLuster und unterstützen dadurch Musterekrennung und Anomaliedetektion. \\
    \bottomrule
  \end{tabularx}
\end{table}

\FloatBarrier 

\subsection{AIOps und verwandte Konzepte}
\label{subsec:aiops}

Definition und Abgrenzung von AIOps, Monitoring, Observability und autonomen Betriebsansätzen.

\section{Verwandte Arbeiten}
\label{sec:verwandte-arbeiten}

Relevante wissenschaftliche Arbeiten, Industriereports und State-of-the-Art-Übersichten werden beschrieben und kritisch eingeordnet.
