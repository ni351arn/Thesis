\chapter*{Abstract}
\setheader{Abstract}
\phantomsection
\addcontentsline{toc}{chapter}{Abstract}

Cloud-Native Platform Engineering ist durch hohe Dynamik und steigende Systemkomplexität geprägt.
Der Einsatz von Künstlicher Intelligenz (KI) nimmt zu, bleibt jedoch in vielen Fällen schwer vergleichbar und einzuordnen.
Eine strukturierte Entscheidungsgrundlage für den gezielten Einsatz solcher Ansätze fehlt bislang.
Die Arbeit basiert auf einer systematischen Mapping Studie mit klar definierten Such- und Auswahlkriterien.
Die identifizierten Studien werden quantitativ ausgewertet und entlang zentraler Merkmale strukturiert zugeordnet.
Die Analyse zeigt vier dominante Anwendungsfelder im Ressourcenmanagement, im Betrieb, in CI/CD-nahen Szenarien sowie im Sicherheitskontext.
Über alle Felder hinweg treten wiederkehrende Herausforderungen auf, insbesondere in Bezug auf Datenqualität, Integration, Governance und Kosten.
Auf Basis der Ergebnisse wird ein Bewertungskonzept zur systematischen Einordnung von KI-Anwendungsfällen entwickelt.
Dieses trennt Implementierungsaufwand und operativen Mehrwert und ergänzt diese um praxisrelevante Zusatzdimensionen.
Das Konzept wird auf der Bosch Digital Manufacturing Plattform angewendet, um ein KI-Einsatzszenario zu bewerten.