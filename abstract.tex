\chapter*{Abstract}
\setheader{Abstract}
\phantomsection
\addcontentsline{toc}{chapter}{Abstract}

Cloud-Native Platform Engineering ist durch häufige Änderungen an Architektur, Betrieb und Werkzeugen sowie steigende Systemkomplexität geprägt.
Der Einsatz von künstlicher Intelligenz nimmt zu, bleibt jedoch oft schwer vergleichbar und einzuordnen.
Eine strukturierte Grundlage für den gezielten Einsatz künstlicher Intelligenz im Plattformbetrieb fehlt bislang.
Die Arbeit basiert auf einer systematischen Mapping-Studie mit definierten Such- und Auswahlkriterien.
Die identifizierten Studien werden quantitativ ausgewertet und anhand zentraler Merkmale kategorisiert.
Daraus ergeben sich vier wiederkehrende Anwendungsfelder: Ressourcenmanagement, Plattformbetrieb, Build- und Bereitstellungsprozesse sowie Sicherheit.
Über alle Felder hinweg treten ähnliche Herausforderungen auf, insbesondere Datenqualität, Integration, Governance und Kosten.
Auf Basis der Ergebnisse wird ein Bewertungskonzept entwickelt, um Anwendungsfälle künstlicher Intelligenz einzuordnen.
Es trennt Implementierungsaufwand und operativen Mehrwert und ergänzt praxisrelevante Zusatzdimensionen.
Das Konzept wird auf der Bosch Digital Manufacturing Platform angewendet, um ein Einsatzszenario zu bewerten.