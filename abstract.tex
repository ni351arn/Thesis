\chapter*{Abstract}
\setheader{Abstract}
\phantomsection
\addcontentsline{toc}{chapter}{Abstract}

Cloud-Native Platform Engineering ist durch hohe Dynamik und steigende Systemkomplexität geprägt.
Der Einsatz von Künstlicher Intelligenz (KI) nimmt zu, bleibt jedoch oft schwer vergleichbar und einzuordnen.
Eine strukturierte Grundlage für den gezielten KI-Einsatz fehlt bislang.
Die Arbeit basiert auf einer systematischen Mapping-Studie mit definierten Such- und Auswahlkriterien.
Die identifizierten Studien werden quantitativ ausgewertet und anhand zentraler Merkmale kategorisiert.
Die Ergebnisse zeigen vier wiederkehrende Anwendungsfelder: Ressourcenmanagement, Plattformbetrieb, CI/CD-Prozesse und Sicherheit.
Über alle Felder hinweg treten ähnliche Herausforderungen auf, insbesondere Datenqualität, Integration, Governance und Kosten.
Auf Basis der Ergebnisse wird ein Bewertungskonzept zur Einordnung von KI-Anwendungsfällen entwickelt.
Es trennt Implementierungsaufwand und operativen Mehrwert und ergänzt praxisrelevante Zusatzdimensionen.
Das Konzept wird auf der Bosch Digital Manufacturing Plattform angewendet, um ein KI-Einsatzszenario zu bewerten.