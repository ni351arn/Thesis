\begin{tikzpicture}[font=\small, node distance=14mm, >=latex]
\tikzstyle{box} = [draw, rounded corners, align=center,
minimum width=32mm, minimum height=10mm]

% Linke Spalte
\node[box] (gitops) {GitOps-Repo};
\node[box, below=22mm of gitops] (runbooks) {Runbooks\\{\scriptsize Fehlerbilder + Lösungsschritte}};
\node[box, below=22mm of runbooks] (k8sgpt) {K8sGPT\\Analyse};

% Mittlere Spalte
\node[box, right=24mm of gitops] (k8s) {Kubernetes-Cluster};
\node[box, below=22mm of k8s] (opsdata) {Betriebsdaten\\{\scriptsize Ereignisse / Protokolle}};

% Rechte Spalte (max. 2 Boxen rechts)
\node[box, right=44mm of k8sgpt] (llm) {Sprachmodell};
\node[box, below=22mm of llm] (teams) {Teams};

% Fluesse
\draw[->, dashed] (gitops) -- node[above]{Sollzustand} node[below]{(Pull)} (k8s);

\draw[->] (k8s) -- node[right]{Ereignisse / Protokolle} (opsdata);

\draw[->] (opsdata) -- node[fill=white, inner sep=2pt, below]{Datenbasis} (k8sgpt);

\draw[->, dashed] (runbooks) -- node[right]{Kontext} (k8sgpt);

\draw[<->] (k8sgpt) -- node[above]{Erklärung / Hinweise} (llm);

\draw[->] (k8sgpt) -- node[right]{Handlungshinweise} (teams);

% Optionaler direkter Pfeil vom Cluster zur Analyse (wenn du ihn willst)
% \draw[->, dashed] (k8s) |- node[pos=0.25, right]{Clusterzustand} (k8sgpt);

% Hinweis rechts neben K8sGPT
\node[align=right, left=4mm of k8sgpt]
{\scriptsize nicht im kritischen Pfad\\
\scriptsize nur unterstützend};

\end{tikzpicture}
