\begin{tikzpicture}[font=\small, node distance=14mm, >=latex]
\tikzstyle{box} = [draw, rounded corners, align=center,
minimum width=34mm, minimum height=10mm]

% Linke Spalte
\node[box] (gitops) {GitOps-Repo\\{\scriptsize Sollzustand}};
\node[box, below=22mm of gitops] (runbooks) {Runbooks\\{\scriptsize Fehlerbilder + Lösungsschritte}};
\node[box, below=22mm of runbooks] (k8sgpt) {K8sGPT\\{\scriptsize Analyse}};

% Mittlere Spalte
\node[box, right=26mm of gitops] (k8s) {Kubernetes-Cluster};
\node[box, below=22mm of k8s] (obs) {Beobachtbarkeit\\{\scriptsize Metriken / Logs / Events}};

% Rechte Spalte
\node[box, right=44mm of k8sgpt] (llm) {Sprachmodell\\{\scriptsize optional}};
\node[box, below=22mm of llm] (user) {Benutzer};

% GitOps: Pull vom Cluster (kein Push)
\draw[->] (k8s) -- node[midway, above, fill=white, inner sep=2pt]{Pull} (gitops);

% Betriebsdatenfluss
\draw[->] (k8s) -- node[midway, right, fill=white, inner sep=2pt]{Betriebsdaten} (obs);
\draw[->] (obs) -- node[midway, below, fill=white, inner sep=2pt]{Datenbasis} (k8sgpt);

% Runbooks -> K8sGPT (gerade, gestrichelt)
\draw[->, dashed] (runbooks) -- node[midway, right, fill=white, inner sep=2pt]{Kontext} (k8sgpt);

% GitOps -> K8sGPT (links um Runbooks herum)
\draw[->, dashed] (gitops.south) -- ++(-15mm,0) -- ++(0,-54mm) -- ++(15mm,0)
node[pos=0.50, left, fill=white, inner sep=2pt]{Sollzustand};

% Sprachmodell nur zur Aufbereitung/Erklärung
\draw[<->, dashed] (k8sgpt) -- node[midway, above, fill=white, inner sep=2pt]{Erklärung / Aufbereitung} (llm);

% Ergebnis
\draw[->] (k8sgpt.east) -- (user.west)
node[midway, above, fill=white, inner sep=2pt]{Hinweise};

\end{tikzpicture}
